\documentclass[envcountsect,runningheads,a4paper]{llncs}
\usepackage{geometry} \geometry{margin=2cm} 


%\graphicspath{{./graphics/}}%helpful if your graphic files are in another directory

\makeatletter
\renewcommand{\@Opargbegintheorem}[4]{%
	#4\trivlist\item[\hskip\labelsep{#3#2\@thmcounterend}]}
\makeatother


\newif\ifreport
\reporttrue
\reportfalse

%\bibliographystyle{splncs04}


%\usepackage{bm}
%\usepackage{tocloft}
%\usepackage{graphicx}
%\usepackage{amsmath}
\usepackage{amssymb}
%\usepackage{amsthm}
\usepackage{mathtools}
%\usepackage{amsfonts}
%\usepackage{yfonts}
\usepackage{hyperref}
\usepackage{cleveref}
%\usepackage{newclude}
\usepackage{mathpartir}
\usepackage[utf8]{inputenc}
\usepackage{tikz}
\usetikzlibrary{matrix, arrows.meta, positioning, calc}
\usepackage[usestackEOL]{stackengine}
\usepackage{proof-at-the-end}
\usepackage[all, cmtip]{xy}
\newcommand{\id}[1]{\mathsf{id}_{#1}}


\DeclareFontFamily{OT1}{pzc}{}
\DeclareFontShape{OT1}{pzc}{m}{it}{<-> s * [1.200] pzcmi7t}{}
\DeclareMathAlphabet{\mathpzc}{OT1}{pzc}{m}{it}
\newcommand{\mathscr}{\mathpzc}
\newcommand{\vin}{\rotatebox[origin=c]{-90}{$\dashv$}}
\newcommand{\catname}[1]{{\normalfont\textbf{#1}}}
\newcommand{\fix}{\mathsf{fix}}
\newcommand{\mon}{\mathsf{mon}}
\newcommand{\avr}{\overrightarrow{\mathsf{R}}}
\newcommand{\lgh}{\mathsf{lg}}
\newcommand{\str}{\mathsf{start}}
\newcommand{\nd}{\mathsf{end}}
\newcommand{\devr}{\overleftarrow{\mathsf{R}}}
\newcommand{\pth}{\mathsf{path}}
\newcommand{\pavr}{\overrightarrow{\mathsf{path}}}
\newcommand{\pevr}{\overleftarrow{\mathsf{path}}}
\newcommand{\suc}{\mathsf{suc}}
\newcommand{\pre}{\mathsf{pre}}
\newcommand{\lfp}{\mathsf{lfp}}
\newcommand{\gfp}{\mathsf{gfp}}
\newcommand{\slice}[2]{(\catname{#1}\downarrow{#2})}
\newcommand{\norm}[1]{\left\lVert#1\right\rVert}
\newcommand{\abs}[1]{\left\lvert#1\right\rvert}
\newcommand{\nat}{\xrightarrow{\bullet}}
\newcommand{\im}[1]{\mathrm{Im}(#1)}
\newcommand{\chain}[1]{\catname{Ch}(\catname{#1})}
\newcommand{\scat}[1]{\mathsf{suc}_{\gamma_{#1}}}
\newcommand{\scato}{\mathsf{suc}}
\newcommand{\diam}[1]{\mathsf{pre}_{\gamma_{#1}}}
\newcommand{\diamo}{\mathsf{pre}}
\newcommand{\homo}[1]{\mathcal{K}(\catname{#1})}
\newcommand{\colim}[0]{\mathrm{colim}}
\newcommand{\rela}[2]{\mathscr{Rel}_\mathcal{#2}(\catname{#1})}
\newcommand{\sub}[1]{\mathscr{Sub}_{\catname{#1}}}
\newcommand{\msub}[1]{\mathscr{M\text{-}Sub}_{\catname{#1}}}
\newcommand{\dom}{\mathrm{dom}}
\newcommand{\cod}{\mathrm{cod}}


\newcommand{\acs}{\catname{aCS}}
\newcommand{\cs}{\catname{CS}}
\newcommand{\ecs}{\catname{eCS}}
\newcommand{\eacs}{\catname{eaCS}}
\newcommand{\ics}{\catname{sCS}}
\newcommand{\tcs}{\catname{tCS}}
\newcommand{\iacs}{\catname{saCS}}
\newcommand{\sg}{\catname{SGraph}}
\newcommand{\rsg}{\catname{Refl}}
\newcommand{\tsg}{\catname{Trans}}
\newcommand{\pro}{\catname{Pre}}
\newcommand{\al}{\catname{Ale}}
\newcommand{\tacs}{\catname{taCS}}
\newcommand{\co}[1]{\{\abs{#1}\}}
\newcommand{\eq}[3]{[#1 = #2]_{#3}}
\newcommand{\equ}[2]{[#1 \equiv #2]}
\newcommand{\rel}[3]{\frak{R}_{\mathcal{#3}}(#1, #2)}
\newcommand{\sat}[0]{\mathsf{cl}}
\newcommand{\cov}{\vartriangleleft}
\newcommand{\pos}[1]{\mathsf{Pos}_{\mathfrak{#1}}}
\newcommand{\rk}[2]{\mathrm{rk}_{#1}({#2})}
\newcommand{\lind}[1]{\mathcal{L}({#1})}
\newcommand{\reach}[1]{\mathsf{reach}_{#1}}
\newcommand{\integ}[1]{\mathsf{int}_{#1}}
\newcommand{\class}[1]{\catname{Cl}({#1})}
\newcommand{\propo}[0]{\mathsf{Prop}}

\newcommand{\formu}[1]{\mathbf{Form}_{#1}}
\newcommand{\ex}[1]{\mathscr{ex(#1)}}
\newcommand{\inte}[1]{\llbracket{#1}\rrbracket}
\newcommand{\sint}[1]{\catname{Syn}^{fo}(#1)}
\newcommand{\fram}[0]{\catname{Frm}}
\newcommand{\topo}[0]{\catname{Top}}
\newcommand{\set}[0]{\catname{Set}}
\newcommand{\escape}[3]{#2 \mathfrak{E}_{#1} #3}
\newcommand{\es}[3]{ \mathsf{isER}_{#1}(p,#2, #3)}
\newcommand{\esca}[4]{ \mathsf{eRoute}_{#1} (#4, #2, #3)}
\newcommand{\er}[3]{\mathsf{EscR}_{#1}({#2},{#3})}
\newcommand{\coalg}[1]{\catname{CoAlg}(\mathscr{#1})}
\newcommand{\modd}[0]{\catname{Mod}}
\newcommand{\des}[1]{Des({#1})}
\newcommand{\prof}[2]{\mathsf{Proof}({#1},{#2})}
\newcommand{\ab}[0]{=-Adj}
\newcommand{\hoclass}[1]{\catname{HoCl}({#1})}
\newcommand{\term}[2]{\mathbf{Term}({#1}, {#2})}
\newcommand{\typ}[1]{\mathsf{Type}({#1})}
\newcommand{\sur}[3]{{#2}\mathfrak{S}_{#1}{#3}}
\newcommand{\fuz}[0]{\mathscr{FzSub}}
\newcommand{\ba}[0]{=-E}
\newcommand{\conte}[0]{\hspace{1pt}|\hspace{1pt}}

\newcommand\functorop[1][l]{\csname#1functor\endcsname}
\newcommand\lfunctorop[3]{%
	\setbox0=\hbox{$#2$}%
	\kern\wd0%
	\ensurestackMath{\Centerstack[c]{#1\\ \mathllap{#2\;\,}\mathclap{\DownArrow}\\#3}}%
}
\newcommand\rfunctorop[3]{%
	\setbox0=\hbox{$#2$}%
	\ensurestackMath{\Centerstack[c]{#1\\\mathclap{\UpArrow}\mathrlap{\,\;#2}\\#3}}%
	\kern\wd0%
}
\newcommand\functoropmapsto{\mathrel{\ensurestackMath{\Centerstack[c]{\longmapsto\\ \\\longmapsto}}}}
\setstackgap{L}{1.3\normalbaselineskip}
\newcommand\UpArrow{\rotatebox[origin=c]{90}{$\longrightarrow$\,}}
\newcommand\DownArrow{\rotatebox[origin=c]{-90}{$\longrightarrow$\,}}

\newcommand{\qedhere}{\qed}

\newcommand\functor[1][l]{\csname#1functor\endcsname}
\newcommand\lfunctor[3]{%
	\setbox0=\hbox{$#2$}%
	\kern\wd0%
	\ensurestackMath{\Centerstack[c]{#1\\ \mathllap{#2\;\,}\mathclap{\DownArrow}\\#3}}%
}
\newcommand\rfunctor[3]{%
	\setbox0=\hbox{$#2$}%
	\ensurestackMath{\Centerstack[c]{#1\\\mathclap{\DownArrow}\mathrlap{\,\;#2}\\#3}}%
	\kern\wd0%
}
\newcommand\functormapsto{\mathrel{\ensurestackMath{\Centerstack[c]{\longmapsto\\ \\\longmapsto}}}}
\setstackgap{L}{1.3\normalbaselineskip}


 %Proof-at-the-end
 %Theorems
%\newtheorem{thm1}[theorem]{Theorem}
%\newtheorem*{thm2}{Theorem}
%\providecommand*\thmautorefname{Theorem}
% Lemmata
%\newtheorem{lem}[theorem]{Lemma}
%\newtheorem*{lem*}{Lemma}
%\providecommand*\lemmaautorefname{Lemma}
% Proposition
%\newtheorem{prop}[theorem]{Proposition}
%\newtheorem*{prop*}{Proposition}
%\providecommand*\lemmaautorefname{Proposition}
% end Proof-at-the-end

%%%%%%%%

\title{Spatio-temporal logic on finite traces}
%\titlerunning{Dummy short title} %TODO optional, please use if title is longer than one line
% If the paper title is too long for the running head, you can set  an abbreviated paper title here
%

\author{%
	Davide Castelnovo\inst{1} \and
	Fabio Gadducci\inst{2} \and
	Marino Miculan\inst{1}
}%\footnote{Optional footnote, e.g. to mark corresponding author}}

\authorrunning{D.~Castelnovo, F. Gadducci, M.~Miculan}

\institute{Dept.~of Mathematics, Computer Science and Physics, University of Udine, Italy
	\and
	Dept.~of Computer Science, University of Pisa, Italy
	\\
	\email{castelnovod@gmail.com}, 
	\email{fabio.gadducci@unipi.it}, 
	\email{marino.miculan@uniud.it}
}


\begin{document}

\maketitle
	
\begin{abstract}
	TO DO: vendere 'sta roba ma in breve
\end{abstract}
	
	
	
	

\begin{credits}	
	%	\subsubsection{\discintname}
	%	The authors have no competing interests to declare that are relevant to the content of this article.
	%It is now necessary to declare any competing interests or to specifically
	%state that the authors have no competing interests. Please place the
	%statement with a bold run-in heading in small font size beneath the
	%(optional) acknowledgments\footnote{If EquinOCS, our proceedings submission
		%system, is used, then the disclaimer can be provided directly in the system.},
	%for example: The authors have no competing interests to declare that are
	%relevant to the content of this article. Or: Author A has received research
	%grants from Company W. Author B has received a speaker honorarium from
	%Company X and owns stock in Company Y. Author C is a member of committee Z.
	
	\subsubsection{\ackname}
	This work has been partially supported 
	% by	the Department Strategic Project on Artificial Intelligence of the University of Udine (2020-25), and
%	by the M4C2 I1.3 ``SEcurity and RIghts In the CyberSpace - SERICS'' (PE00000014 - CUP H73C2200089001), % SERICS spoke 6 Venexia (vale anche per SOP (SWOPS+SCAI))
	%D33C22001300002), % SERICS spoke 4 Genova (vale anche per SECCO-OC)
	by the PRIN 20228KXFN2 ``Spatio-Temporal Enhancement of Neural nets for Deeply Hierarchical Automatised Logic'' (STENDHAL),
	funded by NextGenerationEU.
\end{credits}

\section{Introduction}
TO DO: vendere 'sta roba

\section{Closure operators, topological spaces and simple graphs}

In this section we recall the definition and first properties of \emph{closure spaces} and explore the relationship of the resulting structures with simple graphs and topological spaces \cite{bussi2023spatial,ciancia2016spatial,ciancia2014specifying}. 

\begin{definition}
A \emph{closure operator} on a set $X$ is  a monotone function $c\colon (\mathcal{P}(X), \subseteq )  \to (\mathcal{P}(X), \subseteq)$ preserving finite suprema.
A \emph{closure space} is a pair $(X, c)$ given by a set and a closure operator on it.

A \emph{continuous morphism} between two closure spaces $(X, c)\to (Y, d)$ is a function $f\colon X\to Y$ such that:
\[c\circ f^* \leq f^*\circ d\]
 where $f^*\colon (\mathcal{P}(Y), \subseteq)\to (\mathcal{P}(X), \subseteq)$ assigns to a set its preimage. We will denote by $\cs$ the category so obtained.
\end{definition}

The preimage morphism $f^*\colon (\mathcal{P}(Y), \subseteq)\to (\mathcal{P}(X), \subseteq)$ preserves arbitrary meets and so, by Freyd's adjoint functor theorem \cite{freyd2003abelian,dikranjan2013categorical}, it has a left adjoint $f_! \colon (\mathcal{P}(X), \subseteq)\to (\mathcal{P}(Y), \subseteq)$ assigning to $A\subseteq X$ its image through $f$.


\begin{proposition}\label{prop:im}
	Let $(X, c)$ and $(Y, d)$ be two closure spaces,  then he following are equivalent for a function $f\colon X\to Y$:
	\begin{enumerate}
		\item $f$ is continuous;
		\item $f_! \circ c  \leq d \circ f_!$. 
	\end{enumerate}
\end{proposition}
\begin{proof} 
	$(1\Rightarrow 2).$ Suppose that $f$ is continuous, then:
	\begin{align*}
		f_!\circ c  &\leq f_! \circ  c\circ f^*\circ f_! \\&\leq f_!\circ f^*\circ d\circ f_! \\&\leq d\circ f_!
	\end{align*}
	as wanted.
	
	\smallskip \noindent $(2\Rightarrow 1).$ Computing we have:
	\begin{align*}c\circ f^*&\leq f^*\circ f_! \circ c \circ f^*  \\&\leq f^* \circ d\circ f_!\circ f^* \\&\leq f^*\circ d
	\end{align*}
	proving that $f$ is continuous. \qedhere 
	\end{proof}
	
\subsection{Closure spaces and simple graphs}	
The notion of closure space is intimately related to that of \emph{simple graphs}. The latter are in fact equivalent to the subcategory of $\cs$ given by spaces whose closure operator preserves arbitrary suprema.

\begin{definition}
	A \emph{simple graph} is a pair $(V, R)$ where $V$ is a set of \emph{nodes} and $R$ a subset of $V\times V$. A morphism $(V, R)\to (W, S)$ is a function $f\colon V\to W$ such that there exists $\hat{f}\colon R\to S$ fitting in the diagram below, where the vertical arrows are inclusions.
	\[\xymatrix@C=25pt{R \ar[r]^{\hat{f}} \ar@{>->}[d]_{i_R}& S \ar@{>->}[d]^{i_S}\\ V\times V \ar[r]_-{f\times f}& W\times W}\]
	We will denote by $\sg$ the resulting category.
\end{definition}

\begin{remark}\label{rem:elem}
	In more elementary terms, a function $f\colon V\to W$ defines a morphism  $(V, R)\to (W, S)$ if and only if $(f(x), f(y))$ is in $S$ for every pair $(x, y)\in R$.
\end{remark}

We begin to investigate the relationship by noticing that every simple graph has an associated closure space.

\begin{definition}\label{ex:graph}
	Let $(V, R)$ be a simple graph, for every $v\in V$ we put
	\[R_v:=\{w\in V \mid (v, w)\in R\}\]
	Then we define  $F(V, R)$ as $(V, c_{R})$ where
	\begin{align*}c_R\colon \mathcal{P}(V)&\to \mathcal{P}(V)\\
		A &\mapsto \bigcup_{a\in A} R_a
	\end{align*}
\end{definition}

\begin{proposition}
	$F(V,R)$ is a closure space for every simple graph $(V, R)$.
\end{proposition}
\begin{proof}
	Monotonicity of $c_R$ follows at once from the definition. Since the empty union is empty we have $c_R(\emptyset) = \emptyset$. Finally, given $A, B\subseteq V$, we have: 
	\begin{align*}
		c_R(A\cup B) &=\bigcup_{a\in A \cup B} R_a \\&= \bigcup_{a\in A} R_a \cup \bigcup_{b\in B} R_b \\&=  c_R(A)\cup c_R(B)
	\end{align*}
	and we can conclude. \qedhere 
\end{proof}

The previous proposition provides us with an example of closure space.

\begin{example}\label{ex:metric}
	Let $(X, d)$ be a metric space and consider a positive real number $\delta$. We can define a relation on $X$ putting
	\[R_{d, \delta}:=\{(x,y) \in X\times X \mid d(x,y) \leq \delta\}\] 
	
	Let $(X, c_{d, \delta})$ be $F(X, R_{d, \delta})$, for every $x\in X$ the set of point related trhough $R_{d, \delta}$ to $x$ is simply	the closed ball $B_d(x, \delta)$ of radius $\delta$ and center $x$, thus
	\begin{align*}
		c_{d, \delta}\colon \mathcal{P}(X)&\to \mathcal{P}(X)\\A &\mapsto \bigcup_{a\in A} B_d(a, \delta)
	\end{align*}
\end{example}

We can promote the construction of \Cref{ex:graph} to a functor $\sg \to \cs$.

\begin{proposition}\label{prop:funct}
	There exists a functor $F\colon \sg \to \cs$ sending a simple graph $(V, R)$ to $F(V, R)$ and which is the identity on arrows.
\end{proposition} 
\begin{proof}
	We just have to check that a morphism $f\colon (V, R)\to (W, S)$ induces a continuous morphism $(V, c_R)\to (W, c_S)$.  For every subset $A$ of $W$, let $v$ be an element of $c_R(f^*(A))$. Then there exist $a\in f^*(A)$ such that $(a, v)\in R$. By \Cref{rem:elem} we know that $f(v)$ belongs to $S_{f(a)}$. Thus $f(v)$ is in $c_S(\{f(a)\})$ and, since $f(a)\in A$, we can conclude. \qedhere 
\end{proof}

The functor of \Cref{prop:funct} is a left adjoint.

\begin{lemma}\label{lem:adj}
	The functor $F\colon \sg\to \cs$ has a right adjoint $G\colon \cs \to \sg$.
\end{lemma}
\begin{proof}
	For every closure space $(X, c)$ we can define $G(X,c)$ as $(X, R_c)$ where
	\[R_c:=\{(x,y)\in X\times X \mid y \in c(\{x\})\}\]
	
	Now, for every closure space $(X,c)$ we have $F(G(X,c)) = (X, c_{R_c})$ where, for every $A\subseteq X$ we have
	\begin{align*}
		c_{R_c}(A)&= \bigcup_{a\in A} \{x\in X \mid (a,x)\in R_c\}\\&= \bigcup_{a\in A} \{x\in X \mid x\in c(\{a\})\}\\&= \bigcup_{a\in A} c(\{a\})
	\end{align*}
	
	Now, since $c$ is monotone we have
	\begin{align*}
		c_{R_c}(A) &=\bigcup_{a\in A} c(\{a\}) \\&\subseteq c(\bigcup_{a\in A}\{a\})\\&=c(A)   
	\end{align*}
	
	The last inequality entails that $\id{X}$ is a continuous morphism $(X,  c_{R_c})\to (X, c)$. To see that this arrow has the universal property of a counit it is now enough to show that for every  $f\colon F(V, R)\to (X, c)$ in $\cs$, $f$ itself defines a morphism of simple graphs between $(V, R)\to (X, R_c)$.
	
	To do this, we have to show that $f(w)\in c(\{f(v)\})$ for every element $(v,w)$ of $R$. Now, by definition, $w\in c_R(\{v\})$ thus \Cref{prop:im} entails
	\begin{align*}
		\{f(w)\} &= f_!(\{w\}) \\&\subseteq f_!(c_R(\{v\}))\\& \subseteq c(f_{\!}(\{v\}))
	\end{align*}
	and we can conclude. \qedhere
\end{proof}

\begin{example}\label{ex:cont}
In the proof of \Cref{lem:adj} we have shown that, for every closure space $(X,c)$, $c_{R_c}\subseteq c$. This inequality is in general strict. To see this, consider $X:=\mathbb{N} \cup \{\infty\}$ and define:
	\begin{align*}c\colon \mathcal{P}(X)&\to \mathcal{P}(X)\\
	A & \mapsto \begin{cases}
		\emptyset & A = \emptyset\\
		\mathbb{N} & A \subsetneq \mathbb{N}\\
		X & \text{otherwise}\\
	\end{cases}
\end{align*}

Let us show that in this way we get a closure operator.
\begin{itemize}
	\item $c$ is monotone. Suppose that $A\subseteq B$, we have three cases:
	\begin{itemize}
		\item if $\infty$ is neither in $A$ nor in $B$ then we have two subcases:
		\begin{itemize}
			\item if $A$ is empty then $c(A)=\emptyset$ is contained in $c(B)$;
			\item if $A$ is non-empty then also $B$ contains at least one element and so $c(A)$ and $c(B)$ are both $\mathbb{N}$;
		\end{itemize}
		\item if $\infty$ is in $A$ then it is also in $B$ and $c(A)$ and $c(B)$ are both equal to $X$;
		\item if $\infty$ is in $B$ but not in $A$ then $c(B)$ is $X$ which surely contains $c(A)$.
	\end{itemize}
		\item Preservation of $\emptyset$ holds by definition.
	\item $c$ preserves binary unions. Let $A$ and $B$ be subsets of $X$. We have two cases:
	\begin{itemize}
		\item suppose that $A$ and $B$ are strictly contained in $\mathbb{N}$, then also $A\cup B$ is so, now:
		\begin{itemize}
			\item if $A$ and $B$ are both empty then $A\cup B=\emptyset$ and there is nothing to prove;
			\item if $A$ or $B$ contains one element then $A\cup B$ is non-empty and $c(A\cup B)=\mathbb{N}$, but by hypothesis at least one between $c(A)$ or $c(B)$ is $\mathbb{N}$. 
		\end{itemize}
			
		\item Suppose that at least one of $A$ and $B$ is not strictly contained in $\mathbb{N}$, then $A\cup B$ cannot be a proper subset of $\mathbb{N}$ and so $c(A\cup B)=X$, but at least one of $c(A)$ and $c(B)$ is $X$ and so the thesis follows.
	\end{itemize}
\end{itemize}

Now, for every $x\in X$ we have:
\[c(\{x\})=\begin{cases}
	\mathbb{N} & \{x\} \in \mathbb{N}\\
	X & x = \infty 
\end{cases})\]
and so 
\[R_c = (\mathbb{N}\times \mathbb{N}) \cup (\{\infty\}\times X)\] 
But then $c_{R_c}(\mathbb{N}) = \mathbb{N}$ which is strictly contained in $c(\mathbb{N})$.
\end{example}

\iffalse 
\begin{example}
	In the proof of \Cref{lem:adj} we have shown that, for every closure space $(X,c)$, $c_{R_c}\subseteq c$. This inequality is in general strict. To see this, let $(\mathbb{N}, \mathsf{s})$ be $F(\mathbb{N}, \mathsf{succ})$, where
	\[\mathsf{succ} = \Delta_\mathbb{N} \cup \{(n, m)\in \mathbb{N}\times \mathbb{N} \mid m = n+1\}\] 
	We can consider the coproduct $(X, \{\iota_i\}_{i=0}^1)$ of $\mathbb{N}$ and $\{\infty\}$ and define
	\begin{align*}c\colon \mathcal{P}(X)&\to \mathcal{P}(X)\\
		A & \mapsto \begin{cases}
		\iota_0(\mathsf{s}(A')) & A=\iota_0(A')	\text{ for some finite } A'\subseteq \mathbb{N}\\
		\iota_0(\mathsf{s}(A')) \cup \{\iota_1(\infty)\} & A=\iota_0(A')	\text{ for some infinite } A'\subseteq \mathbb{N}\\
		\iota_0(\mathsf{s}(A')) \cup \{\iota_1(\infty)\} & A=\iota_0(A') \cup \{\iota_1(\infty)\} \text{ for some} A'\subseteq \mathbb{N}\\
		\end{cases}
	\end{align*}
	
	Now, $G(X,c)$ is the simple graph $(X, R_c)$ where
	\[R_c=(\iota_0\times \iota_0)_!(\mathsf{succ}) \cup \{(\infty, \infty)\}\]
	Thus we have:
	\begin{align*}c_{R_c}(\iota_0(\mathbb{N})) &= c_{R_c}(\bigcup_{n\in \mathbb{N}} \{\iota_0(n)\})\\&= \bigcup_{n\in \mathbb{N}} c_{R_c}(\{\iota_0(n)\}) \\&= \bigcup_{n\in \mathbb{N}} \{n, n+1\}\\&= \iota_0(\mathbb{N})
		\end{align*}
	But by definition $c(\iota_0(\mathbb{N}))= X$.
\end{example}\fi 

\begin{remark}\label{rem:unit} 
	In the proof of \Cref{lem:adj} we have showd that $\id{X}\colon F(G(X,c))\to (X,c)$ is the component of the counit of $F\dashv G$ at $(X,c)$. We want now to compute the unit. 
	
	We can begin by noticing that, for every simple graph $(V,R)$,  $G(F(V,R))$ coincides with $(V, R)$. Indeed:
	\begin{align*}
	R_{c_R} &= \{(x,y)\in X\times X \mid y \in c_R(\{x\})\} \\&= \{(x,y)\in X\times X \mid y \in R_x\}\\&= \{(x,y)\in X\times X \mid (x,y)\in R\}\\&= R
		\end{align*}
	
 To see that $\id{(V,R)}$ is the unit of the adjunction $F\dashv G$, we have to show that for every arrow $f\colon (V, R)\to G(X, c)$, with $(X, c)$ a closure space, the function $f\colon V\to X$ defines a morphism $F(V, R)\to (X,c)$.  Take a subset $A$ of $X$, for every $v\in c_R(f^*(A))$, there exists $a\in f^*(A)$ such that $(a, v)\in R$. By \Cref{rem:elem} we know that $(f(a), f(v))$ is in $R_c$ and so $f(v)$ belongs to $c(\{f(a)\})$. But $f(a)\in A$ and so $c(\{f(a)\})\subseteq c(A)$, entailing $v\in f^*(c(A))$ as wanted.
\end{remark}

Our next step is to characterize the fixed points of the adjunction $F\dashv G$ just built. Clearly \Cref{rem:unit} entails that the unit $\id{\sg}\to G\circ F$  is an isomorphism and so $F$ is full and faithful. We are thus left with the characterization of the closure spaces whose counit's component is an isomorphism.

\begin{definition}
	Given a closure space $(X, c)$ and a simple graph $(V, R)$, we say that
	$c$ is \emph{additive} if $c$ preserves arbitrary unions, we will denote by $\acs$ the full subcategory of $\cs$ given by closure spaces with additive closure operators. 
\end{definition}

\begin{remark}\label{rem:impo}
 For every simple graph $(V, R)$, the closure operator $c_R$ of $F(V, c_R)$ is additive: indeed if $\{A_i\}_{i\in I}$ is a family of subsets of $V$ with union $A$ we have:
		\begin{align*}
			c_R(A) &= A \cup \bigcup_{a\in A} R_a \\&= A\cup \bigcup_{i\in I} \bigcup_{b_i\in A_i} R_{b_i} \\&=\bigcup_{i\in I} A_i \cup \bigcup_{i\in I} \bigcup_{b_i\in A_i} R_{b_i} \\&=  \bigcup_{i\in I}( A\cup \bigcup_{b_i\in A_i} R_{b_i} )\\&= \bigcup_{i\in I} c_R(A_i)
		\end{align*}
\end{remark}

\begin{remark} Let $(X, c)$ with $X$ finite, from the fact that $c$ preserves finite unions follows that $c$ is additive.
\end{remark}
The notion of additivity provides the wanted characterization.

\begin{lemma}\label{lem:clos}
	The following are equivalent for a closure space $(X,c)$:
	\begin{enumerate}
		\item $c$ is additive; 
		\item the component of the counit of the adjunction $F \dashv G$ at $(X, c)$ is an isomorphism.
	\end{enumerate}
\end{lemma}
\begin{proof}
	Let $F(G(X,c))$ be $(X, c_{R_c})$. We prove separately the two implications.
	
	\smallskip \noindent
	$(1\Rightarrow 2).$ By the proof of \Cref{lem:adj} we know that, for every $A\subseteq X$ we have:
	\[c_{R_c}(A) = \bigcup_{a\in A} c(\{a\})\]
	but by additivity the right-hand side of the previous equation is $c(A)$ and we can conclude.
	
	\smallskip \noindent
	$(2\Rightarrow 1).$ The counit $F(G(X,c))\to (X, c)$ has $\id{X}$ has underlying function. By hypothesis $\id{X}$ also defines a continuous morphism $(X,c)\to F(G(X,c))$ and this means that $c_{R_c}\leq c$. We already know that $c\leq c_{R_c}$ and so $c=c_{R_c}$. By the first point of \Cref{rem:impo} we know that $c_{R_c}$ is additive and we can conclude. \qedhere 
\end{proof}

Putting together \Cref{lem:clos} and \Cref{rem:unit} we get the following result.

\begin{corollary}\label{cor:eq}Let $I\colon \acs\to \cs$ be the inclusion functors, then $G\circ I \colon \acs \to \sg$ is an equivalence of category whos quasi-inverse $\hat{F}\colon \sg\to \acs$ fits in the diagrams below.
\[\xymatrix@C=10pt{ & \sg \ar[dr]^-{F} \ar[dl]_-{\hat{F}} \\ \acs \ar[rr]_{I} && \cs}\]	
\end{corollary}

\subsection{Reflexivity and transitivity via closure operators}

We have shown that closure spaces with additive closure operators correspond to simple graphs. Since simple graphs are sets equipped with relations, one can wonder if there are characterizations of interesting relations' properties in term of the associated closure spaces. In this section we are going to deal with reflexivity and transitivty, while we defer the discussion of symmetry to \Cref{sec:sym}.

\subsubsection{Expansive closure operators}
 Let us start with reflexivity which corresponds to \emph{expansive} closure operators. 
 
\begin{definition}
	Given a closure space $(X,c)$ and a simple graph $(V, R)$, we say that:
	\begin{enumerate}
		\item $c$ is \emph{expansive} if $\id{X}\leq c$, we will denote by $\ecs$ the full subcategory of $\cs$ given by closure spaces $(X,c)$ with $c$ expansive;
		\item $R$ is \emph{reflexive} if the diagonal $\Delta_V$ is contained in $R$, we will denote by $\rsg$ the full subcategory of $\sg$ given by simple graphs $(V, R)$ with $R$ reflexive.
	\end{enumerate}
	We will also denote by $\eacs$ the full subcategory of $\cs$ given by closure spaces equipped with expansive and additive closure operators.
\end{definition}

We are now going to show that $\eacs$ and $\rsg$ are equivalent. The key result is contained in the following lemma.

\begin{lemma}\label{lem:expref} Let $(X,c)$ be a closure space and let $(X, R_c)$ be its image through the functor $G\colon \cs \to \sg$. Then the following are equivalent:
	\begin{enumerate}
		\item $c$ is expansive;
		\item $R_c$ is reflexive.
	\end{enumerate}
\end{lemma}
\begin{proof}	\smallskip \noindent
	$(1\Rightarrow 2).$ Since $c$ is expansive, then for every $x\in X$ we have $\{x\} \subseteq c(\{x\})$ and so $(x,x)$ belongs to $R_c$. But this means that $\Delta_X \subseteq R_c$ as wanted.
	
	\smallskip \noindent
	$(2\Rightarrow 1).$  Let $A$ be a subset of $X$, since $R_c$ is reflexive then $a\in c(\{a\})$ for every $a\in A$, since $c$ is monotone this entails that $a\in c(A)$ and so $A\subseteq c(A)$. \qedhere 
\end{proof}

Since $G\circ F = \id{\sg}$ we get immediately the following.

\begin{corollary} \label{cor:expref}
	In a simple graph $(V, R)$, $R$ is reflexive if and only if $c_R$ is expansive, where $(V, c_R)$ is its image through the functor $F\colon \sg \to \cs$. 
\end{corollary}

\Cref{lem:expref} and \Cref{cor:expref}, together with \Cref{cor:eq}, give us at once the following.

\begin{corollary}\label{cor:expeq}
	Let $I_e\colon \eacs\to \acs$ and $J_r\colon \rsg\to \sg$ be the inclusion functors. There exists two functors $\check{G}\colon \eacs\to \rsg$ and $\check{F}\colon \rsg \to \eacs $ fitting in the diagram below.
	\[\xymatrix@C=30pt{\eacs \ar[d]_{I_e} \ar@{.>}[r]^-{\check{G}} & \rsg \ar[d]_{J_r} \ar@{.>}[r]^-{\check{F}} & \eacs \ar[d]^{I_e}\\ \acs \ar[r]_-{G\circ I}& \sg \ar[r]_-{\hat{F}} & \acs}\]
\end{corollary}

\subsubsection{Idempotent closure operators} We now move on to the study of the closure spaces induced by transitive relations, and of their relationship with topological spaces.

\begin{definition}
	Given a closure space $(X,c)$ and a simple graph $(V, R)$, we say that:
	\begin{enumerate}
		\item $c$ is \emph{sub-idempotent} if $c\circ c \leq c$, we will denote by $\ics$ the full subcategory of $\cs$ given by closure spaces $(X,c)$ with $c$ idempotent;
		\item $R$ is \emph{transitive} if whenever $(x,y)\in R$ and $(y,z)\in R$, then $(x,z)\in R$., we will denote by $\tsg$ the full subcategory of $\sg$ given by simple graphs $(V, R)$ with $R$ transitive.
	\end{enumerate}
	We will also denote by $\iacs$ the full subcategory of $\cs$ given by closure spaces equipped with sub-idempotent and additive closure operators.
\end{definition}

Our first aim is to prove an analog of \Cref{cor:expeq} by showing that $\iacs$ and $\tsg$ are equivalent.

\begin{lemma}\label{lem:idetran} Let $(X,c)$ be a closure space and let $(X, R_c)$ be its image through the functor $G\colon \cs \to \sg$. If $c$ is sub-idempotent then $R_c$ is transitive. If, moreover, $c$ is additive then also the converse hold.
\end{lemma}
\begin{proof} Let $(x,y)$ and $(y,z)$ be elements of $R_c$. Thus $y\in c(\{x\})$ and $z\in c(\{y\})$. Since $\{y\}\subseteq c(\{x\})$ anc using idempotency we have that 
	\begin{align*}c(\{y\})&\subseteq c(c(\{x\}))\\&=c(\{x\})
	\end{align*}
	proving that $z\in c(\{x\})$ as wanted.
	
	Suppose now that $c$ is additive and consider a subset $A$ of $X$.  Given an element $x$ $c(c(A))$, since $c$ is additive, there exists $y\in c(A)$ such that $y\in c(\{y\})$. Similarly, by additivity we also know that there is $a\in A$ such that $y\in c(\{a\})$. Thus we have $(a,y)$ and $(y,x)$ in $R_c$ and by transitivity we conclude that $(a,x)\in R_c$. But this means that $x\in c(A)$. \qedhere 
\end{proof}
\begin{example}\label{ex:cont1}
	If $(X,c)$ is a closure space and $c$ is not additive, then it is not true that the transitivity of $R_c$ entails the idempotency of $c$. Consider the closure space of \Cref{ex:cont}. On the one hand, $R_c$ is transitive, indeed let $(x,y)$ and $(y,z)$ be in $R_c$, then
	\begin{itemize}
		\item suppose that $x$ is $\infty$, then $(x,z)$ belongs to $R_c$ because $\{\infty\}\times X\subseteq R_c$;
		\item if $x\neq \infty$, then $y\in \mathbb{N}$ and so also $z$ cannot be $\infty$. We conclude that $(x,z)\in R_c$ because $(\mathbb{N}\times \mathbb{N})\subseteq R_c$.
	\end{itemize}
	
	On the other hand, $c(\{a\})=\mathbb{N}$ while $c(\mathbb{N})=X$ and so $c$ is not sub-idempotent.
\end{example}

Using again the fact $G\circ F = \id{\sg}$ and \Cref{cor:eq} we get immediately the following two corollaries.

\begin{corollary} \label{cor:idetran}
	In a simple graph $(V, R)$, $R$ is transitive if and only if $c_R$ is idempotent, where $(V, c_R)$ is its image through the functor $F\colon \sg \to \cs$. 
\end{corollary}

\begin{corollary}\label{cor:ideeq}
	Let $I_s\colon \iacs\to \acs$ and $J_t\colon \tsg\to \sg$ be the inclusion functors. There exists two functors $\bar{G}\colon \iacs\to \tsg$ and $\bar{F}\colon \rsg \to \iacs $ fitting in the diagram below.
	\[\xymatrix@C=30pt{\iacs \ar[d]_{I_s} \ar@{.>}[r]^-{\bar{G}} & \tsg \ar[d]_{J_t} \ar@{.>}[r]^-{\bar{F}} & \iacs \ar[d]^{I_s}\\ \acs \ar[r]_-{G\circ I}& \sg \ar[r]_-{\hat{F}} & \acs}\]
\end{corollary}


\subsubsection{Preorders} We can put together the results contained in the previous sections to characterize \emph{preorders} in term of the associated closure spaces. 


\begin{definition}
	Given a closure space $(X,c)$ and a simple graph $(V, R)$, we say that:
	\begin{enumerate}
		\item $c$ is \emph{topological} if $c$ is expansive and sub-idempotent, we will denote by $\tcs$ the full subcategory of $\cs$ given by spaces equipped with a topological closure operator;
		\item $R$ is a \emph{preorder} if $R$ is both reflexive and transitive, we will denote by $\pro$ the full subcategory of $\sg$ given by simple graph endowed with a preorder.
	\end{enumerate}
	We will also denote by $\tacs$ the full subcategory of $\cs$ given by closure spaces equipped with topological and additive closure operators.
\end{definition}

Our first aim is to prove an analog of \Cref{cor:expeq} by showing that $\iacs$ and $\tsg$ are equivalent.


\begin{remark}
	If $(X,c)$ is a closure space with $c$ topological, then $c$ is actually \emph{idempotent}, that is $c\circ c =c$. Indeed from $\id{X}\leq c$ it follows that $c\leq c\circ c$.
\end{remark}

\begin{lemma}\label{lem:pre} Let $(X,c)$ be a closure space and let $(X, R_c)$ be its image through the functor $G\colon \cs \to \sg$. If $c$ is topological then $R_c$ is a preorder. If, moreover, $c$ is additive then also the converse hold.
\end{lemma}
\begin{proof} This follows at once from \Cref{lem:expref,lem:idetran}. \qedhere 
\end{proof}

\begin{example}\label{ex:cont2} Let us consider again the closure space $(X,c)$ of \Cref{ex:cont,ex:cont1}. The relation $R_c$ is the union of $\mathbb{N}\times \mathbb{N}$ with $\{\infty\}\times X$ and so it is a preorder, but $c$ is not topological as it is not sub-idempotent.  
\end{example}

We can using again the fact $G\circ F = \id{\sg}$ together with \Cref{cor:eq,cor:expref,cor:expeq,cor:ideeq,cor:idetran} to get their analogs for preorders.

\begin{corollary} \label{cor:pre}
	In a simple graph $(V, R)$, $R$ is a preorder if and only if $c_R$ is topological, where $(V, c_R)$ is its image through the functor $F\colon \sg \to \cs$. 
\end{corollary}

\begin{corollary}\label{cor:preeq}
	Let $I_t\colon \tacs\to \acs$ and $J_p\colon \pro\to \sg$ be the inclusion functors. There exists two functors $\tilde{G}\colon \tacs\to \tsg$ and $\tilde{F}\colon \pro \to \tacs $ fitting in the diagram below.
	\[\xymatrix@C=30pt{\tacs \ar[d]_{I_t} \ar@{.>}[r]^-{\tilde{G}} & \pro \ar[d]_{J_p} \ar@{.>}[r]^-{\tilde{F}} & \tacs \ar[d]^{I_t}\\ \acs \ar[r]_-{G\circ I}& \sg \ar[r]_-{\hat{F}} & \acs}\]
\end{corollary}

We can sum up the results of this secion. Let $J_{p,r}\colon \pro \to \rsg$, $J_{p, t}\colon \pro \to \tsg$, $I_{t,e}\colon \tacs \to \eacs$ and $I_{t,s}\colon \tacs \to \iacs$ be the inclusion functors, then the following cubes commute and the vertical arrows are equivalences.

\[\xymatrix@R=15pt@C=15pt{&\pro \ar[rr]^{J_{p,t}} \ar[dr]^{J_p} \ar[dl]_{J_{p,r}} \ar[dd]|\hole^(.7){\tilde{F}}&&\tsg \ar[dd]^{\bar{F}} \ar[dl]_{J_{t}} && \tacs \ar[dd]|\hole^(.7){\tilde{G}}\ar[rr]^{I_{t,s}}\ar[dr]^{I_t} \ar[dl]_{I_{p,e}} &&\iacs \ar[dl]_{I_s} \ar[dd]^{\bar{G}}\\ \rsg  \ar[rr]^(.3){J_{r}} \ar[dd]_{\check{F}}&& \sg \ar[dd]^(.3){\hat{F}} && \eacs \ar[dd]_{\check{G}} \ar[rr]^(.3){I_e} && \acs \ar[dd]^(.3){G\circ I}\\ &\tacs \ar[rr]^(.3){I_{t,s}}|(.49)\hole \ar[dr]^{I_t} \ar[dl]_{I_{p,e}} &&\iacs \ar[dl]_{I_s} && \pro \ar[rr]^(.3){J_{p,t}}|(.49)\hole \ar[dr]^{J_p} \ar[dl]_{J_{p,r}} &&\tsg  \ar[dl]_{J_{t}}\\ \eacs \ar[rr]_{I_e} && \acs  && \rsg  \ar[rr]_{J_{r}} && \sg} \]


\subsection{Closure operators, topological spaces and preorders}

We have already examined the relationship between closure spaces equppied with topological and additive closure operator. If we drop additivity we obtain a subcategory of $\cs$ which is equivalent to that of topological spaces. Under this equivalence, $\tacs$ corresponds to the subcategory of \emph{Alexandroff spaces}(\cite{alexandroff1937diskrete,arenas1999alexandroff,johnstone1982stone}). Moreover, exploiting the functor $F\dashv G$ proved in \Cref{lem:adj}, and the results of the previous sections we recover the well-known construction which assigns to a topological space its \emph{specialization preorder} \cite{goubault2013non,hartshorne2013algebraic}.

We begin with the following proposition.

\begin{proposition}\label{prop:top}

Let $\topo$ be the category of topological spaces and continuous maps, there is a functor $T\colon \topo\to \cs$ which is the identity on arrows and sends a space $(X, \mathcal{O})$ to $(X, c_{\mathcal{O}})$, where
\begin{align*}c_{\mathcal{O}}\colon (\mathcal{P}(X), \subseteq) &\to (\mathcal{P}(X), \subseteq)\\ A &\mapsto \inf(\{C\in \mathcal{P}(X) \mid C \text{ is closed and } A\subseteq C\})
\end{align*}

Moreover, $c_{\mathcal{O}}$ is topological.
\end{proposition}
\begin{proof}
	For the first part of our claim, it is enough to show that $(X, c_{\mathcal{O}})$ is a closure spaces. Clearly $c_{\mathcal{O}}$ is monotone, let us show that it preserves finite unions.
	\begin{itemize}
		\item since $\emptyset$ is closed we immediately have that $c_{\mathcal{O}}(\emptyset)=\emptyset$;
		\item take $A, B\in \mathcal{P}(X)$, since $c_{\mathcal{O}}$ is monotone then we know that 
		\[c_{\mathcal{O}}(A)\cup c_{\mathcal{O}}(B) \subseteq c_{\mathcal{O}}(A\cup B)\]
		Now, $c_{\mathcal{O}}(A)\cup c_{\mathcal{O}}(B)$, being the finite union of closed sets, is a closed set, and it contains $A\cup B$ and so other inclusion holds too. 
	\end{itemize} 
	
	To see that $c_{\mathcal{O}}$ is topological notice that it is expansive by definition. Moreover, since the arbitrary intersection of closed sets is closed, $c_{\mathcal{O}}(A)$ is a closed set for every $A\in \mathcal{P}(X)$, entailing that $c_{\mathcal{O}}$ is idempotent. \qedhere 
\end{proof}

The functor just defined is a left adjoint.

\begin{proposition}\label{prop:adj}
The functor $T\colon \topo \to \cs$ has a right adjoint $H\colon \cs \to \topo$.
\end{proposition}
\begin{proof}
	Given a closure space $(X,c)$ we can define:
	\[\mathcal{O}_c:=\{A\subseteq X \mid A= X\smallsetminus c(B) \text{ for some $B$ in $\mathcal{P}(X)$}\}\]

	It is easy to see that $\mathcal{O}_c$ is a topology on $X$:
	\begin{itemize}
		\item $\emptyset$ and $X$ belong to $\mathcal{O}_c$: indeed 
		\[\emptyset = X\smallsetminus c(X) \qquad X = X \smallsetminus c(\emptyset)\] 
		\item let $A_1$ and $A_2$ be elements in $\mathcal{O}_c$, by definition there are $B_1,B_2\in \mathcal{P}(X)$ such that 
		\[A_1=X\smallsetminus c(B_1)\qquad A_2 = X \smallsetminus c(B_2)\]
		therefore we have:
		\begin{align*}
			A_1\cap A_2 &= X\smallsetminus c(B_1) \cap X \smallsetminus c(B_2)\\&=
			X \smallsetminus (c(B_1)\cup c(B_2))\\&= X \smallsetminus (c(B_1\cup B_2 ))
		\end{align*}
		proving that $A_1\cap A_2$ belongs to $\mathcal{O}_c$. 
		\item  Let $\{A_i\}_{i\in I}$ be a family of elements of $\mathcal{O}_c$ and let also $\{B_i\}_{i\in I}$ be a family of elements such that
		\[A_i= X\smallsetminus c(B_i)\]
		Since $c$ is monotone and idempotent we have:
		\begin{align*}
			c(\bigcap_{i\in I} c(B_i)) &\subseteq \bigcap_{i\in I} c(c(B_i)) \\&= \bigcap_{i\in I}c(B_i)
		\end{align*}
		whilethe other inclusion is guaranteed by expansivity of $c$. Therefore we get:
		\begin{align*}
			\bigcup_{i\in I} A_i&=\bigcup_{i\in I} (X\smallsetminus c(B_i)) \\&= X \smallsetminus \bigcap_{i\in I} c(B_i)\\&= X \smallsetminus c(\bigcap_{i\in I} c(B_i)) 
		\end{align*}
		which proves that $\mathcal{O}_c$ is closed under arbitrary unions.
	\end{itemize}
	
	Notice that, by definition, $C$ is closed in the topology $\mathcal{O}_c$ if and only if it lies in the image of $c$ and this entails immediately that $c_{\mathcal{O}_c}\leq c$. Thus $\id{X}$ defines an arrow $T(H(X,c))\to (X,c)$ in $\cs$. Let us show that it has the universal property of a counit.
	
Let $(Y, \mathcal{O})$ be a topological space, given a morphism $f\colon T(Y, \mathcal{O})\to (X,c)$, the arrow $f$ itself defines a continuous function $(Y, \mathcal{O})\to (X, \mathcal{O}_c)$. Consider a closed set $C\subseteq X$, then there exists $B\subseteq X$ such that $C= c(B)$. We want to show that $f^*(C) = c(f^*(B))$.

On the one hand, if $D$ is a closed set contained in $Y$ which contains $f^*(B)$, then $f^*(C)\subseteq D$. 


 Since $f$ is a morphism un $\cs$ we have
\begin{align*}
f^*(C)= f^*(c(B))
\end{align*}


 $A\in \mathcal{O}_c$
\end{proof}




\begin{proposition} The following are true:
	\begin{enumerate}
		\item a closure space $(X,c)$ is in the essential image of $T$ if and only if it is in $\tcs$.
	\end{enumerate}
\end{proposition}
\begin{proof} \begin{enumerate}
		\item 
		\item  To conclude, it is thus enough to show that every $(X,c)$ in $\tcs$ is isomorphic to $(X, \mathcal{O})$ for some topological space $(X, \mathcal{O})$.  For the other inclusion, if $A\subseteq c(B)$ for some $B\in \mathcal{P}(X)$ then
		\begin{align*}
			c(A)&\subseteq c(c(B))\\&=c(B)
		\end{align*}
		implying that $c(A)\subseteq c_{\mathcal{O}_c}(A)$ as wanted.	\qedhere 
	\end{enumerate}
\end{proof}

\begin{corollary}\label{cor:topeq}
	contenuto...
\end{corollary}



We can sum up the results of this section in the following diagram.
\[\xymatrix{mappa con tutti i funtori qu}\]

\subsection{Derived spatial operators}

This section is devoted to the study of the operators that can be obtained combining the closure operator of a given a closure space $(X,c)$ with the complete boolean algebra structure of $(\mathcal{P}(X), \subseteq)$.

\subsubsection{The opposite closure operators} In the previous section we have examined the relationship between closure spaces and simple graphs. As every relation has an \emph{opposite}, so does every closure operator.

\begin{definition} Given a closure space $(X, c)$ we define the \emph{opposite of $c$} as:
	\begin{align*}
	c^{-1}\colon \mathcal{P}(X)&\to \mathcal{P}(X)\\
	A &\mapsto \bigcup_{a\in A} \{x\in X\mid a\in c(\{x\})\}
	\end{align*}
\end{definition}

\begin{remark}\label{rem:el0}
By definition, $c^{-1}$ defines a monotone morphism $(\mathcal{P}(X), \subseteq)\to (\mathcal{P}(X), \subseteq)$.
\end{remark}

\begin{remark}\label{rem:el1}
Given a set $X$, let $\eta_X\colon X\to \mathcal{P}(X)$ and $\mu_{X}\colon \mathcal{P}(\mathcal{P}(X))\to \mathcal{P}(X)$ consider the functions:
\[\begin{split}
	X & \to \mathcal{P}(X)\\
	x &\mapsto \{x\}
\end{split} \qquad \begin{split}
\mathcal{P}(\mathcal{P}(X)) & \to \mathcal{P}(X)\\
A &\mapsto \bigcup_{B\in A} B
\end{split}
\]

Take now the arrow $c^*\colon \mathcal{P}(\mathcal{P}(X))\to \mathcal{P}(\mathcal{P}(X))$,
then $c^{-1}$ coincides with $\mu_{X}\circ c^*\circ \eta_{\mathcal{P}(X)}$, i.e.~ the following diagram commutes.
\[\xymatrix{\mathcal{P}(X) \ar[r]^{c^{-1}}\ar[d]_{\eta_{\mathcal{P}(X)}}& \mathcal{P}(X) \\\mathcal{P}(\mathcal{P}(X)) \ar[r]_{c^{*}}& \mathcal{P}(\mathcal{P}(X)) \ar[u]_{\mu_{X}}}\]
\end{remark}

\begin{remark}\label{rem:el3}
Consider a closure space $(X,c)$ and a subset $A\subseteq X$. Then
\[c^{-1}(A)= \{x\in X \mid c(\{x\})\cap A \neq \emptyset\}\]
Indeed if $x\in c^{-1}(A)$ then there exists $a\in A$ such that $a\in c(\{x\})$ and so $a\in c(\{x\})\cap A$. Conversely, if $x\in X$ is such that there is an $a\in c(\{x\})\cap A$ then $x\in c^{-1}(A)$ by definition.
\end{remark}

\begin{remark}\label{rem:el2} It is worth to notice that $(X, c^{-1})$ is an additive closure space, indeed:	 
	\begin{itemize}
		\item given $A\subseteq X$, for every $a\in A$ we have $c(\{a\})\subseteq c(A)$ and $c(\{a\})$ is non-empty because it contains at least $a$, so $c(\{a\})\cap A \neq \emptyset$ and, by \Cref{rem:el3} we get that $A\subseteq c^{-1}(A)$; 
		%\item by \Cref{rem:el2} we get $c^{-1}(\emptyset)=\emptyset$;
		\item let $\{A_i\}_{i\in I}$ be a family of subsets of $X$ with union $A$, then:
		\begin{align*}
			c^{-1}(A) &= \bigcup_{a\in A}\{x\in X \mid a \in c(\{x\})\}\\&=\bigcup_{i\in I}\bigcup_{a_i\in A_i} \{x\in X \mid a_i \in c(\{x\})\}\\&=\bigcup_{i\in I}c^{-1}(A_i)
		\end{align*}
		showing that $c^{-1}$ is an additive closure operator.
	\end{itemize}
\end{remark}

Finally, we can show that this construction provides us with a functor $\cs\to \cs$.
\begin{proposition}
	There is a functor $(-)^{\circ}\colon \cs \to \cs$ sending $(X,c)$ to $(X, c^{-1})$ and which is the identity on arrows.
\end{proposition}
\begin{proof}
	Let $f$ be a morphism $(X, c)\to (Y, d)$, given $A\subseteq X$ we have to show that the image of every element of $c^{-1}(A)$ is in $d^{-1}(f_!(A))$.
	
	Let thus $x$ be in $c^{-1}(A)$, hence there is $a\in A$ such that $a\in c(\{x\})$ and we have
	\begin{align*}
		\{f(a)\}&\subseteq f_!(c(\{x\})) \\&\subseteq d(f_!(\{x\}))=d(\{f(x)\})
	\end{align*}
	but this now entails that $f(x)\in d(f_!(A))$ so that we can conclude.
\end{proof}

As anticipated, the functor $(-)^\circ$ is linked to the construction which sends a relation to its opposite. 
\begin{definition}
	Let $(V, R)$ be a simple graph, it \emph{opposite} $(V, R^{op})$ is the simple graph defined by the relation
	\[R^{op}:=\{(x,y)\in X\times X \mid (y,x)\in R\}\]  
	A relation is \emph{symmetric} if $R^{op}$ coincides with $R$.
\end{definition}

\begin{proposition} There is a functor $(-)^\textsf{op}\colon \sg \to \sg$ sending a simple graph $(V, R)$ to $(V, R^{op})$ and fitting in the diagrams below.
	\[\xymatrix@C=30pt{\cs \ar[r]^-{(-)^\circ}& \cs & \cs \ar[r]^-{(-)^\circ}& \cs & \sg \ar[r]^-{(-)^\textsf{op}} \ar@{<-}[d]_{G}& \sg \ar@{<-}[d]^{G}\\ \sg \ar[r]_-{(-)^\textsf{op}} \ar@{<-}[u]^{G}& \sg \ar[u]_{F}&\sg \ar[r]_-{(-)^\textsf{op}} \ar[u]^{F}& \sg \ar[u]_{F} & \cs \ar[r]_-{(-)^\circ} & \cs}\]
\end{proposition}
\begin{proof} We have to define $(-)^{\textsf{op}}$ on arrows. Let $f$ be a morphism $(V, R)\to (W, S)$ and suppose that $(x,y)\in R^{op}$, then $(y,x)$ is in $R$ and so $(f(y), f(x))$ must be in $S$, but this means that $(f(x), f(y))$ is in $S$ and so we can define the action $(-)^{\textsf{op}}$ on arrows as the identity. Let us prove the commutativity of the three diagrams.
	\begin{itemize}
		\item Given a closure space $(X, c)$, let $(X, d)$ be $F(X, R^{op}_c)$, computing we get:
		\begin{align*}
		d(A)&=A \cup \bigcup_{a\in A} \{x \in X \mid  (a,x)\in R^{op}_c\} \\&=A \cup \bigcup_{a\in A} \{x \in X \mid  (x,a)\in R_c\} \\&= A \cup  \bigcup_{a\in A} \{x \in X \mid  a\in c(\{x\})\} \\&= A\cup c^{-1}(A) \\&= c^{-1}(A)
		\end{align*}
			\item Consider a simple grah $(V, R)$ and let $(V, d)$ be $F(V, R^{op})$.  Applying the definitions, for every $A$ in $\mathcal{P}(V)$ we have:
			\begin{align*}
				d(A)&=A\cup \bigcup_{a\in A} \{v \in V\mid (a, v)\in R^{op}\}\\&=A\cup \bigcup_{a\in A} \{v \in V\mid (v, a)\in R\}\\&= A \cup \bigcup_{a\in A} \{v\in V \mid a\in c_R(\{v\})\}\\&=A\cup c^{-1}_R(A)\\&= c^{-1}_R(A)
			\end{align*}

		\item Start with a closure space $(X, c)$ and let $(X, R)$ be $G(X, c^{-1})$, then we have:
		\begin{align*}
			R&=\{(x,y)\in X\times X\mid y \in c^{-1}(\{x\})\} \\ &= \{(x,y)\in X\times X \mid x\in c(\{y\})\}\\&=R^{op}_c
		\end{align*} 
		showing the wanted equality. \qedhere
	\end{itemize}
\end{proof}


\begin{example}
	\Cref{ex:metric}
\end{example}

\begin{remark}Notice that, while $(-)^{\textsf{op}}\circ (-)^{\textsf{op}}$ is the identity on $\sg$, composing $(-)^\circ$ with itself yields $F\circ G$.
\end{remark}

\begin{remark} The composition  $G\circ (-)^\circ \circ F$ is not equal to $(-)^{\textsf{op}}$:  indeed every simple graph in the essential image of $G$ is reflexive, but $(V, R^{op})$ is reflexive if and only if $(V, R)$ is so.
\end{remark}

\subsubsection{The interior operator and symmetry}\label{sec:sym}
\begin{remark}\label{rem:impo}
	We can notice the following elementary but pivotal facts:
	\begin{enumerate}
		\item for every simple graph $(V, R)$, the closure operator $c_R$ of $F(V, c_R)$ is additive: indeed if $\{A_i\}_{i\in I}$ is a family of subsets of $V$ with union $A$ we have:
		\begin{align*}
			c_R(A) &= A \cup \bigcup_{a\in A} R_a \\&= A\cup \bigcup_{i\in I} \bigcup_{b_i\in A_i} R_{b_i} \\&=\bigcup_{i\in I} A_i \cup \bigcup_{i\in I} \bigcup_{b_i\in A_i} R_{b_i} \\&=  \bigcup_{i\in I}( A\cup \bigcup_{b_i\in A_i} R_{b_i} )\\&= \bigcup_{i\in I} c_R(A_i)
		\end{align*}
		\item for every closure space $(X, c)$, $G(X,c)$ is in $\rsg$: indeed if $G(X, c)=(X, R_c)$, then for every $x\in X$ we know that $\{x\}\subseteq c_R(\{x\})$ and so $(x,x)$ belongs to $R_c$.
	\end{enumerate}
\end{remark}

\begin{lemma}\label{lem:clos}
	The following are equivalent for a closure space $(X,c)$:
	\begin{enumerate}
		\item $c$ is additive; 
		\item the component of the counit of the adjunction $F \dashv G$ at $(X, c)$ is an isomorphism.
	\end{enumerate}
\end{lemma}
\begin{proof}
	Let $F(G(X,c))$ be $(X, c_{R_c})$. We prove separately the two implications.
	
	\smallskip \noindent
	$(1\Rightarrow 2).$ By the proof of \Cref{lem:adj} we know that, for every $A\subseteq X$ we have:
	\[c_{R_c}(A) = \bigcup_{a\in A} c(\{a\})\]
	and we can conclude by additivity.
	
	
	\smallskip \noindent
	$(2\Rightarrow 1).$ The counit $F(G(X,c))\to (X, c)$ has $\id{X}$ has underlying function. By hypothesis $\id{X}$ also defines a continuous morphism $(X,c)\to F(G(X,c))$ and this means that $c_{R_c}\leq c$. We already know that $c\leq c_{R_c}$ and so $c=c_{R_c}$. By the first point of \Cref{rem:impo} we know that $c_{R_c}$ is additive and we can conclude. \qedhere 
\end{proof}



\begin{lemma}\label{lem:graph}
	The following are equivalent for a simple graph $(V,R)$:
	\begin{enumerate}
		\item $R$ is reflexive; 
		\item the component of the unit of the adjunction $F \dashv G$ at $(V, R)$ is an isomorphism.
	\end{enumerate}
\end{lemma}
\begin{proof}
	Let $G(F(V,R)$ be $(V, R_{c_R})$. As before we divide the proof of the two implications.
	
	\smallskip \noindent
	$(1\Rightarrow 2).$  By \Cref{rem:unit} $R_{c_R} = \Delta_V\cup R$, but if $R$ is reflexive this entails $R_{c_R}=R$.
	
	\smallskip \noindent
	$(2\Rightarrow 1).$ By \Cref{rem:unit} we know that the unit of $F\dashv G$ has $\id{V}$ has underlying function. The hypothesis that this arrow $(V, R)\to (V, R_{c_R})$ is an isomorphsim entails that $R_{c_R}\subseteq R$ and so $R$ must contain $\Delta_V$. \qedhere 
\end{proof}

Putting together \Cref{lem:clos,lem:graph} we get the following result.

\begin{corollary}Let $I\colon \acs\to \cs$ and $J\colon \rsg \to \sg$ be the inclusion functors, then ther exists an equivalence of category $\hat{F}\colon \rsg\to \acs$, with quasi-inverse $\hat{G}\colon \acs \to \rsg$, fitting in the diagram below.
	\[\xymatrix{ \acs\ar[d]_{I} \ar@{.>}[r]^-{\hat{G}}& \rsg \ar[d]^{J}\ar@{.>}[r]^-{\hat{F}}& \acs \ar[d]^{I}\\\cs\ar[r]_-{G} & \sg \ar[r]_-{F} & \cs}\]
\end{corollary}

\begin{example}
	Starting with a topological space $(X, \mathcal{O})$, we can compute the value of $G(T(X, \mathcal{O}))$. By definition it is a simple graph $(X, R_{\mathcal{O}})$, where $R_ {\mathcal{O}}$ is defined as:
	\[R_ {\mathcal{O}}:= \{(x,y)\mid y \in c_{\mathcal{O}}(\{x\})\}\]
	Notice that $R_ {\mathcal{O}}$ is a preorder: 
	\begin{enumerate}
		\item we know by the second point of \Cref{rem:impo} that $R_{\mathcal{O}}$ is reflexive;
		\item suppose that $y\in c_{\mathcal{O}}(\{x\})$ and $z\in c_{\mathcal{O}}(\{y\})$, if $C$ is a closed set containing $x$, then $C$ must contain $y$ and so it must also contain $z$, proving that $z\in c_{\mathcal{O}}(\{x\})$.
	\end{enumerate}
	
	Summing up, $G(T(X, \mathcal{O}))$ is $X$ equipped with the so-called \emph{specialization preorder} \cite{hartshorne2013algebraic}. Notice that in some texts this name is attributed to the opposite of $R_{\mathcal{O}}$ \cite{goubault2013non}. 
\end{example}


\subsubsection{The reachability operators}

In this section we introduce two \emph{reachability} operators which intuitively capture the meaning of being able to reach, or being reached from, a certain region, remaining in another given one.


\begin{definition} Let $n$ be a natural number greater than $0$, and let $S_n\subseteq n\times n$ be the \emph{successor relation} on it:
	\[S_n:=\{(k,l)\in n\times n \mid l = l+1\}\]  
Let also $(n, s_n)$ be $F(n, S_n)$. A \emph{spatial path of length $n-1$} in a closure space $(X,c)$ is a continuous morphism $(n, s_n)\to (X, c)$. Given a spatial path $(n, s_n)\to (X, c)$, we will denote the length $n$ of the path by $\lgh(p)$, we will also denote the \emph{starting} and \emph{ending} points $p(0)$ and $p(n-1)$ by $\str(p)$ and $\nd(p)$ and write $p\colon \str(p) \rightsquigarrow_{\lg(p)} \nd(p)$. Finally, we define $\pth(X,c)$ as the set of spatial path in $(X,c)$
\end{definition}

We can recast the definition of spatial paths in terms of finite sequences of points.

\begin{proposition}\label{prop:equipath}
	Let $(X, c)$ be a closure space and let $\{x_i\}_{i=0}^{n-1}$ be a sequence of points in $X$, then the following are equivalent:
	\begin{enumerate}
		\item there exists a spatial path $p\colon (n, s_n)\to (X, c)$ such that $p(i)=x_i$ for every $i\in n$;
		\item for every $i\in n$ we have $x_{i+1}\in c(\{x_i\})$.
	\end{enumerate}
\end{proposition}
\begin{proof}
	\begin{enumerate}
		$(1\Rightarrow 2).$ We start by using \Cref{prop:im} to get:
		\begin{align*}
			p_!(s_n(\{i\}))&\subseteq c(p_!(\{i\})) \\&=c(\{p(i)\})\\&=c(\{x_i\})
		\end{align*}
		But by definition of $S$ we have $s_n(\{i\}) = \{i, i+1\}$ and so $x_{i+1}$ belongs to $p_!(s_n(\{i\}))$, allowing us to conclude.
		
		\smallskip \noindent
		$(2\Rightarrow 1).$ We have to check that the function 
	\begin{align*}
		p \colon n &\to X\\
		i &\mapsto x_i
	\end{align*}
	is continuous. Let $A$ be a subset of $n$, then
	\[s_n(A)= A \cup \bigcup_{a\in A}\{i \in n\mid i= a+1\}\]
	Thus we have
	\begin{align*}
p_!(c_S(A)) &= p_!(A)\cup \bigcup_{a\in A}\{x \in X\mid  x=x_{a+1}\}\\&\subseteq \bigcup_{a\in A} c(\{x_a\}) \\&\subseteq c(\bigcup_{a\in A}\{x_a\})\\&= c(p_!(A))
	\end{align*}

	\end{enumerate}
\end{proof}

\begin{corollary}\label{cor:sv}
	For every closure space $(X,c)$, $\pth(X,c)=\pth(F(G(X,c)))$.
\end{corollary}
\begin{proof}
	Let $(X, c_{R_c})$ be $F(G(X,c))$, the thesis follows immediately from \Cref{prop:equipath} noticing that, for every point $x\in X$ we have
$c_{R_c})\{x\} = c(\{x\})$. \qedhere 
\end{proof}

\begin{remark}\label{rem:rev}
	Let $x$ and $y$ be two points in a closure space $(X, c)$ and $p\colon x\rightsquigarrow_{n} y$ a path of length $n$ between them. We can define
	\begin{align*}
		p^{-1} \colon n + 1 &\to X\\
		i & \mapsto p(n-i)
	\end{align*} 
	to get a spatial path $y\rightsquigarrow_n x$ in $(X, c^{-1})$. To see this, we can apply \Cref{prop:equipath} to $\{p^{-1}(i)\}_{i=0}^{n}$, to deduce that $p(i+1)$ belongs to $c(\{p(i)\})$ for every $i\in n+1$. But
	\[c(\{p^{-1}(i+1)\})=c(\{p(n-i-1)\})\]
	and we can conclude applying again \Cref{prop:equipath}.
	
	In this way we get a function $\mathsf{r}_{(X,c)}\colon \pth(X,c)\to \pth(X, c^{-1})$ sending $p$ to $p^{-1}$ which, by \Cref{cor:sv}, is a bijection with $\mathsf{r}_{(X,c)^\circ}$ has its inverse.
\end{remark}

\begin{remark}\label{rem:grpath}
Consider the closure space $F(V, R)$ associated to the simple graph $(V, R)$, then a spatial path is simply a finite sequence of vertices $\{v_i\}_{i=0}^{n-1}$ such that $(v_i, v_{i+1})\in R\cup \Delta_V$.
\end{remark}

\begin{example}
	fai esempio con i delta
\end{example}

So equipped we can now define two promised reachability operators 

\begin{definition}[\cite{bussi2023spatial}] Let $(X, c)$ be a closure space, given two subsets $A, B\in \mathcal{P}(X)$, we define
	\begin{align*}
		\avr_c(A,B)&:=\{x\in X \mid \text{ there exists } p\colon x \rightsquigarrow_{n+1} y \text{ such that } y\in A \text{ and } p_!(\{i\}_{i=1}^{n-1}) \subseteq B \} \\
		\devr_c(A,B)&:=\{x\in X \mid \text{ there exists } p\colon y \rightsquigarrow_{n+1} x \text{ such that } y\in A \text{ and } p_!(\{i\}_{i=1}^{n-1}) \subseteq B \}
	\end{align*}
	
	We then define the \emph{forward} and \emph{backward reachability} operators as the functions:
	\[
	\begin{split}
		\avr_c \colon \mathcal{P}(X)\times \mathcal{P}(X) &\to \mathcal{P}(X) \\
		(A, B)&\mapsto \avr(A,B) 
	\end{split} \qquad 
	\begin{split}
		\devr \colon \mathcal{P}(X)\times \mathcal{P}(X) &\to \mathcal{P}(X) \\
		(A, B)&\mapsto \devr_c(A,B) 
	\end{split}\]
\end{definition}

\begin{remark}\label{rem:sv}
	Consider a closure space $(X,c)$ and its image $(X, c_{R_c})$ through the functor $F\circ G$. By \Cref{cor:sv} a path in $(X,c)$ is the same as a path in $F(G(X,c))$ and so we deduce that:
	\[\avr_c = \avr_{c_{R_c}} \qquad \devr_c = \devr_{c_{R_c}}\]
\end{remark}

\begin{remark}\label{rem:difbus} differenza con Bussi
	contenuto...
\end{remark}

\begin{example}
	
	fai esempio con un grafo
\end{example}


We can prove a first lemma regarding the interplay of the reachability operators and the functor $(-)^\circ$. In order to easier the proof we can make a preliminar remark.


\begin{lemma}\label{lem:dual}
	Given a closure space $(X,c)$, the following equalities hold true:
	\[\avr_c = \devr_{c^{-1}} \qquad \devr_c = \avr_{c^{-1}} \]
\end{lemma}
\begin{proof}
	Let $(X,c)$ be a closure space, for every $A, B\in \mathcal{P}(X)$ we can define
\begin{align*}
	\pavr_{c}(A,B)&:=\{ p \in \pth(X,c) \mid \nd(p) \in A \text{ and } p_!(\{i\}_{i=1}^{\lgh(p)-2}) \subseteq B \}\\
	\pevr_{c}(A,B)&:=\{ p \in \pth(X,c) \mid \str(p) \in A \text{ and } p_!(\{i\}_{i=1}^{\lgh(p)-2}) \subseteq B \}
\end{align*}
Let $i^c_{A,B}\colon \pavr_c(A,B)\to \pth(X,c)$ and $j^c_{A,B}\colon \pevr_c(A,B)\to \pth(X,c) $ be the inclusions. 

If we consider the evaluations 
\[\begin{split}\mathsf{start}\colon \pth(X,c) &\to X\\
	p & \mapsto \str(p)
\end{split} \qquad \begin{split}\mathsf{end}\colon \pth(X,c) &\to X\\
p & \mapsto \nd(p)
\end{split}\]
then it is immediate to notice that $\avr_c(A,B)$ is the image of $\str\circ i^c_{A,B}$ and $\devr_c(A,B)$ that of $\nd \circ j^c_{A,B}$. 

Now, we can also notice that isomorphism $\mathsf{r}_{(X,c)}\colon \pth(X,c)\to \pth(X, c^{-1})$ induces a bijection $\mathsf{r}^{c}_{A,B}\colon \pavr_c(A,B)\to \pevr_{c^{-1}}(A,B)$ fitting in the diagram below:
\[\xymatrix@C=15pt@R=15pt{\pavr_c(A,B) \ar@{.>}[rr]^-{\mathsf{r}^{c}_{A,B}}&& \pevr_{c^{-1}}(A,B)\\ & X \\ \pth(X,c) \ar[rr]_-{\mathsf{r}_{(X,c)}} && \pth(X,c^{-1})} \]

But then ... have the same image and the thesis follows. \qedhere
\end{proof}

Through the reachability oberators we can recover both the original closure operator and its opposite.

\begin{proposition}\label{prop:cloreach}Let $(X,c)$ be a closure space and $(X, c_{R_c})$ be $F(G(X,c))$. Then for every $A\subseteq X$ the following hold true:
\begin{enumerate}
	\item  $c^{-1}(A) = \avr_c(A, \emptyset)$;
	\item  $c_{R_c}(A) = \devr_c(A, \emptyset)$.
\end{enumerate}	
In particular, if $c$ is additive then $c(A) = \devr_c(A, \emptyset)$.
	\end{proposition}
\begin{proof}
	\begin{enumerate}
		\item  We prove separately the two inclusions.
		
		\smallskip \noindent
		$c^{-1}(A) \subseteq \avr_c(A, \emptyset)$.  Given a point $x$ in $c^{-1}(A)$, we know that there exists $a\in A$ such that $a\in c(\{x\})$. Thus by \Cref{prop:equipath} the family $\{x_i\}_{i=0}^1$ with
		\[x_0:= a \qquad x_1:= x\]
		defines a path $p\colon x\rightsquigarrow_{n+1}a$ which by definition witnesses that $x\in \avr_c(A, \emptyset)$.
		
		\smallskip \noindent 
		$ \avr_c(A, \emptyset) \subseteq c^{-1}(A)$. Let $x$ be an element of $\avr_c(A, \emptyset)$, thus there exists $p\colon x\rightsquigarrow_{n+1} a$ with $a\in A$ and such that $p_!(\{x_i\}_{i=1}^{n-1}) \subseteq \emptyset$, implying $n=0$. By \Cref{prop:equipath} it must be that $a\in c(\{x\})$ and so $x\in c^{-1}(A)$.
		\item As in the previous point we are going to prove the two inclusions.
		$c(A) \subseteq \devr_c(A, \emptyset)$.  Given a point $x$ in $c^{-1}(A)$, we know that there exists $a\in A$ such that $a\in c(\{x\})$. Thus by \Cref{prop:equipath} the family $\{x_i\}_{i=0}^1$ with
		\[x_0:= a \qquad x_1:= x\]
		defines a path $p\colon x\rightsquigarrow_{n+1}a$ which by definition witnesses that $x\in \avr_c(A, \emptyset)$.
		
		\smallskip \noindent 
		$ \avr_c(A, \emptyset) \subseteq c^{-1}(A)$. Let $x$ be an element of $\avr_c(A, \emptyset)$, thus there exists $p\colon x\rightsquigarrow_{n+1} a$ with $a\in A$ and such that $p_!(\{x_i\}_{i=1}^{n-1}) \subseteq \emptyset$, implying $n=0$. By \Cref{prop:equipath} it must be that $a\in c(\{x\})$ and so $x\in c^{-1}(A)$.
		
		 \qedhere 
	\end{enumerate}
\end{proof}

We are now going to give a semantics to $\avr$ and $\devr$ which doesn't rely on points and paths. A first step is given by the following lemma.

\begin{lemma}\label{lem:util}
Let $(X,c)$ be a closure spaces, then for every $A, B\subseteq X$ we have:
\begin{enumerate}
\item $\avr_c(A,B)=c^{-1}(A)\cup c^{-1}(B\cap \avr_c(A,B))$;
\item $\devr_c(A,B)=c(A)\cup c(B\cap \devr_c(A,B))$.
\end{enumerate}
\end{lemma}
\begin{proof}
	\begin{enumerate}
		\item 
		\item This follows at once from the previous point and \Cref{lem:dual}.
	\end{enumerate}
\end{proof}
In order to proceed further we need to recall the following well known result \cite{knaster1928theoreme,tarski1955lattice}

\begin{lemma}[Knaster-Tarski's Theorem]\label{lem:KT}
	Let $(P, \leq)$ be a complete poset, then every monotone morphism $f\colon (P, \leq)\to (P, \leq)$ has a least fixed point $\lfp(f)$ which is also the minimum of the set
	\[A_f:=\{p\in P\mid f(p) \leq p\}\]
\end{lemma}
\begin{proof}
Since $(P, \leq)$ is complete we can consider the infimum $\lfp(f)$  of $A_f$. If we show that $\lfp(f)$ is a fixed point for $f$ we are done. To see this, let us begin by noticing that for every $p\in A$ we have:
\begin{align*}
	f(\lfp(f)) &\leq f(p) \\&\leq p
\end{align*}
and so $f(\lfp(f))\leq \lfp(f)$. But this entails that $\lfp(f)$ belongs to $A$ and so $\lfp(f)=f(\lfp(f))$. \qedhere 
\end{proof}

\begin{remark}\label{rem:fix_mon}
	Given a complete partial order $(P, \leq)$ the set $\mon(P)$ of monotone functions $(P, \leq)\to (P, \leq)$ can be ordered componentwise. With respect to this order, taking the least fixed point defines a monotone morphism $\lfp\colon (\mon(P), \leq)\to (P, \leq)$. Indeed, let $f,g\in \mon(P)$ be two functions such that $f\leq g$, then for every $p\in A_g$ we have:
	\begin{align*}
		f(p)&\leq g(p)\\&\leq p 
	\end{align*}
	and so $A_g \subseteq A_f$. But then \Cref{lem:KT} entails $\lfp(f)\leq \lfp(g)$.
\end{remark}

We can use the previous result to define two new operators on a given closure space.

\begin{definition} Let $(X, c)$ be a closure space, for every $A, B\in \mathcal{P}(X)$ we can define a monotone function $\Phi_{(X,c)}^{A,B}\colon (P, \leq)\to (P,\leq)$ as:
\begin{align*}
\Phi_{(X,c)}^{A,B}\colon (\mathcal{P}(X), \subseteq)&\to (\mathcal{P}(X), \subseteq)\\
	Z & \mapsto c(A) \cup c(B\cup Z)
\end{align*}

The \emph{forward} and \emph{backward} operator are, respectively,  the functions:
\[
\begin{split}
	\avr \colon \mathcal{P}(X)\times \mathcal{P}(X) &\to \mathcal{P}(X) \\
	(A, B)&\mapsto \lfp(\Phi_{(X,c)}^{A, B}) 
\end{split} \qquad 
\begin{split}
	\devr \colon \mathcal{P}(X)\times \mathcal{P}(X) &\to \mathcal{P}(X) \\
	(A, B)&\mapsto \lfp(\Phi_{(X,c)^\circ}^{A, B}) 
\end{split}\]
\end{definition}

\begin{remark}\label{rem:mon}
	Notice that $\avr$ and $\devr$ are monotone in both their arguments: if $A\subseteq C$  and $B\subseteq D$ then
	\[\avr(A, B) \subseteq \avr(C,D) \qquad \devr(A, B) \subseteq \devr(C,D) \]
	
	Indeed for every $Z$ in $\mathcal{P}(X)$ we have
	\[\Phi_{(X,c)}^{A,B}(Z) \subseteq \Phi_{(X,c)}^{C,D}(Z) \qquad \Phi_{(X,c)^\circ}^{A,B}(Z) \subseteq \Phi_{(X,c)^\circ}^{C,D}(Z) \]
	and we get the wanted inequalities from \Cref{rem:fix_mon}.
\end{remark}

\begin{proposition}\label{prop:cam}
cammini
\end{proposition}
\begin{proof}
	contenuto...
\end{proof}

\section{Sheaves of closure spaces}

TO DO: introdurre operatori temporali necessariamente, possibilmente, until, next per i discreti

\section{A spatio-temporal logic for closure spaces}

\subsection{The calculus STLCS}
\subsubsection{Spatial logic for closure spaces}

\subsubsection{Adding temporality}

\subsection{Interpreting STLCS}

\section{Conclusions}
TO DO: vendere 'sta roba parte 2

%
%% Bibliography
%
\bibliographystyle{splncs04}
\bibliography{bibliog}

\clearpage

\appendix

\iffalse 
\section{Some facts about frames}\label{sec:frames}
This appendix is devoted to recalli some definitions and well-known facts about frames and morphism between them. Standard textbook references are \cite{borceux1994handbook3,dikranjan2013categorical,johnstone1982stone,maclane2012sheaves,vickers1989topology}. 

\begin{definition}
	A poset $(H, \leq)$ is a \emph{frame} if:
	\begin{enumerate}
		\item it has finite meets;
		\item it has arbitrary joins;
		\item for every family $(y_i)_{i\in I}\subseteq H$ and $x \in H$ we have:
		\[x\wedge \bigvee_{i\in I} y_i \leq \bigvee_{i\in I} x\wedge y_i\] 
	\end{enumerate}
	
	A morphism of frames $f\colon (H, \leq)\to (K, \leq)$ is a function $f\colon H\to K$ which preserves finite meets and arbitrary joins. We will denote by $\fram$ the resulting category.
\end{definition}

\begin{example}\label{ex:set}
	Given a set $X$, its power set $(\mathcal{P}(X), \subseteq)$ is a frame:  joins and meets are provided by unions and intersections which distribute on each other.
\end{example}

\begin{example}\label{ex:topo}
	Topologies generalize the previous example. Indeed, let $(X, \mathcal{O}_X)$ be a topological spaces. Then $(\mathcal{O}_X, \subseteq)$ has finite meets as $\emptyset\in \mathcal{O}_X$ and the intersection of two opens is open. Since the arbitrary union of open sets is open then $(\mathcal{O}_X, \subseteq)$ also has arbitrary joins. The distributivity of binary meets over joins follows from the fact that intersections distributes over unions.
\end{example}



We can put \Cref{ex:topo,ex:set} in a broader context.
\begin{proposition}\label{prop:fu}
	Let $\topo$ be the category of topological spaces and continuous map. There is a functor $\mathcal{O}\colon \topo^{op}\to \fram$ sending a topological space $(X, \mathcal{O}_X)$ to its topology $(\mathcal{O}_X, \subseteq)$ and a continuous function $f\colon (X, \mathcal{O}_X)\to (Y, \mathcal{O}_Y)$ to
	\begin{align*}
		\mathcal{O}_{f}\colon (\mathcal{O }_Y, \subseteq)&\to (\mathcal{O}_X, \subseteq) \\ U &\mapsto f^{-1}(U)  
	\end{align*} 
\end{proposition}
\begin{proof} By  \Cref{ex:topo} we only have to show that $\mathcal{O}_f$ is a morphism of frames, but this follows immediately from the fact that the preimage function $f^{-1}\colon \mathcal{P}(Y)\to \mathcal{P}(X)$ preserves arbitrary unions and intersections.
\end{proof}

Precomposing with the opposite of the discrete topology functor $D\colon \set \to \topo$ we obtain the following result.

\begin{corollary}\label{cor:fu}
	There exists a functor $\mathcal{P}\colon \set^{op}\to \fram$ sending a set $X$ to $(\mathcal{P}(X), \subseteq)$ and a function $f\colon X\to Y$ to $f^{-1}\colon (\mathcal{P}(Y), \subseteq) \to (\mathcal{P}(X), \subseteq)$.
\end{corollary}

\iffalse 
\begin{remark}\label{rem:nomeet}
	Notice that the functors of \Cref{prop:fu} is nor faithful nor full.  To see that it is not faithful it is enough to notice that every function $f:\mathbb{R}\to \mathbb{R}$ induces a continuous one $(\mathbb{R}, \{\emptyset, \mathbb{R}\})\to (\mathbb{R}, \{\emptyset, \mathbb{R}\})$ between the associate indiscrete spaces. It is immediate that for every such $f$, $\mathcal{O}_f$ is the identity.
	
	Since $D$ is full if $\mathcal{P}$ is full than $\mathcal{O}$ is full too, so any counter-example to the fullness of the former disproves also the fullness of the latter.  Now, for every function $f\colon X\to Y$ we can notice that $f^{-1}$ preserves all meets, but not all morphisms of frames enjoy this property.
	
	Consider the function
	\begin{align*}
		g\colon \mathcal{P}(\mathbb{N})&\to \mathcal{P}(\mathbb{N})\\
		A & \mapsto \begin{cases}
			\mathbb{N} & \mathbb{N}\smallsetminus A \text{ is finite}\\
			\emptyset & \text{otherwise}
		\end{cases}
	\end{align*}
	$g$ does not preserve arbitrary meets: for every $n\in \mathbb{N}$ let $A_n$ be $\mathbb{N}\smallsetminus\{n\}$, then we have:
	\[g(\bigcap_{n\in \mathbb{N}} A_n)=g(\emptyset)=\emptyset\neq \mathbb{N} = \bigcap_{n\in \mathbb{N}} \mathbb{N} =
	\bigcap_{n\in \mathbb{N}}g(A_n)\]
	
	But $g$ defines a morphism of frames $(\mathcal{P}(\mathbb{N}), \subseteq)\to (\mathcal{P}(\mathbb{N}), \subseteq)$. 
	\begin{itemize}
		\item To prove preservation of finite meets, notice that on the one hand $g$ sends $\mathbb{N}$ to itself and so it preserves the empty meet while, on the other hand, given $X, Y \subseteq\mathbb{N}$ we have that:
		\begin{align*}
			g(X\cap Y) &= \begin{cases}
				\mathbb{N} & \mathbb{N}\smallsetminus (X\cap Y)  \text{ is finite}\\
				\emptyset & \text{otherwise}
			\end{cases}= \begin{cases}
				\mathbb{N} &( \mathbb{N}\smallsetminus X) \cup (\mathbb{N}\smallsetminus Y)  \text{ is finite}\\
				\emptyset & \text{otherwise}
			\end{cases}\\&=
			\begin{cases}
				\mathbb{N} &( \mathbb{N}\smallsetminus X) \text{ and } (\mathbb{N}\smallsetminus Y)  \text{ are both finite}\\
				\emptyset & \text{otherwise}
			\end{cases}
		\end{align*}
		and this immediately implies that $g(X\cap Y)=g(X)\cap g(Y)$.
		\item For arbitrary joins, consider a family $\{A_i\}_{i\in I}$ of elements of $\mathcal{P}(\mathbb{N})$, then:
		\begin{align*}
			g(\bigcup_{i\in I} A_i) &= \begin{cases}
				\mathbb{N} & \mathbb{N} \smallsetminus \bigcup_{i\in I} A_i  \text{ is finite}\\
				\emptyset  &\text{otherwise }
			\end{cases} = \begin{cases}
				\mathbb{N} &  \bigcap_{i\in I} \mathbb{N} \smallsetminus A_i  \text{ is finite}\\
				\emptyset  &\text{otherwise }
			\end{cases}\\&= \begin{cases}
				\mathbb{N} &   \text{there exists $k\in I$ such that $\mathbb{N}\smallsetminus A_k$ is finite}\\
				\emptyset  &\text{otherwise }
			\end{cases}\\&=
			\begin{cases}
				\mathbb{N} &   \text{there exists $k\in I$ such that $g(A_k)= \mathbb{N}$}\\
				\emptyset  &\text{otherwise }
			\end{cases}
		\end{align*}
		fromm which preservation of joins follows at once.
	\end{itemize}
	
	Summing up, $g$ is a morphism of frames $(\mathcal{P}(\mathbb{N}), \subseteq) \to (\mathcal{P}(\mathbb{N}), \subseteq)$ which is not in the image of $\mathcal{P}$.
\end{remark}
\fi 
To continue we need to recall the classical version of Freyd 's  Adjoint Functor Theorem \cite{borceux1994handbook1,dikranjan2013categorical,freyd2003abelian,freyd1990categories,mac2013categories} for posets.
\begin{theorem}[Adjoint functor theorem for posets]\label{thm:radj} Consider two posets $(P, \leq)$ and $(Q, \leq)$, with  $(P, \leq)$ complete. If $f\colon P\to Q$ is a function which preserves arbitrary  joins, then $f$ has a left adjoint given by
	\begin{align*}
		g\colon (Q, \leq) &\to (P, \leq)\\q &\mapsto \sup\{p \in P \mid f(p) \leq q\}
	\end{align*}
\end{theorem}
\begin{proof}
	We can start noticing that preservation of joins entails that $f$ is monotone. For every $q$ in $Q$ let us define 
	\[P_q :=\{p \in P \mid f(p) \leq q\} \]
	Since $(P, \leq)$ is complete we can define $g(q)$ as the supremum of $P_q$. We can immediately notice two things:
	\begin{itemize}
		\item  if $q\leq q'$ then $P_q\subseteq P_{q'}$ and so $g(q)\leq g(q')$,  hence $g$ defines a monotone function $(Q, \leq)\to (P, \leq)$.
		\item since $f$ preserves arbitrary joins, then $f(g(q))$ is the supremum of the family
		\[\{f(p)\mid f(p)\leq q\}\]
		and so $f(g(q))\leq q$. 
	\end{itemize}
	
	The previous two observation easily entails that $f\dashv g$:
	\begin{itemize}
		\item if $p\in P$ and $q\in Q$ are such that $f(p)\leq q$, then $p\in P_q$ and so $p\leq g(q)$;
		\item viceversa, suppose that $p\in P$ and $q\in Q$ satisfy $p\leq g(q)$, then we have $f(p)\leq f(g(q))$ and this is less or equal than $q$. \qedhere
	\end{itemize}
\end{proof}

Dualizing \Cref{thm:radj} we get immediately the following.
\begin{corollary}\label{cor:ladj}
	Let $(P, \leq)$ be a complete poset. If $f\colon (P,\leq)\to (Q, \leq)$ is a function which preserves arbitrary  meets, then $f$ has a right adjoint given by
	\begin{align*}
		g\colon (Q, \leq) &\to (P, \leq)\\q &\mapsto \inf\{p \in P \mid f(p) \leq q\}
	\end{align*}
\end{corollary}

In turn, \Cref{thm:radj} gives us other two results.

\begin{corollary}\label{cor:facts}
	Let $(H, \leq)$ be a frame, the following facts are true:
	\begin{enumerate}
		\item $(H, \leq)$ is an \emph{Heyting algebra}, i.e.~for every $h\in H$ the function $h \wedge -$ has a right adjoint $h\Rightarrow -$;
		\item a morphism of frames $f\colon (H, \leq)\to (K, \leq)$ has a right adjoint $f_*$.
	\end{enumerate}
\end{corollary}

Finally, we can use \Cref{cor:ladj} to show that $\mathcal{P}\colon \set^{op} \to \fram$ is full.
\begin{corollary} Consider a morphism of frames $f\colon (\mathcal{P}(X), \subseteq)\to (\mathcal{P}(Y), \subseteq)$, then the following hold true:
	\begin{enumerate}
		\item $f$ preserves arbitrary meets;
		\item $f$ is in the image of $\mathcal{P}\colon \set^{op} \to \fram$.
	\end{enumerate}
\end{corollary}
\begin{proof} 
	\begin{enumerate}
		\item Let $\{A_i\}_{i\in I}$ be a family of subsets of $X$ and let $A$ be their intersection. For every $i\in I$ define:
		\[B_i := \bigcup_{x\in A_i} f(\{x\}) \]
		We claim that
		\[\bigcup_{x\in A} f(\{x\}) = \bigcap_{i\in I} B_i\]
		
		\smallskip \noindent
		$(\subseteq)$ If $z$ is an element of $\bigcup_{x\in A} f(\{x\}$, then there exists $x\in A$ such that $z\in f(\{x\})$. But then $f(\{x\})\subseteq B_i$ for every $i \in I$ and so $z$ belongs to the intersection of $\{B_i\}_{i\in I}$. 
		
		\smallskip \noindent
		$(\supseteq)$ Let $z$ be an element in $\bigcap_{i\in I} B_i$. Then for every $i\in I$ there exists $x_i\in A_i$ such that $z\in f(\{x_i\})$. Now, let $i,j$ be two elements of $I$, if $x_i\neq x_j$ then we have:
		\[f(\{x_i\}) \cap f(\{x_j\}) = f(\{x_i\}\cap \{x_j\}) = f(\emptyset) = \emptyset\]
		We can then deduce that there exists a single $x\in A$ such that $z\in f(\{x\})$, as wanted.
		
		\smallskip 
		Using the previous results we get:
		\begin{align*}
			&f(\bigcap_{i\in I} A_i) = f(A) = f(\bigcup_{x\in A} \{x\}) = \bigcup_{x\in A} f(\{x\}) \\= &\bigcap_{i\in I} B_i = \bigcap_{i\in I} \bigcup_{x\in A_i} f(\{x\}) = \bigcap_{i\in I} f(\bigcup_{x\in A_i} \{x\})= \bigcap_{i\in I} f(A_i)
		\end{align*}
		
		\item By the previous point an \Cref{cor:ladj}, we know that $f$ has a left adjoint $f_{!}\colon (\mathcal{P}(X), \subseteq)\to (\mathcal{P}(Y), \subseteq)$.  We are now going to prove that $f_{!}(\{x\})$ has cardinality $1$ for every $x\in X$.
		
		On the one hand $f_{!}(\{x\})$ cannot be  empty. Indeed, if $f_{!}(\{x\}) \subseteq \emptyset$ then by adjointness we get $\{x\}\subseteq f(\emptyset)$, but $f(\emptyset) = \emptyset$ and this is absurd. On the other hand, by preservation of arbitrary joins we have
		\[\{x\}\subseteq X = f(Y) = f(\bigcup_{y\in Y}\{y\}) = \bigcup_{y\in Y} f(\{y\}) \]
		and so there exists $y\in Y$ such that $\{x\}\subseteq f(\{y\})$ for some $y\in Y$. Hence $f_{!}(\{x\})\subseteq \{y\}$ and so $f_!(\{x\})$ cannot have cardinality bigger than one.
		
		Let us define $g(x)$ as the unique element of $f_{!}(\{x\})$. In this way we get a function $g\colon X\to Y$. Now, for every $y\in Y$, by \Cref{thm:radj}, \Cref{cor:ladj} and uniqueness of adjoints, we have:
		\[g^{-1}(\{y\})= \{x\in X \mid f_{!}(\{x\}) \subseteq \{y\}\}= f(y)\] 
		But then, preservation of unions yields $g^{-1}= f$ as wanted.
	\end{enumerate}
\end{proof}

\begin{example}
	contenuto...
\end{example}

\begin{definition}\label{rem:bool} regolari e
	booleane
\end{definition}


\begin{example}\label{ex:reg}
	contenuto...
\end{example}


\begin{example}\label{ex:top}
	contenuto...
\end{example}


\begin{proposition} riflessione in booleane
\end{proposition}
\begin{proof}
	contenuto...
\end{proof}

\begin{remark}
	not functor
\end{remark}
\begin{definition}
	atomi
\end{definition}

In this paper we are particularly interested in locales arising as power sets. Luckily we have all the ingredients needed to fully characterize them.

\begin{theorem}
	Johnstone
\end{theorem}
\begin{proof}
	contenuto...
\end{proof}

\fi 

\end{document}


%%%%%%%%%%%%%%%%%%%%%%%
%%%%%%% Qui sotto il vecchio paper

\section{Introduction}	
	
	
% It should begin with a succinct statement of the issues, a summary of the main results, and a brief explanation of their significance and relevance to the conference and to computer science, all phrased for the non-specialist. 
	
% Contesto generale, e sua rilevanza per CSL
% i CAS e sistemi nello spazio richiedono logiche che parlano dello spazio.
Recently, much attention has been devoted in Computer Science to systems distributed in physical space; 
a typical example is provided by the so called \emph{collective adaptive systems}, 
%which are usually composed by a large number of computational devices interacting with each other and the surrounding environment; 
such as drone swarms, sensor networks, autonomic vehicles, etc.
%In these systems, the global behaviour emerges by local interactions between objects, which in turn depend on their distribution in physical space.
This arises the problem of how to model and reason formally about spatial aspects of distributed systems. 
To this end, several researchers have advocated the use of \emph{spatial logics}, i.e.~modal logics whose modalities are interpreted using topological concepts of neighbourhood and connectivity.\footnote{Not to be confused with spatial logics for reasoning on the structure of agents, such as the Ambient Logic \cite{CardelliG00} or the Brane Logic \cite{MiculanB06}.}

In fact, the interpretation of modal logics in topological spaces goes back to Tarski;
we refer to \cite{Aiello2007} for a comprehensive discussion of variants and computability and complexity aspects. 
More recently, Ciancia \emph{et al.}~\cite{ciancia2014specifying,ciancia2016spatial} extended this approach to \emph{preclosure spaces}, also called \emph{\v{C}ech closure spaces}, which generalise topological spaces by not requiring idempotence of closure operator.
This generalization unifies the notions of neighbourhood arising from topological spaces and from  \emph{quasi-discrete closure spaces}, like those induced by graphs and images.
Building on this generalization, \cite{ciancia2014specifying} introduced \emph{Spatial Logic for Closure Spaces} (SLCS), a modal logic for the specification and verification on spatial concepts over preclosure spaces.
This logic features a \emph{closure} modality and a spatial \emph{until} modality: intuitively $\phi \mathcal{U} \psi$ holds in an area where $\phi$ holds and it is not possible to ``escape'' from it unless passing through an area where $\psi$ holds.
\ifreport
There is also a \emph{surrounded} constructor, to represent a notion of (un)reachability.  
\fi
SLCS has been proved to be quite effective and expressive, as it has been applied to reachability problems, vehicular movement, digital image analysis (e.g., street maps, radiological images \cite{BelmonteCLM19}), etc. 
The model checking problem for this logic over finite quasi-discrete structures is decidable in linear time \cite{ciancia2014specifying}.

Despite these results, an axiomatisation for SLCS is still missing.
%; this would be useful for identifying the models and the class of properties that the logic can express. 
Moreover, it is not obvious how to extend this logic to other spaces with other closure operators, such as probabilistic automata  (e.g.~Markov chains).
\ifreport
Also, it is not immediate generalizing current definitions of reachability to other cases, e.g., within a given number of steps, or non-deterministic, or probabilistic, etc.
\fi
% Problema specifico affrontato in questo articolo, e sua rilevanza 
More generally, we miss an abstract theoretical framework for investigating the logical aspects of closure spaces. Such a framework would be the basis for analysing the logic SLCS, but also for developing further extensions and applications thereof.
 
% Soluzione proposta 
This is the main aim of this paper: we introduce the new notion of \emph{closure (hyper)doctrine} as the theoretical basis for studying the logical aspects of closure spaces.
Doctrines were introduced by Lawvere \cite{lawvere1969adjointness} as a general way for endowing (the objects of) a category with logical notions from a suitable 2-category $\catname{E}$, which can be the category of Heyting algebras in the case of intuitionistic logic, of Boolean algebras in the case of classical logic, etc..
Along this line, in order to capture the logical aspects of closure spaces we introduce the notion of \emph{closure operators} on doctrines, that is, families of inflationary morphisms over objects of  $\catname{E}$ (subject to suitable conditions); a closure (hyper)doctrine is a (hyper)doctrine endowed with a closure operator.
These structures arise from many common situations: we provide many examples ranging from topology to algebraic structures, from coalgebras to fuzzy sets. These examples cover the usual cases from literature (e.g., graphs, quasi-discrete spaces, (pre)topological spaces) but include also new settings, such as categories of coalgebras and probabilistic frames (i.e., Markov chains).
%This categorical approach is very general, as it can be applied for modeling logics with different features.
Then, leveraging general machinery from categorical logic, we introduce a first order logic for closure spaces for which we provide an axiomatisation and a sound and complete categorical semantics.  The propositional fragment corresponds to the SLCS from \cite{ciancia2014specifying}.

\ifreport
Within this framework, we can accommodate also the notion of \emph{surroundedness} of properties, in order to model spatial operators like SLCS's $\mathcal{S}$ \cite{ciancia2016spatial}.
Actually, surroundedness is not a structural property of the logical domain (differently from closure operators); rather, it depends on the kind of paths we choose to explore the space. To this end, we introduce the notion of \emph{closure doctrine with paths}. Again, the foundational approach we follow allows for many kinds of paths, and hence many notions of surroundedness.
\fi

Overall, the importance of this work is twofold: on one hand, closure hyperdoctrines \ifreport (with paths) \fi are useful for analysing and improving the theory of existing spatial logics; in particular, the proposed axiomatisation can enable both new proof methodologies and minimisation techniques. On the other, closure hyperdoctrines are useful for the definition of new logics in various situations where we have to deal with closure operators, connectivity, surroundedness, etc.

% Contributi principali
%Summarizing, the main contributions of this paper are:
%\begin{enumerate}
%	\item the new notions of \emph{preclosure algebras} and \emph{preclosure hyperdoctrines}, which can be used as a general theoretical framework for constructing models for logics with pre-closure modalities;
%	\item a proof system for (first order) SLCS;
%	\item a sound and complete categorical semantics for this logic in preclosure hyperdoctrines;
%\end{enumerate}

% citare l'interdisciplinarietà?
	
% Sinossi dell'articolo	
\paragraph{Synopsis}
In \cref{sec:closurehd} we recall (hyper)doctrines and introduce the key notion of closure doctrine.
Many examples of closure doctrines are provided in \cref{sec:examples}.
In \cref{sec:slcs} we introduce \emph{logics for closure operators}, together with a sound and complete semantics in closure hyperdoctrines.
\ifreport
Then, in order to cover the notion of surroundedness, we introduce the notion of \emph{closure doctrine with paths} (\cref{sec:paths}), and the corresponding logics with the ``surrounded'' operator (\cref{sec:slcswp}).
\fi
Conclusions and directions for future work are in \cref{sec:concl}.
Longer proofs are in \cref{sec:proofs}.



\section{Closure (hyper)doctrines}
\label{sec:closurehd}

\subsection{Kinds of doctrines}
\label{sec:background}
In this section we recall the notion of elementary hyperdoctrine, due to Lawvere \cite{lawvere1969adjointness,lawvere1970equality}.
The development of semantics of logics in this context or in the equivalent fibrational context is well established; we refer the reader to, e.g., \cite{jacobs1999categorical,makkai2006first,pitts1995categorical}.
%, and to Appendix~\ref{sec:fibration} for a brief introduction to fibrations and the Groethendieck construction.
	

	\begin{definition}[(Existential) Doctrine, Hyperdoctrine]
		A \emph{primary doctrine} or simply a \emph{doctrine} on a category $\catname{C}$ is a functor $\mathscr{P}:\catname{C}^{op}\rightarrow \catname{InfSL}$ where $\catname{InfSL}$ is the category of finite meet semilattices.
		
		A primary doctrine is \emph{existential} if:
		\begin{itemize}
			\item  $\catname{C}$ has finite products;
			\item the image $\mathscr{P}_{\pi_C}$ of any projection $\pi_C: C\times D\rightarrow C$ admits a left adjoint $\exists_{\pi_C}$;
			\item 	\parbox[t]{75mm}{for each pullback like aside, the \emph{Beck-Chevalley condition} $\exists_{\pi_{C'}}\circ \mathscr{P}_{1_D\times f}=\mathscr{P}_f\circ \exists _{\pi_C}$ holds;}
				\begin{tikzpicture}[baseline=(current bounding box.center)]
				\node(A) at(0,2) {$D\times C'$};
				\node(B) at(0,0) {$D\times C$};
				\node(C) at(2,2) {$C'$};
				\node(D) at(2,0) {$C$};
				\draw[->] (A)--(B) node[pos=0.5, left]{$1_D\times f$};
				\draw[->] (A)--(C) node[pos=0.5, above]{$\pi_{C'}$};
				\draw[->] (C)--(D) node[pos=0.5, right]{$f$};
				\draw[->] (B)--(D) node[pos=0.5, below]{$\pi_C$};
				\end{tikzpicture}
			
			\item  for any $\alpha \in \mathscr{P}(C)$ and $\beta \in \mathscr{P}(D\times C)$ the \emph{Frobenius reciprocity}
			$\exists_{\pi_C}(\mathscr{P}_{\pi_C}(\alpha)\wedge \beta)=\alpha \wedge \exists_{\pi_C}(\beta)$
			holds.
		\end{itemize}
		A \emph{hyperdoctrine} is an existential doctrine $\mathscr{P}$ such that:
		\begin{itemize}
			\item $\mathscr{P}$ factors through the category $\catname{HA}$ of Heyting algebras and Heyting algebras morphisms;
			\item for all projections $\pi_C:D\times C\rightarrow C$, $\mathscr{P}_{\pi_C}$ has a right adjoint $\forall_{\pi_C}:\mathscr{P}(D\times C)\to \mathscr{P}(C)$ which must satisfy the Beck-Chevalley condition:
			$\forall_{\pi_{C'}}\circ \mathscr{P}_{1_D\times f}=\mathscr{P}_f\circ \forall _{\pi_C}$
			for any $f:C'\rightarrow C$.
		\end{itemize}
	A primary doctrine, an existential doctrine or a hyperdoctrine, is \emph{elementary} if
	\begin{itemize}
	\item $\catname{C}$ has finite products;
	\item for each object $C$ there exists a \emph{fibered equality}  $\delta_C \in \mathscr{P}(C\times C)$  such that \begin{equation*}
		\mathscr{P}_{(\pi_1, \pi_2)}(-)\wedge \mathscr{P}_{(\pi_2, \pi_3)}(\delta_C) \dashv \mathscr{P}_{1_D\times \Delta_C}
	\end{equation*} 	
	
	where $\pi_1, \pi_2$ and $\pi_3$ are projections $D\times C\times C\rightarrow D\times C$.
	This left adjoint will be denoted by $\exists_{1_D\times \Delta_C}$
	\end{itemize} 
	\end{definition}

	\begin{remark}
		Since $\catname{C}$ has a terminal object it follows that 
		$\mathscr{P}_{\pi_1}(-)\wedge \delta_C \dashv \mathscr{P}_{\Delta_C}$.
		This left adjoint will be denoted by $\exists_{\Delta_C}$.
	\end{remark}

\begin{remark}
	In this paper, we work with hyperdoctrines over $\catname{HA}$, the category of Heyting algebras and their morphisms; hence the resulting logic is inherently intuitionistic.
	Clearly, all the development still holds if we restrict ourselves to the subcategory of Boolean algebras $\catname{BA}$, yielding a classical version of the logic.	
\end{remark}
\begin{example}
	\looseness=-1
	Let $\catname{C}$ be a category with finite limits and $(\mathscr{E}, \mathscr{M})$ a stable and proper factorization system on it (see \cite{kelly1991note}). Fix an object $C\in \catname{C}$  we define a relation on arrows in $\mathscr{M}$ with codomain $C$ putting $m\leq n$ if and only if there exists $t$ such that $n\circ t =m$. If we ignore size issues this gives us a preorder, from which we get a partial order $\msub{C}(C)$ by quotienting by the relation $m\simeq n$ if and only if $m\leq n$ and $n\leq m$. The top element is $[1_C]$, while meets are given by pullbacks, and we can pullback any $m$ along any arrow $f:D\rightarrow C$ getting an arrow $f^*m$ in $\mathscr{M}$ with codomain $D$. Summarizing we have a functor $	\catname{C}^{op}\rightarrow \catname{InfSL}$ sending $C$ to $\msub{C}(C)$.
%\begin{align*}
 %
%	\functor[l]{D}{f}{C}
%	& \functoropmapsto
%	\rfunctorop{\msub{C}(D)}{f^*}{\msub{C}(C)}
%	\end{align*}
	This is actually an elementary existential doctrine in which $\delta_C$ is the class of the diagonal $C\rightarrow C\times C$ (which can be shown to be an element of $\mathscr{M}$) and $\exists_{\pi_C}([m])$ is the $\mathscr{M}$-component of $\pi_C\circ m$, in the sense that it is the class of $n\in \mathscr{M}$ such that  $n\circ e=\pi_C\circ m$ for some $e\in \mathscr{E}$ (see \cite{hughes2003factorization} for the correspondence between factorization systems and elementary existential doctrines).
%\end{example}
%\begin{example}
	 In general this functor  is very far from having Heyting algebras as values but this is the case when $\catname{C}$ is a topos and $\mathscr{M}$ the class of all monomorphisms; in this case we get an elementary hyperdoctrine \cite{maclane2012sheaves}.
\end{example}
	\begin{proposition}\label{trans}
		Let $\mathscr{P}:\catname{C}^{op}\rightarrow \catname{InfSL}$ be an existential doctrine, $\catname{D}$ a category with finite products and $\mathscr{F}:\catname{D}\rightarrow \catname{C}$ a product preserving functor. 
		Then, $\mathscr{P}\circ \mathscr{F}^{op}$ is a existential doctrine.
		If $\mathscr{P}$ is elementary (resp., a hyperdoctrine) then $\mathscr{P}\circ \mathscr{F}^{op}$ is elementary (resp., a hyperdoctrine).
	\end{proposition}
	
	\begin{proposition}\label{lad}
		Let $\mathscr{P}:\catname{C}^{op}\rightarrow \catname{HA}$ be an elementary existential doctrine. For every arrow $f:C\rightarrow D$, the functor $\mathscr{P}_f$ has a left adjoint $\exists_f$ that satisfies the \emph{Frobenius reciprocity}:
		$\exists_{f}(\mathscr{P}_{f}(\beta)\wedge \alpha )=\beta\wedge \exists_{f}(\alpha)$.
		If $\mathscr{P}$ is a hyperdoctrine then $\mathscr{P}_f$ has a right adjoint $\forall_f$ too.
	\end{proposition}

	\begin{definition}
		Let  $\mathscr{P}:\catname{C}^{op}\rightarrow \catname{InfSL}$, $\mathscr{S}:\catname{D}^{op}\rightarrow \catname{InfSL}$ be primary doctrines.
		A morphism $\mathscr{P}\rightarrow \mathscr{S}$ is a pair $(\mathscr{F}, \eta )$ where $\mathscr{F}:\catname{C}\to\catname{D}$ is a functor and $\eta:\mathscr{P}\rightarrow \mathscr{S}\circ \mathscr{F}^{op}$ is a natural transformation.
		
		 $(\mathscr{F}, \eta)$ is a \emph{morphism of elementary doctrines}, or \emph{elementary}, if $\mathscr{F}$ preserves finite products and for any object $C$ of $\catname{C}$,
		 $\eta_{C\times C}(\delta_C)=\mathscr{S}_{(\mathscr{F}(\pi_1), \mathscr{F}(\pi_2))}(\delta_{\mathscr{F}(C)})$.
		  
\noindent\parbox[c]{55mm}{$(\mathscr{F}, \eta)$ is a \emph{morphism of existential doctrine} if for any pair of objects $C,D$ of $\catname{C}$  the  diagram aside commutes.}\hfill
	\begin{tikzpicture}[baseline=(current bounding box.center),scale=0.75]
		\node(E)at(-4,0){$\mathscr{P}(D\times C)$};
		\node(F)at(-4,-1.5){$\mathscr{S}(\mathscr{F}(D\times C))$};
		\node(I)at(-4,-3){$\mathscr{S}(\mathscr{F}(D)\times \mathscr{F}(C))$};
		\node(G)at(0,-3){$\mathscr{S}(\mathscr{F}(D))$};
		\node(H)at(0,0){$\mathscr{P}(C)$};
		\draw[->](E)--(H)node[pos=0.5, above]{$\exists_{\pi_C}$};
		\draw[->](E)--(F)node[pos=0.5, left]{$\eta_{D\times C}$};
		\draw[->](I)--(G)node[pos=0.5, above]{$\exists_{\pi_{\mathscr{F}(C)}}$};
		\draw[<-](I)--(F)node[pos=0.5, left]{$\mathscr{S}_{(\mathscr{F}(\pi_D),\mathscr{F}(\pi_C))^{}}$};
		\draw[->](H)--(G)node[pos=0.5, right]{$\eta_C$};
%		\node(L)at(-2,-1.5){(a)};
	\end{tikzpicture}%

\noindent\parbox[c]{55mm}{$(\mathscr{F}, \eta)$ is a \emph{morphism of hyperdoctrines} if 
		the diagram aside commutes too and any component of $\eta$ preserves finite suprema and implication.}
	\hfill
			\begin{tikzpicture}[baseline=(current bounding box.center),scale=0.75]
			\node(E)at(-4,-5){$\mathscr{P}(D\times C)$};
			\node(F)at(-4,-6.5){$\mathscr{S}(\mathscr{F}(D\times C))$};
			\node(I)at(-4,-8){$\mathscr{S}(\mathscr{F}(D)\times \mathscr{F}(C))$};
			\node(G)at(0,-8){$\mathscr{S}(\mathscr{F}(D))$};
			\node(H)at(0,-5){$\mathscr{P}(C)$};
			\draw[->](E)--(H)node[pos=0.5, above]{$\forall_{\pi_C}$};
			\draw[->](E)--(F)node[pos=0.5, left]{$\eta_{D\times C}$};
			\draw[->](I)--(G)node[pos=0.5, above]{$\forall_{\pi_{\mathscr{F}(C)}}$};
			\draw[<-](I)--(F)node[pos=0.5, left]{$\mathscr{S}_{(\mathscr{F}(\pi_D),\mathscr{F}(\pi_C))^{}}$};
			\draw[->](H)--(G)node[pos=0.5, right]{$\eta_C$};
%			\node(L)at(-2,-6.5){(b)};
			\end{tikzpicture}
		
		If $(F,\eta)$ is elementary then we call it a \emph{morphism of elementary existential doctrines} or of \emph{elementary hyperdoctrines}.
		
	   Let $(\mathscr{F},\eta ), (\mathscr{G}, \epsilon): \mathscr{P}\rightarrow \mathscr{S}$ be two morphisms; a $2$-arrow $(\mathscr{F},\eta )\rightarrow (\mathscr{G}, \epsilon)$ is a natural transformations $\theta:\mathscr{F} \rightarrow \mathscr{G}$ such that
		$\eta_C(\alpha)\leq \mathscr{S}_{\theta_C}(\epsilon_C(\alpha))$.
		
		This defines the $2$-categories $\catname{PD}$, $\catname{ED}$, $\catname{HD}$ of primary doctrines, existential doctrines and hyperdoctrines, and the subcategories $\catname{EPD}$, $\catname{EED}$, $\catname{EHD}$ of their elementary variants.
	\end{definition}



	\subsection{Closure operators on doctrines}
	In this section we introduce the key notion of closure operators on doctrines.
	\begin{definition}
		Let $\mathscr{P}$ be a doctrine.
		A \emph{closure operator} on $\mathscr{P}$ is a (possibly large) family $\mathfrak{c} = \{\mathfrak{c}_C\}_{C\in \catname{Ob(C)}}$ of functions $\mathfrak{c}_C:\mathscr{P}(C)\rightarrow\mathscr{P}(C)$ such that:
		\begin{itemize}
			\item for any object $C$, $\mathfrak{c}_C$ is monotone and \emph{inflationary}, i.e., $1_{\mathscr{P}(C)} \leq \mathfrak{c}_C$
			\item any arrow $f:C\rightarrow D$ is \emph{continuous}, i.e.,
			$\mathfrak{c}_C\circ \mathscr{P}_f \leq \mathscr{P}_f \circ \mathfrak{c}_D$
		\end{itemize}
		A closure operator $\mathfrak{c}$ is said to be
		\begin{itemize}
			\item \emph{grounded} if  $\mathfrak{c}_C(\bot)=\bot$
			for all objects $C$ such that $\mathscr{P}(C)$ has a minimum;
			\item \emph{additive} if $\mathfrak{c}_C(\alpha \vee \beta)=\mathfrak{c}_C(\alpha) \vee \mathfrak{c}_C(\beta)$
			 for all objects $C$ such that $\mathscr{P}(C)$ has binary suprema;
			\item \emph{finitely additive} if it is grounded and additive;
			\item \emph{full additive} if 
			$\mathfrak{c}_C(\bigvee_{i\in I}\alpha_i)=\bigvee_{i\in I}\mathfrak{c}_C(\alpha_i)$
			for all $I\neq \emptyset$ and $C$ such that $\mathscr{P}(C)$ has $I$-indexed suprema;
			\item \emph{idempotent} if 
			$\mathfrak{c}_C\circ \mathfrak{c}_C=\mathfrak{c}_C$
			for all object $C$.
		\end{itemize} 
	
		A \emph{closure doctrine} is a pair $(\mathscr{P}, \mathfrak{c})$ where $\mathscr{P}$ is a primary doctrine and $\mathfrak{c}$ a closure operator on it.
		We say that $(\mathscr{P}, \mathfrak{c})$ is \emph{elementary, existential}, or a \emph{hyperdoctrine},  if $\mathscr{P}$ is. 
	\end{definition}
	
	\begin{remark} Full additivity does not imply groundedness since we explicitly ask for preservation of suprema indexed on non empty set.
	\end{remark}
\begin{proposition}
	\begin{theoremEnd}{prop}\label{image}
		Let $\mathscr{P}\in \catname{EED}$ be an elementary existential doctrine and  $\mathfrak{c}$ a closure operator on it; then, for any $f:C\rightarrow D$, continuity of $f$ is equivalent to
		$\exists_f\circ \mathfrak{c}_C\leq \mathfrak{c}_D\circ \exists_f$.
	\end{theoremEnd}
\end{proposition}
\begin{proof}
	\begin{proofEnd}
	Let's compute:
		\begin{gather*}
				\mathfrak{c}_C\circ \mathscr{P}_f \leq \mathscr{P}_f\circ \exists_{f}\circ \mathfrak{c}_C\circ \mathscr{P}_f  \leq \mathscr{P}_f \circ \mathfrak{c}_D\circ \exists_{f}\circ \mathscr{P}_f \leq \mathscr{P}_f\circ \mathfrak{c}_D
				\\
		\exists_{f}\circ \mathfrak{c}_C  \leq \exists_{f}\circ \mathfrak{c}_C\circ \mathscr{P}_f\circ \exists_{f} \leq \exists_{f}\circ \mathscr{P}_f\circ \mathfrak{c}_D\circ \exists_{f} \leq \mathfrak{c}_D\circ\exists_{f}
		\end{gather*}\qedhere
	\end{proofEnd} \qedhere
\end{proof}
If we think to a morphism of (primary, existential, elementary, hyper)doctrines $(\mathscr{F}, \eta):\mathscr{P}\rightarrow \mathscr{Q}$ as a  `translation' of `types' and `predicates' then, when closure operators are available, it is natural to ask for this `translation' to take place in a continuous way.
	\begin{definition}
		A \emph{morphism of closure (elementary, existential, hyper)doctrines} $(\mathscr{F},\eta):(\mathscr{P}, \mathfrak{c})\rightarrow (\mathscr{Q}, \mathfrak{d})$ is a morphism of (elementary, existential, hyper)doctrines $\mathscr{F}:\mathscr{P}\to\mathscr{Q}$ such that $\eta$ is \emph{continuous}, i.e., for all $C$:
		$\mathfrak{d}_{\mathscr{F}(C)}\circ \eta_C \leq \eta_C \circ \mathfrak{c}_C$.
		We say that it is \emph{open} if equality holds for all the objects $C$.
		A $2$-cell $\theta:(\mathscr{F},\eta)\rightarrow (\mathscr{G}, \epsilon)$ is defined as in the case of doctrines. In this way we get the $2$-categories $\catname{cPD}$, $\catname{cED}$, $\catname{cHD}$ of closure doctrines, closure existential doctrines, closure hyperdoctrines and the subcategories $\catname{cEPD}$, $\catname{cEED}$, $\catname{cEHD}$ of their elementary variants.
	\end{definition}



	\section{Examples of closure hyperdoctrines}\label{sec:examples}
		
	\subsection{Topological examples}
	As a first class of examples, we introduce three closure hyperdoctrines starting from the usual category $\catname{Top}$ of topological spaces and continuous maps.
	The first one corresponds to the \emph{closure spaces} used in, e.g., \cite{ciancia2014specifying,ciancia2016spatial,galton2003generalized}.
	\begin{definition}\label{def:pretopspaces}
		The category $\catname{PrTop}$ of  \emph{pretopological spaces} (or  \emph{closure spaces}) is the category in which:
		\begin{itemize}
			\item objects are pairs $(X,\mathfrak{c})$ of a set $X$ and a monotone function $\mathfrak{c}:\mathcal{P}(X)\rightarrow\mathcal{P}(X)$ such that $1_{\mathcal{P}(X)}\leq \mathfrak{c}$
			and $\mathfrak{c}$ preserves finite (even empty) suprema;
			
			\item an arrow $f:(X,\mathfrak{c}_X)\rightarrow (Y,\mathfrak{c}_Y)$ is a function $f:X\rightarrow Y$ such that $f^{-1}:(\mathscr{P}(Y), \mathfrak{c}_Y)\rightarrow (\mathscr{P}(X), \mathfrak{c}_X)$ is continuous.
		\end{itemize}
	\end{definition}
	
Another example is given by so called \emph{convergence spaces} (cfr. \cite{dikranjan2013categorical}).	
	\begin{definition}
		For any set $X$ let $\catname{Fil}(X)$ be the set of proper filters (i.e., $\emptyset$ is not among them) on it.
	   The category $\catname{FC}$ of \emph{filter convergence spaces} is the category in which: \begin{itemize}
			\item an object is a pair $(X,q_X)$ given by a set $X$ and a function
			$q_X:X\rightarrow \mathcal{P}(\catname{Fil}(X))$
			such that,  for any $x\in X$,
			$q_X(x)$ is upward closed and $\dot{x}:=\{A\subset X \mid x\in A\}$ belongs to $q_X(x)$. 
			\item an arrow $f:(X,q_x)\rightarrow (Y,q_Y)$ is a function $f:X\rightarrow Y$ such that the filter $f(F)$ generated by the images of $F$'s elements belongs to $q_Y(f(x))$ whenever $F\in q_X(x)$.
		\end{itemize}
	\end{definition}
	
	\begin{proposition}
		The obvious forgetful functors from $\catname{Top}$, $\catname{PrTop}$ and $\catname{FC}$ to $\catname{Set}$ preserves finite products.
	\end{proposition}
	\begin{proof}
		For $\catname{Top}$ it is clear, for the other two categories see \cite[Ch.3]{dikranjan2013categorical}.
	\end{proof}
	By \cref{trans} and the previous one, we have three elementary hyperdoctrines
	\begin{equation*}
	\mathscr{P}^t:\catname{Top}^{op}\rightarrow \catname{HA}\quad \mathscr{P}^p:\catname{PrTop}^{op}\rightarrow \catname{HA}\quad \mathscr{P}^f:\catname{FC}^{op}\rightarrow \catname{HA}
	\end{equation*}
	which we now endow with closure operators.
	
	\begin{definition}\label{def:topclosures}
		We define the following closure operators:
		\begin{enumerate}
			\item the \emph{Kuratowski closure operator} $k=\{k_{(X, \theta)}\}_{(X,\theta)\in \catname{Ob}(\catname{Top})}$ on $\mathscr{P}^t$ where $k_{(X, \theta)}$ is the closure operator associated with the topology $\theta$;
			\item the \emph{\v{C}ech closure operator} $c=\{c_{(X, \mathfrak{c})}\}_{(X,\mathfrak{c})\in \catname{Ob}(\catname{PrTop})}$ on $\mathscr{P}^p$ where $c_{(X, \mathfrak{c})}$ is just $\mathfrak{c}$;
			\item the \emph{Kat\v{e}tov closure operator} $\mathfrak{k}=\{\mathfrak{k}_{(X,q_X)}\}_{(X,\mathfrak{q_X})\in \catname{Ob}(\catname{FC})}$ on $\mathscr{P}^f$ where
			\begin{align*}
			\mathfrak{k}_{(X,q_X)}:\mathcal{P}(X) & \rightarrow \mathcal{P}(X)\\
			A & \mapsto \{x\in X\mid \exists F\in q_X(x).A\in F \}
			\end{align*}
		\end{enumerate}
	\end{definition}
	
	\begin{proposition}[{\cite[Ch.3]{dikranjan2013categorical}}]
	\begin{theoremEnd}{prop}\label{prtop}
		\begin{enumerate}
			\item $k$, $c$ and $\mathfrak{k}$ are grounded and additive closure operators, moreover $k$ is idempotent.
			\item There exists a sequence of inclusion functors 
			$\catname{Top}\xrightarrow{\mathscr{i}} \catname{PrTop}\xrightarrow{\mathscr{j}} \catname{FC}$
			each of which has a left adjoint.
			\item 	We have a sequence $(\mathscr{P}^t, k)\xrightarrow{(\mathscr{i}, \eta)}	(\mathscr{P}^p, c)	\xrightarrow{(\mathscr{j}, \epsilon)}	(\mathscr{P}^f, \mathfrak{k})$
			of morphisms in $\catname{cEHD}$ where $\eta$ and $\epsilon$ have identities as components.
		\end{enumerate}
	\end{theoremEnd}
\end{proposition}
\begin{proof}
	\begin{proofEnd}%\hspace{1pt}
		\begin{enumerate}
			\item For $k$ and $c$ the proposition is obvious, let us examine $\mathfrak{k}$:
			since $\dot{x}\in q_X(x)$ then $A\subset \mathfrak{k}_X(A)$, if $A\subset B$ then any filters that contains the former contains the latter too and this implies monotonicity, groundedness follows from the fact that $\emptyset$ does not belong to any proper filter, for additivity we can complete any filter $\mathcal{F}$ to which $A\cup B$ belong to an ultrafilter $\mathcal{U}$ that belongs to $q_X(x)$ since the latter is upward closed, either $A$ or $B$ must belong to $\mathscr{U}$ and we are done.
			\item $\mathscr{i}$ sends a topological space to the pretopological space given by the closure operator associate to its topology, $\mathscr{j}$ sends $(X, \mathfrak{c})$ to $(X,q^{\mathfrak{c}}_X)$ where
			\begin{align*}
			q^{\mathfrak{c}}_X:X & \rightarrow \mathcal{P}(\catname{Fil}(X))\\
			x & \mapsto \{\mathcal{F}\in\catname{Fil}(X) \mid \mathcal{V}_x\subset \mathcal{F} \}
			\end{align*}
			where $\mathcal{V}_x:=\{S \subset X \mid x\notin \mathfrak{c}(X\smallsetminus S)\}$.
			For the left adjoints see \cite{dikranjan2013categorical}.
			\item This is obvious.\qedhere
		\end{enumerate}
	\end{proofEnd} \qedhere
\end{proof}
	
	For many other examples of closure operators on topological spaces we refer the interested reader to \cite{dikranjan2013categorical}.
	
	
	\subsection{Algebraic examples}	

\begin{proposition}
	\begin{theoremEnd}{prop}
		Let $\catname{Grp}$ be the category of groups and $\catname{CRing}$ that of commutative, unital rings (where we require that $f(1_A)=1_B$ for any $f:A\rightarrow B$).
		Then, $\sub{Grp}$ and $\sub{CRing}$ are elementary existential doctrines.
	\end{theoremEnd}
\end{proposition}
\begin{proof}
	\begin{proofEnd}
		The existence of products in any of the two categories is clear, $f:G\rightarrow H$ is a morphism of  groups then $f^{-1}(K)$ is a subgroup for any $K\leq H$; if $g:A\rightarrow B$ is an arrow in $\catname{CRing}$ and $C$ a subring of $B$ then $0_A$, $1_A\in f^{-1}(C)$ and it is closed under sums and products. So $\sub{Grp}$ and $\sub{CRing}$ are functors, now, the intersection of any two subgroups or subrings is again a subgroups or a subring and for any $G$ with unit $e_G$, $\{e_G\}$ is the minimal subgroup, while for any ring $A$ its characteristic subring $A^\dagger$ is the minimal subring, so the codomain of this two functors is $\catname{InfSL}$. Since the image of a subgroup or a subring is a subgroup or a subring we can define the left adjoint $\exists_{f}$ as images.\qedhere 
	\end{proofEnd} \qedhere
\end{proof}
	\begin{remark}
		Notice that, even if $\sub{Grp}(G)$ and $\sub{CRing}(A)$ admit finite suprema for any group $G$ or commutative ring $A$ with unity, preimages do not preserve them in general: for instance they do not preserve the bottom subobject. Then $\sub{Grp}$ or $\sub{CRing}$ cannot be universal doctrines.
	\end{remark}


	The following examples are taken from \cite{dikranjan2013categorical}.		
		\begin{definition}[Groups] The \emph{normal closure} on a group $G$ is given by
			\begin{align*}
			\nu_G:\sub{Grp}(G) & \rightarrow \sub{Grp}(G)\\
			H & \mapsto \bigcap\{N\leq G\mid H\leq N \trianglelefteq G\}
			\end{align*}
			where we have chosen the image of a monomorphism as a canonical representative of it.
		\end{definition}
	\begin{proposition}	
		\begin{theoremEnd}{prop}
			The family previous defined forms a closure operators $\nu$ on $\sub{Grp}$that is idempotent, fully additive and grounded.
		\end{theoremEnd}
\end{proposition}	 
	\begin{proof}
		\begin{proofEnd}
			Since the preimage of a normal subgroup is normal we have that the $\nu$ actually exists as a closure operator. The three poperties of it follow immediately by the fact that $\{0\}$ is normal and so are the arbitrary intersections or sums of normal subgroups.\qedhere
		\end{proofEnd} \qedhere
		\end{proof}
		\begin{definition}[Rings]
			Let $A$ be a unital commutative ring and $B$ a subring, we define $\integ{A}(B)$ to be the \emph{integral closure of $B$}:
			\begin{equation*}
			\integ{A}(B):=\{a\in A\mid p(a)=0 \ \text{for some} \ p\in B[x]\}
			\end{equation*}
			Again we are denotating a subobject by the image of any representative of it.
		\end{definition}
	\begin{proposition}
		\begin{theoremEnd}{prop} For any $A$ $\integ{A}$ is a function $\sub{CRing}(A)\rightarrow\sub{CRing}(A)$, moreover the family of this functions forms an idempotent closure operator $\integ{}$. 
		\end{theoremEnd}
\end{proposition}	
\begin{proof}
		\begin{proofEnd}
			To show that $\integ{A}(B)$ is a subring of $A$ and idempotency we refer to \cite[Cor.~5.3, 5.5]{atiyah2018introduction}.
			Let us show that $\integ{}$ is actually a closure operator. Consider $f:A\rightarrow B$ and $C$ a subring of $B$, let $a\in A$ such that $p(a)=0$ for some $p\in f^{-1}(C)[X]$ with coefficients $\{p_i\}_{i=0}^{\deg(p)}$, then $q(f(a))=0$ where $q\in C[X]$ has coefficients $\{f(p_i)\}_{i=0}^{\deg(p)}$ and we are done.\qedhere
		\end{proofEnd} \qedhere
\end{proof}
	\subsection{A representable example}\label{sec:reprex}
	\begin{proposition}
		$\catname{Set}(-, [0,1]):\catname{Set}^{op}\to\catname{HA}$ is an elementary hyperdoctrine on $\catname{Set}$.
	\end{proposition}
	\begin{proof}
		$[0,1]$, with the usual ordering, is a boolean algebra, hence a Heyting algebra and so, by Yoneda lemma, 	$\catname{Set}(-, [0,1])$ factors through $\catname{HA}$. For $f:X\times Y\rightarrow [0,1]$ we can define
		\begin{equation*}
		\begin{aligned}
		\exists_{\pi_X}(f):X & \rightarrow [0,1]\\
		x & \mapsto \bigvee_{y\in Y}f(x,y)
		\end{aligned}\qquad 
		\begin{aligned}
		\forall_{\pi_X}(f):X & \rightarrow [0,1]\\
		x & \mapsto \bigwedge_{y\in Y}f(x,y)
		\end{aligned}
		\end{equation*}
		Frobenius reciprocity comes for free, for the Beck-Chevalley conditions fix $f:Y\rightarrow Z$, another set $X$ and $g:X\times Z\rightarrow [0,1]$ and compute:
		\begin{equation*}
		\begin{split}
		\exists_{\pi_Y}(g\circ (1\times f))(y)&=\bigvee_{x\in X}g(x, f(y))\\&=\exists_{\pi_Z}(g)(f(y))\\&=(\exists_{\pi_Z}(g)\circ f)(y)
		\end{split}\qquad 
		\begin{split}
		\forall_{\pi_Y}(g\circ (1\times f))(y)&=\bigwedge_{x\in X}g(x, f(y))\\&=\forall_{\pi_Z}(g)(f(y))\\&=(\forall_{\pi_Z}(g)\circ f)(y)
		\end{split}
		\end{equation*}
		The fibered equality $\delta_X:X\times X\rightarrow [0,1]$ is defined as usual
	    $(x,y)\mapsto \begin{cases}
		0 & x\neq y\\
		1 &x=y
		\end{cases}$. \qedhere
	\end{proof}
	\begin{definition}
		For any fixed  real $\epsilon \geq 0$, and any set $X$ we define, for an $f:X\rightarrow [0,1]$ we define
		\begin{gather*}
		\begin{aligned}
		\mathfrak{c}_{X, \epsilon}(f):X & \rightarrow [0,1]\\
				x & \mapsto f(x)\dot{+}\epsilon
		\end{aligned}
	   \quad \text{where} \quad
		\begin{aligned}
		\dot{+}:[0,1]\times [0,1]&\rightarrow [0,1]\\
		(t,s)& \mapsto \max(t+s, 1)
		\end{aligned}
		\end{gather*}
		In this way we get a function
		\begin{align*}
		\mathfrak{c}_{X, \epsilon}: \catname{Set}(X, [0,1]) & \rightarrow \catname{Set}(X, [0,1])\\
		f & \mapsto 	\mathfrak{c}_{X, \epsilon}(f)
		\end{align*}
	\end{definition}
\begin{proposition}
	\begin{theoremEnd}{prop}
		For any $\epsilon \geq 0$, the collection $\mathfrak{c}_\epsilon$ of all the functions $\mathfrak{c}_{X, \epsilon}$ is a closure operator.
	\end{theoremEnd}
\end{proposition}
\begin{proof}
	\begin{proofEnd}
		Clearly $f\leq\mathfrak{c}_{X,\epsilon}(f)$ for any $f:X\rightarrow [0,1]$, monotonicity is clear, let's check continuity of any function $g:X\rightarrow Y$:
		\begin{align*}
		\mathfrak{c}_{X,\epsilon}(f\circ g)(x)
			&=	(f\circ g)(x)\dot{+}\epsilon \\
		&= f(g(x))\dot{+}\epsilon \\
		&=\mathfrak{c}_{x,\epsilon}(f)(g(x))\\&=(\mathfrak{c}_{x,\epsilon}(f)\circ g)(x)
		\qedhere
		\end{align*}
	\end{proofEnd} \qedhere
\end{proof}
\begin{remark}
	 $\mathfrak{c}_{\epsilon}$ is not grounded if $\epsilon \neq 0$ (in that case it reduces to the discrete closure operator) but it is additive.\end{remark}
	
	
	\subsection{Fuzzy sets}
	We can refine the previous example considering \emph{fuzzy sets}.
	\begin{definition}
		The category \emph{$\catname{Fzs}$ of fuzzy sets} has:
		\begin{itemize}
			\item pairs $(A, \alpha)$ with $\alpha:A\rightarrow [0,1]$ as objects;
			\item as arrows $f:(A, \alpha)\rightarrow (B, \beta)$ functions $f:A\rightarrow B$ such that 
			$\alpha(x)\leq \beta(f(x))$.
		\end{itemize}
	\end{definition}
	%\begin{remark}
	%	$\catname{Fzs}$ coincides with $\int \catname{Set}(-, [0,1])$.
	%\end{remark}
	\begin{definition}
		A \emph{fuzzy subset} of $(A, \alpha)$ is a function $\xi:A\rightarrow [0,1]$ such that
		$\xi(x)\leq \alpha(x)$	for all $x\in A$.
	\end{definition}
	Let us summarize some results about $\catname{Fzs}$.
	\begin{proposition}
%		The following are true:
		\begin{enumerate}
			\item $\catname{Fzs}$ is a quasitopos; 
			\item there exists a proper and stable factorization system given by strong monomorphisms and epimorphisms;
			\item fuzzy subsets of $(A,\alpha)$ correspond to equivalence of strong monomorphisms of codomain $(A, \alpha)$;
			\item the functor 
			\begin{align*}
			\catname{Fzs}^{op}& \rightarrow \catname{HA}\\
			\functor[l]{(A, \alpha)}{f}{(B, \beta)}
			& \functoropmapsto
			\rfunctorop{\fuz(A, \alpha)}{f^*}{\fuz(B, \beta)}
			\end{align*}
			where $\fuz(A, \alpha)$ is the set of fuzzy subsets of $(A, \alpha)$ and 
			\begin{align*}
			f^*(\xi):A&\rightarrow [0,1]\\
			x&\mapsto \alpha(x)\wedge \xi(f(x))
			\end{align*}
			for any $\xi \in \fuz(B, \beta)$,
			is an elementary hyperdoctrine.
		\end{enumerate}
	\end{proposition}
	\begin{proof}
		See \cite[Ch.~8]{oswald1991lecture}. Explicitly the hyperdoctrine structure is given by:
		\begin{gather*}
		\begin{aligned}
		\exists_f(\xi):B&\rightarrow [0,1]\\
		y &\mapsto \bigvee_{x\in f^{-1}(y)}\xi(x)
		\end{aligned}\qquad
		\begin{aligned}
		\forall_f(\xi):B&\rightarrow [0,1]\\
		y &\mapsto \beta(y) \wedge \bigwedge_{x\in f^{-1}(y)}(\alpha(x)\Rightarrow \xi(x))
		\end{aligned}
		\end{gather*}
		for any $f:(A, \alpha)\rightarrow (B, \beta)$ and $\xi \in \fuz(A, \alpha)$.
		\qedhere
	\end{proof}
	\begin{remark}
		Implication in $[0,1]$ is given by:
		\begin{equation*}
		t \Rightarrow s = \begin{cases}
		1 &t\leq s\\
		s &s < t
		\end{cases}
		\end{equation*}
		Moreover the fibered equality for a fuzzy set $(A,\alpha)$ must be $\exists_{\Delta_{(A,\alpha)}}(\alpha)$, i.e.:
		\begin{align*}
		\delta_{(A, \alpha)}:A\times A & \rightarrow [0,1]\\
		(x,y) & \mapsto \begin{cases}
		\alpha (x) &x=y\\
		0 &x\neq y
		\end{cases}
		\end{align*}
		Notice that in $\catname{Fzs}$,  $(A, \alpha)\times (B, \beta)$ is $(A\times B, \alpha \wedge \beta)$.
	\end{remark}

\begin{proposition}	
	\begin{theoremEnd}{prop} 
		Let $\mathscr{E}=\{\epsilon_{(A, \alpha)}\}_{(A,\alpha)\in \catname{Ob(Fzs)}}$ be a family of functions $\epsilon_{(A, \alpha)}:(A,\alpha)\rightarrow [0,1]$ such that, for any $f:(A,\alpha)\rightarrow (B,\beta)$
		\begin{equation*}
		\epsilon_{(A, \alpha)}(x)\leq \epsilon_{(B,\beta)}(f(x))
		\end{equation*}
	then, we get an additive closure operator on $\fuz$ defined as follows:
	 \begin{align*}
		\mathfrak{c}^{\mathscr{E}}_{(A,\alpha)}: \fuz(A,\alpha)&\rightarrow \fuz(A,\alpha)\\
		\xi &\mapsto (\xi+\epsilon_{(A, \alpha)})\wedge \alpha
		\end{align*}
	\end{theoremEnd}
\end{proposition}
\begin{proof}
	\begin{proofEnd}
		We have to show continuity of all arrows $f:(A,\alpha)\rightarrow (B,\beta)$. Let $\xi\in (B,\beta)$ and $x\in A$, we have four cases:
		\begin{enumerate}
			\item $f^*(\xi)(x) + \epsilon_{(A,\alpha)}(x)< \alpha(x)$ and $\xi(x) + \epsilon_{(B,\beta)}(x)< \beta(x)$. 
			\begin{align*}
			(\mathfrak{c}^{\mathscr{E}}_{(A,\alpha)}(f^*(\xi)))(x)&=(f^*(\xi) + \epsilon_{(A,\alpha)})(x)\\&=(\alpha(x)\wedge \xi(f(x)))+\epsilon_{(A, \alpha)}(x)\\&=\alpha(x)\wedge (\xi(f(x))+\epsilon_{(A, \alpha)}(x))\\&\leq \alpha(x)\wedge (\xi(f(x))+\epsilon_{(B,\beta)}(f(x)))\\&=f^*(\mathfrak{c}^{\mathscr{E}}_{(B,\beta)}(\xi))(x)
			\end{align*}
			\item $f^*(\xi)(x) + \epsilon_{(A,\alpha)}(x)< \alpha(x)$ and $\xi(f(x)) + \epsilon_{(B,\beta)}(f(x))\geq  \beta(f(x))$. Notice that $\alpha(x)\leq\beta(f(x))$ so
			\begin{equation*}
			f^*(\mathfrak{c}^{\mathscr{E}}_{(B,\beta)}(\xi))(x)=\alpha(x)
			\end{equation*}
			from which:
			\begin{align*}
			(\mathfrak{c}^{\mathscr{E}}_{(A,\alpha)}(f^*(\xi)))(x)&=(f^*(\xi) + \epsilon_{(A,\alpha)})(x)\\&=(\alpha(x)\wedge \xi(f(x)))+\epsilon_{(A, \alpha)}(x)\\&=\alpha(x)\wedge  (\xi(f(x))+\epsilon_{(A, \alpha)}(x))\\&=\alpha(x)\\&=f^*(\mathfrak{c}^{\mathscr{E}}_{(B,\beta)}(\xi))(x)
			\end{align*}
			\item $f^*(\xi)(x) + \epsilon_{(A,\alpha)}(x)\geq \alpha(x)$ and $\xi(x) + \epsilon_{(B,\beta)}(x)< \beta(x)$.
			\begin{align*}
			(\mathfrak{c}^{\mathscr{E}}_{(A,\alpha)}(f^*(\xi)))(x)&=\alpha(x)\\&=\alpha(x)\wedge  (\xi(f(x))+\epsilon_{(A, \alpha)}(x))\\&\leq \alpha(x)\wedge (\xi(f(x))+\epsilon_{(B,\beta)}(f(x)))\\&=f^*(\mathfrak{c}^{\mathscr{E}}_{(B,\beta)}(\xi))(x)
			\end{align*}
			\item $f^*(\xi)(x) + \epsilon_{(A,\alpha)}(x)\geq \alpha(x)$ and $\xi(x) + \epsilon_{(B,\beta)}(x)\geq \beta(x)$. \begin{align*}
			(\mathfrak{c}^{\mathscr{E}}_{(A,\alpha)}(f^*(\xi)))(x)&=\alpha(x)\\&=\alpha(x)\wedge\beta(f(x))\\&=f^*(\mathfrak{c}^{\mathscr{E}}_{(B,\beta)}(\xi))(x)
			\end{align*}
			We are left with additivity, but this follows immediately since, for $\xi$ and $\zeta\in \fuz(A,\alpha)$ and $x\in A$ 
			$(\xi \vee \zeta)(x)$ is $\xi(x)$ or $\zeta(x)$.
			\qedhere
		\end{enumerate}
	\end{proofEnd} \qedhere
\end{proof}
\begin{remark}
	 $\mathfrak{c}^{\mathscr{E}}$ is not grounded in general.
\end{remark}	

The condition on the elements of $\mathscr{E}$ is very restrictive. In fact, it can be eased restricting to a suitable subclass of arrows and using the following lemma.
	\begin{lemma}\label{rest}
		Let $\mathscr{P}:\catname{C}^{op}\rightarrow \catname{InfSL}$ be a  doctrine, and $\mathfrak{c}=\{\mathfrak{c}_C:\mathscr{P}(C)\rightarrow \mathscr{P}(C)\}_{C\in \catname{Ob(C)}}$ be a family of monotone and inflationary operators.
		Let $\mathscr{A}$ be a (possibly large) family of \catname{C}-arrows such that:
		\begin{itemize}
			\item $\mathscr{A}$ is closed under composition;
			\item if $f\in \mathscr{A}$ then $1_{\dom(A)}$ and $1_{\cod(A)}$ are in $\mathscr{A}$;
			\item $f:C\rightarrow D$ in $\mathscr{A}$ implies
			$\mathfrak{c}_C\circ \mathscr{P}_f\leq \mathscr{P}_f \circ \mathfrak{c}_D$.
		\end{itemize}
		Then $\mathscr{P}$ induces a doctrine $\mathscr{P}^{\mathscr{A}}$ on the subcategory $\catname{C}_{\mathscr{A}}$ induced by $\mathscr{A}$ for which $\mathfrak{c}=\{\mathfrak{c}_C\}_{C\in \catname{Ob(C}_\mathscr{A}\catname{)}}$ is a closure operator.
		Moreover, if for all $f,g$ in $\mathscr{A}$ also $(f, g)$ and the projections from $\cod(f)\times \cod(g)$ are in $\mathscr{A}$, 
	then $\mathscr{P}^{\mathscr{A}}$ is existential, elementary or an hyperdoctrine if $\mathscr{P}$ is. 
	\end{lemma}
	\begin{proof}
		This is almost tautological since the condition on $\mathscr{A}$ guarantee that the inclusion functor $\catname{C}_{\mathscr{A}}$ preserves limits and we can use \cref{trans}.\qedhere
	\end{proof}
	
	\subsection{Coalgebraic examples}
	
	\begin{definition}[\cite{kupke2011coalgebraic,jacobs2017introduction}]
		Let $\catname{C}$ be a category with finite products and $\mathscr{F}:\catname{C}\rightarrow \catname{C}$ an endofunctor. 
		The category $\coalg{F}$ of \emph{coalgebras for $\mathscr{F}$} has
		\begin{itemize}
			\item arrows $\gamma_C:C\rightarrow \mathscr{F}(C)$ as objects;
			\item arrows $f:C\rightarrow D$ such that $ \gamma_D\circ f = \mathscr{F}(f)\circ\gamma_C$
%			\begin{center}
%				\begin{tikzpicture}
%				\node(A)at (0,0) {$C$};
%				\node(B)at (3,0) {$D$};
%				\node(C)at (0,-1.5) {$\mathscr{F}(C)$};
%				\node(D)at (3,-1.5) {$\mathscr{F}(D)$};
%				\draw[->](A)--(C) node[pos=0.5, left]{$\gamma_C$};
%				\draw[->](B)--(D) node[pos=0.5, right]{$\gamma_D$};
%				\draw[->](A)--(B) node[pos=0.5, above]{$f$};
%				\draw[->](C)--(D) node[pos=0.5, below]{$\mathscr{F}(f)$};
%				\end{tikzpicture}
%			\end{center}
%			commutes,
		 as morphisms $f:\gamma_C\rightarrow \gamma_D$.
		\end{itemize}
	\end{definition}
	Notice that in general $\coalg{F}$ is not complete and products in it can be very different from products in $\catname{C}$ \cite{gumm2001products}, so it does not make much sense to look for an existential doctrine on it. However, for $\catname{Set}$-based coalgebras we get a primary doctrine $\mathscr{P}^c:\coalg{F}^{op}\rightarrow \catname{InfSL}$ composing the contravariant power object $\mathscr{P}:\catname{Set}^{op}\rightarrow \catname{InfSL}$ with the opposite of the obvious forgetful functor $\coalg{F}\rightarrow \catname{Set}$.
	
	\begin{definition} Let $\mathscr{F}:\catname{C}\rightarrow \catname{C}$ be a functor and $\mathscr{P}$ a primary doctrine on $\catname{C}$.
		A \emph{predicate lifting} is a natural transformation $\Box: \mathscr{U}\circ\mathscr{P} \rightarrow \mathscr{U}\circ \mathscr{P}\circ \mathscr{F}^{op}$ where $\mathscr{U}$ is the forgetful functor $\catname{InfSL}\rightarrow \catname{Poset}$.
	\end{definition}
		
Let $\Box$ be a predicate lifting.
We define two closure operators on $\mathscr{P}^c$.
	\begin{enumerate}
		\item For any coalgebra $\gamma_X:X\rightarrow \mathscr{F}(X)$, 
		notice that $\mathscr{P}^c(\gamma_X) = \mathscr{P}(X)$; hence we can define
		\vspace{-1.5ex}
			\begin{align*}
			\diam{X}:\mathscr{P}(X)&\rightarrow \mathscr{P}(X)\\
			\alpha &\mapsto \alpha \vee \mathscr{P}_{\gamma_X}( \Box_X (\alpha))
			\end{align*}
			\item Suppose that $\mathscr{P}$ admits arbitrary meets; for $\gamma_X:X\rightarrow \mathscr{F}(X)$ and $\alpha\in \mathscr{P}(X)$ we define
			\begin{gather*}
			\mathsf{s}_{\gamma_X}(\alpha):= \bigwedge_{\beta\in\mathcal{N}_{\gamma_X}(\alpha)}\beta
			\qquad \text{ where }
			\mathcal{N}_{\gamma_X}(\alpha):=\{\beta \in \mathscr{P}(X)\mid \alpha \leq 	\mathscr{P}_{\gamma_X}(\Box_X(\beta))\}
			\end{gather*}
			 where  $
			\mathcal{N}_{\gamma_X}(\alpha):=\{\beta \in \mathscr{P}(X)\mid \alpha \leq 	\mathscr{P}_{\gamma_X}(\Box_X(\beta))\}$
			Now we set: %\vspace{-2ex}
			\begin{align*}
			\scat{X}:\mathscr{P}(X) & \rightarrow \mathscr{P}(X)\\
			\alpha & \mapsto \alpha\vee \mathsf{s}_{\gamma_X}(\alpha)
			\end{align*}
		\end{enumerate}	
	\begin{lemma}
	\begin{theoremEnd}{lem} Let $\mathscr{F}:\catname{C}\rightarrow \catname{C}$ be a functor and $\Box$ a predicate lifting, then:
		\begin{enumerate}
			\item $\{\diam{X}\}_{\gamma_X \in \catname{Ob}(\coalg{F})}$ defines a closure operator $\diamo$ on $\mathscr{P}^c$.
			\item  $\mathsf{s}_{\gamma_X}(\alpha)$ is the minimum of $\mathcal{N}_{\gamma_X}(\alpha)$ whenever $\mathscr{P}$ has arbitrary meets and, for any coalgebra $\gamma_X:X\rightarrow \mathscr{F}(X)$, $\mathscr{P}_{\gamma_X}$ and $\Box_X$ commute with them;
			\item in the hypothesis above if $\mathscr{P}_f$ commutes with arbitrary meets for all arrows $f$ then  $\{\scat{X}\}_{\gamma_X \in \catname{Ob}(\coalg{F})}$ defines a closure operators $\scato$ on $\mathscr{P}^c$.
			
		\end{enumerate}
	\end{theoremEnd}
\end{lemma}
\begin{proof}
	\begin{proofEnd}
		\begin{enumerate}
			\item Clearly $\alpha\leq 
			\diam{X}(\alpha)$; if $\alpha \leq \beta$ we have that
			\begin{equation*}
			\mathscr{P}_{\gamma_{X}}(\Box_X(\alpha))\leq \mathscr{P}_{\gamma_{X}}(\Box_X(\beta)) 
			\end{equation*}  
			from which monotonicity follows; for $f$ an arrow between $\gamma_X:X\rightarrow \mathscr{F}(X)$ and $\gamma_Y:Y\rightarrow \mathscr{F}(Y)$, we have a commutative diagram
			\begin{center}
				\begin{tikzpicture}
				\node(A)at (0,0) {$X$};
				\node(B)at (3,0) {$Y$};
				\node(C)at (0,-1.5) {$\mathscr{F}(X)$};
				\node(D)at (3,-1.5) {$\mathscr{F}(Y)$};
				\draw[->](A)--(C) node[pos=0.5, left]{$\gamma_X$};
				\draw[->](B)--(D) node[pos=0.5, right]{$\gamma_Y$};
				\draw[->](A)--(B) node[pos=0.5, above]{$f$};
				\draw[->](C)--(D) node[pos=0.5, below]{$\mathscr{F}(f)$};
				\end{tikzpicture}
			\end{center} and computing we get the thesis:
			\begin{align*}
			\diam{X}(\mathscr{P}_f(\alpha))&=\mathscr{P}_f(\alpha)\vee \mathscr{P}_{\gamma_X}(\Box_X(\mathscr{P}_f(\alpha)))\\&=\mathscr{P}_f(\alpha)\vee \mathscr{P}_{\gamma_X}(\mathscr{P}_{\mathscr{F}(f)}(\Box_Y(\alpha)))\\&=\mathscr{P}_f(\alpha)\vee \mathscr{P}_f(\mathscr{P}_{\gamma_Y}(\Box_Y(\alpha)))\\&=
			\mathscr{P}_f(\alpha \vee \mathscr{P}_{\gamma_Y}(\Box_Y(\alpha)))\\&=\mathscr{P}_f(\diam{Y}(\alpha))
			\end{align*} 
			
			
			\item By hypothesis:
			\begin{align*}
			\alpha &\leq \bigwedge_{\beta \in \mathcal{N}_{\gamma_X}(\alpha)}\mathscr{P}_{\gamma_X}(\Box_X(\beta))\\&=
			\mathscr{P}_{\gamma_X}(\bigwedge_{\beta \in \mathcal{N}_{\gamma_X}(\alpha)}\Box_X(\beta))\\&=			\mathscr{P}_{\gamma_X}(\Box_X(\bigwedge_{\beta \in \mathcal{N}_{\gamma_X}(\alpha)}\beta))\\&=\mathscr{P}_{\gamma_X}(\Box_X(\mathsf{s}_{\gamma_X}(\alpha)))
			\end{align*}
			
			\item The inequality $\alpha\leq 
			\scat{X}(\alpha)$ follows at once, if $\alpha \leq \beta$ we have $\mathscr{P}_{\gamma_{X}}(\Box_X(\alpha))$ as in the first point but this implies that 
			$\mathcal{N}_{\gamma_X}(\beta)\subset \mathcal{N}_{\gamma_X}(\alpha)$. 
			Hence, $\bigwedge_{\theta \in \mathcal{N}_{\gamma_X}(\alpha)}\theta\leq \bigwedge_{\theta \in \mathcal{N}_{\gamma_X}(\beta)}\theta$, 
			from which we deduce the monotonicity of $\scat{X}$. Let now $f:X\rightarrow Y$ be an arrow such that		\begin{center}
				\begin{tikzpicture}
				\node(A)at (0,0) {$X$};
				\node(B)at (3,0) {$Y$};
				\node(C)at (0,-1.5) {$\mathscr{F}(X)$};
				\node(D)at (3,-1.5) {$\mathscr{F}(Y)$};
				\draw[->](A)--(C) node[pos=0.5, left]{$\gamma_X$};
				\draw[->](B)--(D) node[pos=0.5, right]{$\gamma_Y$};
				\draw[->](A)--(B) node[pos=0.5, above]{$f$};
				\draw[->](C)--(D) node[pos=0.5, below]{$\mathscr{F}(f)$};
				\end{tikzpicture}
			\end{center}
			commutes, and notice that for all $\theta\in \mathcal{N}_Y(\alpha)$ then
			\begin{align*}
			\mathscr{P}_f(\alpha)&\leq \mathscr{P}_f(\mathscr{P}_{\gamma_Y}(\Box_Y(\theta))) 
			=\mathscr{P}_{\gamma_X}(\mathscr{P}_{\mathscr{F}(f)}(\Box_Y(\theta))) =\mathscr{P}_{\gamma_X}(\Box_X(\mathscr{P}_f(\theta)))
			\end{align*}
			hence $\mathscr{P}_f(\theta)\in \mathcal{N}_X(\mathscr{P}_f(\alpha))$ and thus
			\begin{align*}
			\scat{X}(\mathscr{P}_f(\alpha))& = \mathscr{P}_f(\alpha)\vee \mathsf{s}_{\gamma_X}(\mathscr{P}_f(\alpha))=\mathscr{P}_f(\alpha)\vee \bigwedge_{\beta \in \mathcal{N}_X(\mathscr{P}_f(\alpha))}\beta\\&\leq \mathscr{P}_f(\alpha)\vee \bigwedge_{\beta \in \mathcal{N}_X(\mathscr{P}_f(\alpha))}\beta
			\\&\leq \mathscr{P}_f(\alpha)\vee\bigwedge_{\beta \in \mathcal{N}_Y(\alpha)}\mathscr{P}_f(\beta)\\&\leq \mathscr{P}_f(\alpha)\vee \mathscr{P}_f(\bigwedge_{\beta \in \mathcal{N}_Y(\alpha)}\beta)\\&=\mathscr{P}_f(\alpha\vee \mathsf{s}_{\gamma_Y}(\alpha)) =\mathscr{P}_f(\scat{Y}(\alpha))
			\end{align*}
			and we are done.
			\qedhere
		\end{enumerate}
	\end{proofEnd} \qedhere
\end{proof}

The previous proposition provides us with many examples with practical applications.
\begin{example}[Kripke frames]
Let $\mathcal{P}:\catname{Set}\rightarrow \catname{Set}$ be the covariant powerset functor, and 
$\mathscr{P}:\catname{Set}^{op}\rightarrow \catname{InfSL}$ be the controvariant one, seen as primary doctrine.
We can define a predicate lifting $\Box$ taking as components:
		\begin{align*}
		\Box_X: \mathscr{P}(X) & \rightarrow \mathscr{P}(\mathcal{P}(X))\\
		A & \mapsto \small{\downarrow} A
		\end{align*} 
		where $\small{\downarrow} A$ denotes the set of downward-closed subsets of $A$.
		In this case for any coalgebra $\gamma_X:X\rightarrow \mathcal{P}(X)$ we have
		\begin{align*}
		x\in \gamma_X^{-1}(\Box_X(A)) & \iff \gamma_X(x)\subset A\\
		B\in \mathcal{N}_{\gamma_X}(A) & \iff \gamma_X(a)\subset B \text{ for any } a\in A 
		\end{align*}
		so 
		$\mathsf{s}_{\gamma_X}(A)=\bigcup_{a\in A} \gamma_X(a)$ and $\scat{X}(A)=A\cup 	\bigcup_{a\in A}\gamma_X(a)$.
		
		By this description it is clear that $\scato$ is grounded and fully additive.  
		$\diamo$ is grounded too but it is not even finitely additive: take $4:=\{0,1,2,3\}$ with stuctural map $\gamma_4$ given by 
		\begin{gather*}
		0\mapsto \{3\} \quad
		1\mapsto \{2,3\} \quad
		2\mapsto \{2\} \quad
		3\mapsto \{3\}
		\end{gather*}
		Now take $A:=\{2,3\}$, it is immediate to see that $\diam{4}(A)=4$, on the other hand $\diam{4}(\{2\})=\{2\}$ and $\diam{4}(\{3\})=\{0,3\}$.

		\begin{remark}
			In this case the meaning of (and the notation for) $\diamo$ and $\scato$ becomes clearer: if we think to the value of $\gamma_X(x)$ as the family of points accessible from $x\in X$ then $\diam{X}$ add to a subset $A$ the set of its \emph{predecessors}, i.e. points from which some  $a\in A$ is accessible, while $\scat{X}$ add the set of point \emph{successors}, i.e. points which are accessible from some point of $A$.
		\end{remark} 
\end{example}

\begin{example}[Probabilistic frames \cite{giry1982categorical,avery2016codensity,DBLP:journals/jcss/BacciM15}]
Let $\catname{Meas}$ be the category of measurable space and measurable functions; then we can take as primary doctrine $\mathscr{P}$ the functor 
		\begin{equation*}
		\functor[l]{(X,\Omega_X)}{f}{(Y,\Omega_Y)}
		\functormapsto
		\functor[r]{\Omega_X}{f^{-1}}{\Omega_Y}	
		\end{equation*}
		As endofunctor we can take the \emph{Giry monad} $\mathscr{G}:\catname{Meas}\to\catname{Meas}$:
		\begin{itemize}
			\item given an object $(X,\Omega_X)$, $\mathscr{G}(X,\Omega_X)$ is the set
			\begin{equation*}
			\{\mu:\Omega_X\rightarrow [0,1]\mid \mu \text{ is a probabilty measure on } \Omega_X \}
			\end{equation*}
			equipped with the smallest $\sigma$-algebra for which all the \emph{evaluation functions}
			\begin{align*}
			\mathsf{ev}_A:\mathscr{G}(X,\Omega_X)&\rightarrow [0,1]\\
			\mu &\mapsto \mu(A)
			\end{align*} 
			with $A\in \Omega_X$, are Borel-measureable.
			
			\item for a measurable $f:(X,\Omega_X)\rightarrow (Y,\Omega_Y)$,
			\begin{align*}
			\mathscr{G}(f):\mathscr{G}(X,\Omega_X)&\rightarrow \mathscr{G}(Y,\Omega_Y)\\
			\mu& \mapsto \mu \circ f^{-1}
			\end{align*}
		\end{itemize}
		(For the measurability of $\mathscr{G}(f)$ notice that given a Borel subset $L$ of $[0,1]$ and $A\in \Omega_Y$ we have that 
		$\mu \in \mathscr{G}(f)^{-1}(\mathsf{ev}_A(L))\iff \mu \in \mathsf{ev}_{f^{-1}(A)}(L)$)
		
		For a coalgebra $\gamma_{(X,\Omega_X)}$ and $p\in [0,1]$ we define
		\begin{align*}
		\Box_{(X,\Omega_X), p}:\Omega_X&\rightarrow \mathcal{P}(\mathscr{G}(X))\\
		A & \mapsto \{\mu \in \mathscr{G}(X,\Omega_X) \mid\mu(A)\geq p \}
		\end{align*}
		notice that the set on the right is $\mathsf{ev}_A^{-1}([p,1])$ and so $\Box_{(X,\Omega_X), p}$ is well defined. In this situation we have
		\begin{equation*}
		\diam{(X,\Omega_X)}(A):=A\cup \{x\in X\mid p\leq \gamma_{(X,\Omega_X)}(x)(A)\}
		\end{equation*}
\end{example}
\begin{remark}
	If we think of  a coalgebra $\gamma_{(X,\Omega_X)}$ as describing how likely is a transition from a state to the various $A\in \Omega_X$ then, given a $p\in [0,1]$, $\diam{(X,\Omega_X)}(A)$ is the set of points which access  $A$  with probability at least  $p$.
\end{remark}





%\marginpar{ci vorrebbero due parole di spiegazione sul significato di questi operatori di chiusura, e come si possono usare}


	\section{Logics for Closure Operators} \label{sec:slcs}
	In this section, we provide a sound and complete logic for closure hyperdoctrines. 
	This logic is a (first order) version of Spatial Logic for Closure Spaces (SLCS) \cite{ciancia2016spatial}, although with a slightly different presentation.
	
	\subsection{Syntax and derivation rules}
	
	We briefly recall the categorical presentation of signatures, as in \cite{jacobs1999categorical}.
	\begin{definition}\label{def:signature}
		A \emph{signature} $\Sigma$ is a triple $(\abs{\Sigma}, \Gamma, \Pi )$ where
		\begin{itemize}
			\item $\abs{\Sigma}$ is a set, called the set of \emph{basic types};
			\item $\Gamma$ is a functor $\abs{\Sigma}^\star \times \abs{\Sigma}\rightarrow \catname{Sets}$. We will call \emph{function symbol} an element $f$ of $\Gamma((\sigma_1,\dots,\sigma_n), \sigma_{n+1})$ and we will write $f:\sigma,\dots,\sigma_n\rightarrow, \sigma_{n+1}$;
			\item $\Pi$ is a functor $\abs{\Sigma}^\star\rightarrow \catname{Set}$,  we will call \emph{predicate symbol} an element $P$ of $\Pi(\sigma_1,\dots,\sigma_n)$ and we will write $P:\sigma_1,\dots,\sigma_n$.
		\end{itemize}
		A morphism of signature $\phi:\Sigma_1\rightarrow \Sigma_2$ is a triple $(\phi_1,\phi_2,\phi_3)$ such that
		\begin{itemize}
			\item $\phi_1$ is a function $\abs{\Sigma_1}\rightarrow \abs{\Sigma_2}$;
			\item $\phi_2$ is a natural transformation $\Gamma_1\rightarrow \Gamma_2\circ (\phi_1^{\star}\times \phi_1)$;
			\item $\phi_3$ is a natural transformation $\Pi_1\rightarrow \Pi_2\circ \phi_1^{\star}$.
		\end{itemize}
		For any $\sigma\in \abs{\Sigma}$ we fix an countably infinite set $X_\sigma$ of \emph{variables}; definition of terms is straightforward (\cite{jacobs1999categorical}).
	\end{definition}
	
\iffalse
	\begin{proposition}\label{pb}
		Signature and their morphisms with componentwise composition form a category $\catname{SignPred}$, moreover this category is the pullback of the codomain fibration $\cod:\catname{Set}^{\catname{2}}\rightarrow \catname{Set}$
		along the functor $\catname{Set}\rightarrow\catname{Set}$
		\begin{equation*}
		\functor[l]{A}{f}{B}
		\functormapsto
		\functor[r]{(A^\star\times A)\sqcup A^\star}{(f^\star \times f)\sqcup f^\star}{(B^\star\times B)\sqcup B^\star}
		\end{equation*}
	\end{proposition}
	\begin{proof}
		This is straightforward (\cite{jacobs1999categorical} definition $4.1.1$). \qedhere
	\end{proof}
\fi	
	
	\begin{definition}
		Given a signature $\Sigma$, its \emph{classifying category} is the category $\class{\Sigma}$ in which
		\begin{itemize}
			\item objects are contexts;
			\item Given $\Gamma:=[x_i:\sigma_i]_{i=1}^n$ and $\Delta=[y_i:\tau_i]_{i=1}^m$ an arrow $\Gamma\rightarrow \Delta$ is a $m$-uple of terms $(T_1,...,T_m)$ such that $\Gamma \vdash T_i:\tau_i$ for any $i$;
			\item composition is given by substitution.
		\end{itemize}
	\end{definition}
\begin{proposition}
	\begin{theoremEnd}{prop}
		$\class{\Sigma}$ is a category with finite products for any signature $\Sigma$.
	\end{theoremEnd}
\end{proposition}
\begin{proof}
	\begin{proofEnd}Associativity of composition and the fact that $(x_1,...,x_n)$ is the identity for $[x_i:\sigma_i]_{i=1}^n$ follows from a straightforward computation. The empty context is clearly terminal while, given two contexts $\Gamma:=[x_i:1\sigma_i]_{i=1}^n$ and $\Delta=[y_i:\tau_i]_{i=1}^m$ we can take their concatenation as a product $\Gamma \times \Delta$, the universal property follows immediately.\qedhere 
	\end{proofEnd} \qedhere
\end{proof}
% le vere definizioni iniziano qui

Now we can introduce the rules for context and closure operators of the Spatial Logic for Closure Spaces, over any given signature. 

As usual, we denote by $\Gamma \vdash t : \tau$ the judgment ``$t$ has type $\tau$ in context $\Gamma$'', and by $\Gamma \vdash \phi : \propo$ the judgment ``$\phi$ is a well-formed formula in context $\Gamma$''.
\begin{definition}\label{def:slcssynt}
	The rules for contexts and well-formed formulae for the closure operators for a signature $\Sigma$ are the usual ones for a first order signature  (see \cite{jacobs1999categorical}) plus:
	\begin{equation*}
	\inferrule*[right=$\mathcal{C}$-F]{\Gamma \vdash \phi:\propo}{\Gamma \vdash \mathcal{C}(\phi):\propo}\qquad 
	\inferrule*[right=$\mathcal{U}$-F]{\Gamma \vdash \phi:\propo \\ \Gamma\vdash \psi:\propo}{\Gamma  \vdash \phi \mathcal{U}\psi:\propo}
	\end{equation*}
	For any context $\Gamma$ we define $\formu{\Sigma}(\Gamma)$ to be the set of formulae $\phi$ such that $\Gamma \vdash \phi:\propo$. 
\end{definition}


Then, we can introduce the rules for the logical judgments of the form $\Gamma \mid \Phi \vdash \phi$, where $\Phi$ is a finite set of propositions well-formed in $\Gamma$.
	\begin{definition}\label{def:slcsrules}
		We define four rules for the well-formed formulae previously defined:
		\begin{itemize}
			\item $\mathcal{C}$'s rules:
			\begin{equation*}	
			\inferrule*[right=Cl-$1$]{\Gamma \mid \Phi\vdash \psi}{\Gamma \mid \Phi \vdash \mathcal{C}(\psi) }\quad
			\inferrule*[right=Cl-$2$]{\Gamma \mid \Phi, \psi \vdash \phi }{\Gamma  \mid \Phi, \mathcal{C}(\psi ) \vdash \mathcal{C}(\phi)}
			\end{equation*}
			\iffalse 
			\item $\mathcal{C}$'s rules:
			\begin{equation*}	
			\begin{gathered}
			\inferrule*[right=Cl-$1$]{\hspace{1pt}\hspace{1pt}}{\Gamma \mid \Phi, \mathcal{C}(\bot)\vdash \bot}\\
			\inferrule*[right=Cl-$3$]{\hspace{1pt}\hspace{1pt} }{\Gamma  \mid \Phi, \mathcal{C}(\phi\vee \psi ) \vdash \mathcal{C}(\phi)\vee \mathcal{C}(\psi)}
			\end{gathered}\quad
			\begin{gathered}
			\inferrule*[right=Cl-$2$]{\hspace{1pt}\hspace{1pt} }{\Gamma  \mid \Phi, \phi \vdash\mathcal{C}(\phi) }	\\
			\inferrule*[right=Cl-$4$]{\hspace{1pt}\hspace{1pt}}{\Gamma \mid  \Phi, \mathcal{C}(\phi)\vee \mathcal{C}(\psi) \vdash\mathcal{C}(\phi\vee \psi ) } 
			\end{gathered}
			\end{equation*}
			\fi 
			\item $\mathcal{U}$'s rules \vspace{-2ex}
			\begin{gather*}
			\inferrule*[right=$\mathcal{U}$-I]{\Gamma \mid \Phi, \varphi \vdash \phi\\\Gamma \mid \Phi, \mathcal{C}(\varphi), \neg \phi \vdash \psi}{\Gamma \mid \Phi, \varphi \vdash \phi\mathcal{U}\psi}
			\\	
%			\inferrule*[right=$\mathcal{U}$-E
%			]{\{\Gamma \mid \Phi, \varphi \vdash \theta  \mid \varphi \in \mathsf{u}_{(\Gamma, \Phi)}(\phi, \psi)\}}{\Gamma \mid \Phi, \phi\mathcal{U}\psi  \vdash \theta }
%			\\	
			\inferrule*[right=$\mathcal{U}$-E]
			{\text{for all } \phi \text{ such that }  \Gamma \vdash \varphi:\propo:\quad 
				\Gamma \mid \Phi, \varphi \Rightarrow \phi, (\mathcal{C}(\varphi)\wedge \neg\varphi) \Rightarrow \psi,\varphi \vdash \theta}{\Gamma \mid \Phi, \phi\mathcal{U}\psi  \vdash \theta }
			\end{gather*}
%			where 
%			\begin{equation*}
%			\mathsf{u}_{(\Gamma, \Phi)}(\phi, \psi):=\{\varphi \ \mathrm{such \ that}\ \Gamma \vdash \varphi:\propo,  \Gamma \mid \Phi, \varphi \vdash \phi, \Gamma \mid \Phi, \mathcal{C}(\varphi), \neg \varphi \vdash \psi\}
%			\end{equation*}	 	
		\end{itemize}
		
		The \emph{Propositional Logic for Closure Operators on $\Sigma$} (PLCO) is given by the usual propositional rules (i.e., without the quantifiers) for the typed (intuitionistic) sequent calculus (see e.g.~\cite{jacobs1999categorical}), extended with the four rules above.
		
		 The \emph{Regular Logic for Closure Operators on $\Sigma$} (RLCO) is given by the four rules above, plus the rules for conjunction, $\top$ and the existential quantifier only.
		 
		Finally, the \emph{First Order Logic for Closure Operators on $\Sigma$} (FOLCO) is given by the four rules above added to the usual rules for first order logic. Similarily with equality.	
		 
		 Derivability of sequents is defined in the usual way (\cite{pitts1995categorical}).
		\end{definition}
\begin{remark}
		PLCO corresponds to the Spatial Logic for Closure Spaces considered in 	\cite{ciancia2014specifying}.
\end{remark}
	\begin{remark}
		Notice that $\mathcal{U}$-E is an \emph{infinitary} rule saying that a formula $\theta$ can be derived from $\phi\mathcal{U}\psi$ if it can be derived from \emph{all} the formulae $\varphi$ satisfying precise conditions.
		Thus, this rule shows the second-order nature of the $\mathcal{U}$ operator.
%	$\mathsf{u}_{(\Gamma, \Phi)}(\phi, \psi)$ 
	\end{remark}
	
	
	
	\subsection{Categorical semantics of closure logics}
	In this section we provide a sound and complete categorical semantics of the logics for the closure operators defined above.
	 
	\begin{definition}
   Two formulae $\phi,\psi \in \formu{\Sigma}(\Gamma)$ are \emph{provably equivalent} if $\Gamma \mid \psi \vdash \phi$ and $\Gamma \mid \phi \vdash \psi$.
   We will denote the quotient of  $\formu{\Sigma}(\Gamma)$ by this relation with $\lind{\Sigma}(\Gamma)$, $[\phi]$ will denote the class of $\phi$ in it.
	\end{definition}

	\begin{theoremEnd}{prop}
		For any signature $\Sigma$ the following are true:
		\begin{enumerate}
			\item $\lind{\Sigma}(\Gamma)$ equipped with the order $[\phi]\leq [\psi]$ if and only if $\Gamma \mid \phi\vdash \psi$ is derivable is:
			\begin{itemize}
				\item a meet semilattice in the case we are considering regular logic;
				\item  a Heyting algebra if we are considering propositional or first order logic;
			\end{itemize} 
			\item $[\phi \mathcal{U}\psi]$ is
			the supremum of the set 
			\begin{equation*}
			\mathsf{u}_{\Gamma}(\phi, \psi):=\{[\varphi]\in \lind{\Sigma}(\Gamma) \ \mathrm{such \ that}\  \Gamma \mid \varphi \vdash \phi, \Gamma \mid \mathcal{C}(\varphi), \neg \varphi \vdash \psi\}
			\end{equation*}
			\item 
			there exists a (elementary) closure or existential doctrine or a (elementary) hyperdoctrine  $(\lind{\Sigma}, \mathfrak{c}_\Sigma)$ on $\class{\Sigma}$ sending $\Gamma$ to $\lind{\Sigma}(\Gamma)$.
		\end{enumerate}
	\end{theoremEnd}
	\begin{proof}
		\begin{enumerate}
			\item 	The logical connectives induce a Heyting algebra or a meet semilattice structure on $\lind{\Sigma}(\Gamma)$ which has precisely $\leq$ as associated order.
			\item From $\mathcal{U}$-I follows that $[\phi \mathcal{U}\psi]$ is an upper bound for $	\mathsf{u}_{\Gamma}$ while $\mathcal{U}$-E	implies that $[\phi \mathcal{U}\psi]$ is the least of them.
			\item For any morphism $(T_1,...,T_n):\Gamma \rightarrow \Delta$ substitution of terms gives us a morphism of Heyting algebras/meet semilattices $\lind{\Sigma}(\Delta)\rightarrow \lind{\Sigma}(\Gamma)$; quantifiers gives us the existential doctrine/hyperdoctrine structure (cfr. \cite{pitts1995categorical} for the details). In any case have to define a preclosure operator $\mathfrak{c}_{\Sigma, \Gamma}$ on each $\lind{\Sigma}(\Gamma)$ but this is easily done defining
			\begin{align*}
			\mathfrak{c}_{\Sigma, \Gamma}: \lind{\Sigma}(\Gamma)&\rightarrow \lind{\Sigma}(\Gamma)\\
			[\phi] & \mapsto [\mathcal{C}(\phi)]
			\end{align*}
			The $\mathcal{C}$'s rules assure us that $\mathfrak{c}_\Sigma$ is well defined, inflationary and monotone, while an easy induction shows that 
			\begin{align*}
			\lind{\Sigma}_{(T_1,...,T_n)}([\mathcal{C}(\phi)])&=	\mathfrak{c}_{\Sigma, \Gamma}(\lind{\Sigma}_{(T_1,...,T_n)}(\phi))
			\end{align*}
			for any $(T_1,...,T_n):\Gamma\rightarrow \Delta$.
			We can add fibered equalities, given $\Gamma:=[x_i:\sigma_i]$ putting:
			\begin{equation*}
			\delta_{\Gamma \times \Gamma}:=\bigwedge_{i=1}^n[ x_i=_{\sigma_i}y_i]
			\end{equation*}
			where $\{y_i\}_{i=1}^n$ is a set of fresh variables such that $y_i:\sigma_i$ for any $i$.
			\qedhere
		\end{enumerate} 
	\end{proof}
	
	Let us prove the soundness and completeness of the categorical semantics wrt.~the various logical fragments.
	\begin{definition}\label{def:slcsmodel}
		Let $(\mathscr{P}, \mathfrak{c}):\catname{C}^{op}\rightarrow \catname{InfSL}$ be an (elementary) closure doctrine (existential doctrine/hyperdoctrine) then a morphism of $\catname{cPD}$ ($\catname{cED}$, $\catname{cEED}$, $\catname{cEHD}$, $\catname{cHD}$) $(\mathscr{M}, \mu): (\lind{\Sigma}, \mathcal{C})\rightarrow (\mathscr{P}\mathfrak{c})$ is a \emph{model of the propositional (regular, first-order) logic (with equality) of closure operators in $(\mathscr{P}, \mathfrak{c})$} if it is open. 
		
		\iffalse and
		\begin{equation*}
		\mu_\Gamma([\phi \mathcal{U}\psi])=\bigvee_{[\varphi]\in [\mathsf{u}_\Gamma](\phi, \psi)}\mu_\Gamma([\varphi])
		\end{equation*}
		for any two formulae $\phi$ and $\psi$ such that $[\phi]$ and $[\psi]\in \lind{\Sigma}(\Gamma)$.\fi  
		
		A sequent $\Gamma \mid \Phi\vdash \psi$ is \emph{satisfied by $(\mathscr{M}, \mu)$} if 
		$\bigwedge_{\phi \in \Phi}\mu_{\Gamma}(\phi) \leq \mu_{\Gamma}(\psi)$.
		%We will use $\modd(\mathscr{P}, \mathfrak{c})$ to denote the full subcategory of $\catname{cD}((\lind{\Sigma}, \mathfrak{c}_\Sigma), (\mathscr{P}, \mathfrak{c}))$ (or ) given by models in $(\mathscr{P}, \mathfrak{c})$.
	\end{definition}

	\begin{theorem}\label{th:completeness}
		A sequent $\Gamma \mid \Phi\vdash \psi$ is satisfied by the \emph{generic model} $(1_{\class{\Sigma}}, 1_{\lind{\Sigma}})$ 
		if and only if it is derivable.
	\end{theorem}
	\begin{proof}
		By definition, $\Gamma \mid \Phi\vdash \psi$ is satisfied if and only if 
		\begin{equation*}
		\bigwedge_{\phi \in \Phi}[\phi]\leq [\psi]
		\end{equation*}
		in $\lind{\Sigma}(\Gamma)$ but this is equivalent to the derivability of
		\begin{equation*}
		\Gamma \mid \bigwedge_{\phi \in \Phi}\phi \vdash \psi
		\end{equation*}
		whose derivability is equivalent (applying the conjunction rules a finite number of times) to
%		\begin{equation*}
$		\Gamma \mid \Phi \vdash \psi $,
%		\end{equation*} 
		and we are done.\qedhere
	\end{proof}
	
	\begin{corollary}
		The above defined categorical semantics for PLCO/RLCO/FOLCO (with or without equality) is sound and complete.
	\end{corollary}
	\begin{proof}
		The only thing left to show is soundness for an arbitrary $(\mathscr{P}, \mathfrak{c})$ but this follows at once since each component $\mu_\Gamma$ of $\mu$ is monotone. %and preserves the supremum of $[\mathsf{u}_\Gamma](\phi, \psi)$.
		\qedhere
	\end{proof}
	
	\subsection{About the semantics of $\mathcal{U}$}
	As we have remarked before, the rule $\mathcal{U}$-E for the operator $\mathcal{U}$ is infinitary.
	Although in general this is needed, in this section we will define a class of hyperdoctrines in which the semantics of $\mathcal{U}$ can be given as a supremum of approximants. %, and thus with a more compact formalization.
	\begin{definition}\label{def:extbound}
		Let $(\mathscr{P}, \mathfrak{c}):\catname{C}^{op}\rightarrow \catname{InfSL}$ be a closure doctrine that factors through the category of Heyting algebras. 
		For any object $C$ define the \emph{external boundary}:
		\begin{align*}
		\partial^+_C:\mathscr{P}(C)&\rightarrow \mathscr{P}(C)\\
		\alpha &\mapsto \mathfrak{c}_C(\alpha)\wedge \neg \alpha 
		\end{align*}
		For $\phi$ and $\psi\in \mathscr{P}(C)$, we define $\phi\mathfrak{U}_C\psi\in \mathscr{P}(C)$ as the supremum, if it exists, of the set
		\begin{equation*}
		\mathfrak{u}_C(\phi, \psi):=	\{\varphi\in \mathscr{P}(C) \mid \varphi\leq \phi \text{ and } \partial^+_C(\varphi) \leq\psi \}
		\end{equation*} 
	\end{definition}
	
	\begin{remark}
		If $\mathscr{P}$ is $\lind{\Sigma}$ then
		$[\phi]\mathfrak{U}_\Gamma[\psi]=[\phi\mathcal{U}\psi]$
		for any $[\phi]$ and $[\psi]\in \lind{\Sigma}(\Gamma)$.
	\end{remark}
	\begin{remark}\label{inclu}
		If $(\mathscr{M}, \mu)$ is a model then 
		$\mu_\Gamma(\mathsf{u}_{\Gamma}(\phi, \psi))\subset  \mathfrak{u}_{\mathscr{M}(\Gamma)}(\mu_\Gamma([\phi]), \mu_\Gamma([\psi]) )$
		for any $\Gamma$.
	\end{remark}

	\begin{example}\label{exa}
		Let $(X,\mathfrak{c})$ be a pretopological space and $S$, $T\in \mathcal{P}^p(X,\mathfrak{c})$, then
		\begin{equation*}
		S\mathfrak{U}_{(X, \mathfrak{c})}T = \bigcup \{W\subset S\mid \partial^+_{(X,\mathfrak{c})}(W) \subset T\}
		\end{equation*}
		i.e. $x\in S\mathfrak{U}_{(X, \mathfrak{c})}T$ if and only if there exists $W\subset S$ such that $X\in W$ and $\partial^+_{(X,\mathfrak{c})}(W) \subset T$.
	\end{example}
	\begin{example}
		Let us consider the closure operator $\mathfrak{c}_\epsilon$ on $\catname{Set}(-, [0,1])$ (see \cref{sec:reprex}).
		For any $f:X\rightarrow [0,1]$, it is $(\neg f)(x)= 1$ if and only if $f(x)=0$. 
		So,
		\[(\mathfrak{c}_{X, \epsilon}(f)\wedge \neg f)(x)=\begin{cases}
		\epsilon  & f(x)=0 \\
		0 & f(x)\neq 0
		\end{cases},
		\] 
hence, given $g, h:X\rightarrow [0,1]$, $f\in \mathsf{u}_\Gamma(g,h)$ if and only if $f\leq g$ and $h(x)\geq \epsilon$ for any $x\in f^{-1}(0)$.
	\end{example}


\begin{remark}
	If $(\mathscr{M}, \mu)$ is a model then for any $[\varphi]\in \lind{\Sigma}(\Gamma)$ such that $\varphi \in \mathsf{u}_\Gamma(\phi, \psi)$ we have 
	$\mathscr{\mu}_\Gamma([\varphi])\leq \mu_{\Gamma}([\phi \mathcal{U}\psi])$.
\end{remark}

	\begin{definition}
		Let $(\mathscr{P}, \mathfrak{c})$ be as in \cref{def:extbound}. 
		A model $(\mathscr{M}, \mu):\lind{\Sigma}\rightarrow (\mathscr{P},\mathfrak{c})$ is said \emph{continuous} if the equality
		\begin{equation*}
		\mu_{\Gamma}([\phi \mathcal{U}\psi])=\mu_{\Gamma}([\phi])\mathfrak{U}_{\mathscr{M}(\Gamma)}\mu_{\Gamma}([\psi])
		\end{equation*}
		holds for any context $\Gamma$ and $[\phi], [\psi]\in \lind{\Sigma}(\Gamma)$.
	\end{definition}


	\begin{proposition}
		Let $\Sigma$ be a signature and $(\mathscr{P}, \mathfrak{c})$ a complete (elementary, existential, or hyper)doctrine, i.e. $\mathscr{P}(C)$ is complete for any object $C$ of $\catname{C}$; then, for any product preserving functor: $\mathscr{M}:\class{\Sigma} \rightarrow \catname{C}$ and functions 
		\begin{equation*}
		\mu^*_{\Gamma}:\Pi(\sigma_1,...,\sigma_n)\rightarrow \mathscr{P}(\mathscr{M}(\Gamma))
		\end{equation*}
		for all $\Gamma=[x_i:\sigma_i]_{i=1}^n$, there exists a unique continuous model $(\mathscr{M}, \mu)$ in $(\mathscr{P}, \mathfrak{c})$ such that
		\begin{equation*}
		\mu_\Gamma([P(x_1,...,x_n)])=\mu^*_\Gamma(P)
		\end{equation*}
	\end{proposition}
	\begin{proof}
		This follows immediately by induction. \qedhere
	\end{proof}
	
	
	
	
	\begin{example}
		Let $\mathcal{X}=\{(X_i, \mathfrak{c}_{i})\}_{i\in I}$ be a small family of pretopological spaces and let us define $\Sigma$ as follows:
		\begin{gather*}
		\abs{\Sigma}:=\mathcal{X}\qquad  \Gamma(((X_{i_1}, c_{i_1}),...,(X_{i_n},c_{i_n})), (X_j, c_{j})):=\catname{PrTop}(\prod_{k=1}^{n}(X_{i_k},c_{i_k}), (X_j, c_j)) \\ \Pi((X_{i_1}, c_{i_1}),...,(X_{i_n},c_{i_n})):=\mathcal{P}(\prod_{k=1}^{n}X_{i_k}) 
		\end{gather*}
		We can take as $\mathscr{M}$ the unique product preserving functor  
		$\class{\Sigma}\rightarrow \catname{PrTop}$ such that
		\begin{gather*}
		\functor[l]{(X_i,c_i)}{f}{(X_i,c_i)}
		\functormapsto
		\functor[r]{(X_i,c_i)}{f}{(X_i,c_i)}
		\end{gather*}
		i.e., $\mathscr{M}$ sends contexts to products and list of terms to the corresponding product arrow. We can define $\mu^*$ sending each predicate $P:(X_{i_1},\mathfrak{c}_{i_1}),...,(X_{i_n},\mathfrak{c}_{i_n})$ to corresponding subset of $\prod_{k=1}^{n}$ $(X_{i_k},\mathfrak{c}_{i_k})$.
		\cref{exa} guarantees that this semantics is the same as the one developed in \cite{ciancia2014specifying}.
	\end{example}
	
	\begin{proposition}
		For any signature $\Sigma$ a sequent is derivable if and only if it is satisfied by any continuous model.
	\end{proposition}
	\begin{proof}
		This follows immediately by the fact that the generic model is continuous. \qedhere
	\end{proof}
	
\ifreport

\section{Paths in closure doctrines}\label{sec:paths}
Often, in spatial logics we are interested also on \emph{reachability} of some property.
Differently from closure and the ``until'' operator, reachability is not a structural property of the logical domain; rather, it depends on the kind of paths we choose to explore the space.
In this section we formalise this idea, and show how to interpret also the $\mathcal{S}$ operator from SLCS.

\subsection{The reachability closure operator}
\begin{definition}
	Let $\mathscr{P}:\catname{C}^{op}\rightarrow \catname{HA}$ be an hyperdoctrine, an \emph{internal preorder} in $\mathscr{P}$ is a pair $(I, \rho)$ where $I$ is an object of $\catname{C}$ and $\rho\in \mathscr{P}(I\times I)$ such that is reflexive 
	($\delta_I \leq \rho$) and transitive ($\mathscr{P}_{(\pi_1,\pi_2)}(\rho)\wedge\mathscr{P}_{(\pi_2,\pi_3)}(\rho)\leq \mathscr{P}_{(\pi_1,\pi_3)}(\rho)$).
	
	$(I,\rho)$ is called an \emph{internal order} if in addition $\rho $ is \emph{antisymmetric}, i.e.
	$\rho\wedge\mathscr{P}_{(\pi_2,\pi_1)}(\rho) \leq \delta_I$.
	Moreover $(I, \rho)$ is \emph{total} if $\rho \vee \mathscr{P}_{(\pi_2,\pi_1)} (\rho) = \top$.

	A \emph{internal monotone arrow} $f:(I,\rho)\rightarrow (J, \sigma)$ is an arrow of $\catname{C}$ such that $\rho \leq \mathscr{P}_{f\times f}(\sigma)$.
\end{definition}

\begin{definition}
	Let $(\mathscr{P}, \mathfrak{c}):\catname{C}^{op}\rightarrow \catname{HA}$ be an elementary existential closure doctrine, we say that $\phi\in \mathscr{P}(C)$ is \emph{connected} if $\varphi \vee \psi = \phi$ and  $\mathfrak{c}(\varphi)\wedge \psi = \bot $ imply $\varphi=\bot$.
	
	An object $C$ is \emph{$\mathscr{P}$-connected} if $\top\in \mathscr{P}(C)$ is connected.
\end{definition}

\begin{definition}
	Given a preorder $(I, \rho)$ in an elementary existential doctrine $\mathscr{P}$ and $\alpha\in \mathscr{P}(I)$ we define the \emph{downward} and \emph{upward closure of $\alpha$} as
	\begin{equation*}
	\downarrow \alpha:= \exists_{\pi_1}(\mathscr{P}_{\pi_2}(\alpha)\wedge \rho )\qquad \uparrow \alpha:= \exists_{\pi_2}(\mathscr{P}_{\pi_1}(\alpha)\wedge \rho )
	\end{equation*} 
	We define the \emph{reachability operator} $\reach{}$ as the family of functions, indexed over the objects: 
	\begin{align*}
	\reach{C}:\mathscr{P}(C)&\rightarrow \mathscr{P}(C)\\
	\varphi &\mapsto \varphi \vee \bigvee_{p\in \catname{C}(I,C)}\exists_{p}(\uparrow\mathscr{P}_{p}(\varphi)).
	\end{align*}
\end{definition}

\begin{theoremEnd}{prop}
	Given $\mathscr{P}:\catname{C}^{op}\rightarrow \catname{InfSL}$ in $\catname{EED}$ and $(I,\rho)$ an internal preorder in it, $\reach{}=\{\reach{C}\}_{C\in \catname{Ob}(\catname{C})}$  is a grounded closure operator on $\mathscr{P}$. If, moreover, $\mathscr{P}$ is an hyperdoctrine, $\reach{}$ is fully additive.
\end{theoremEnd}
\begin{proofEnd}
	Monotonicity and inflationarity comes at once, take an arrow $f:C\rightarrow D$, for any $\alpha \in \mathscr{P}(D)$ we have:
	\begin{align*}
	\exists_f(\reach{C}(\mathscr{P}_f(\varphi)))&=\exists_f(\mathscr{P}_f(\varphi))\vee \bigvee_{p\in \catname{C}(I,C)}\exists_f(\exists_p(\uparrow \mathscr{P}_p(\varphi)))
	\\&=\exists_f(\mathscr{P}_f(\varphi))\vee \bigvee_{p\in \catname{C}(I,C)}\exists_f(\exists_{p}(\exists_{\pi_2}(\rho \wedge \mathscr{P}_{\pi_1}(\mathscr{P}_{p}(\mathscr{P}_{f}(\varphi))))))\\&\leq
	\varphi \vee \bigvee_{p\in \catname{C}(I,C)} \exists_{f\circ p}(\exists_{p}(\exists_{\pi_2}(\rho \wedge \mathscr{P}_{\pi_1}(\mathscr{P}_{f\circ p}(\varphi))))\\&\leq \varphi \vee \bigvee_{q\in \catname{C}(I,D)} \exists_{q}(\exists_{\pi_2}(\rho \wedge \mathscr{P}_{\pi_1}(\mathscr{P}_{q}(\varphi))))\\&=\varphi \vee \bigvee_{q\in \catname{C}(I,D)}\exists_q(\uparrow \mathscr{P}_q(\varphi)) \\&=\reach{D}(\varphi)
	\end{align*}
	Groundedness is immediate; suppose now that $\mathscr{P}$ is an hyperdoctrine, then $\mathscr{P}_f$ commutes with suprema for any arrow $f$ and, since $\mathscr{P}(C)$ is an Heyting algebra, infima distribute over them, so:
	\begin{align*}
	\reach{C}(\bigvee_{k\in K}\varphi_k)&= (\bigvee_{k\in K}\varphi_K )\vee \bigvee_{p\in \catname{C}(I,C)} \exists_{p}(\exists_{\pi_1}(\rho \wedge \mathscr{P}_{\pi_2}(\mathscr{P}_p(\bigvee_{k\in K}\varphi_k))))\\&=(\bigvee_{k\in K}\varphi_k)\vee \bigvee_{p\in \catname{C}(I,C)}\bigvee_{k\in K}\exists_{p}(\exists_{\pi_1}(\rho \wedge \mathscr{P}_{\pi_2}(\varphi_k)))\\&=\bigvee_{k\in K}(\varphi_k\vee \bigvee_{p\in \catname{C}(I,C)}\exists_{p}(\exists_{\pi_1}(\rho \wedge \mathscr{P}_{\pi_2}(\varphi_k))))\\&= \bigvee_{k\in K} \reach{C}(\varphi_k)
	\end{align*}
	\iffalse 
	\begin{align*}
	\reach{C}(\bot)&=\bot \vee \bigvee_{p\in \catname{C}(I,C)}\exists_p(\exists_{\pi_2})(\rho \wedge \mathscr{P}_{\pi_1}(\mathscr{P}_p(\bot)))\\&=\bigvee_{p\in \catname{C}(I,C)}\exists_p(\exists_{\pi_2}(\bot))\\&=\bot
	\end{align*}
	\fi 
\end{proofEnd}

\begin{example}
	In $\catname{Set}$, for any non empty $I$ we have that, for any set $X$:
	\begin{gather*}
	\reach{X}(S)=\begin{cases}
	X & S\neq \emptyset\\
	\emptyset & S=\emptyset
	\end{cases}
	\end{gather*}  
\end{example}

\begin{example}
	Take the elementary hyperdoctrine $\mathscr{P}^p$ on $\catname{PrTop}$ (\cref{def:topclosures}) and fix an $n\in \mathbb{N}$, as an internal order we can take $n=\{0,1,...,n-1\}$ with the closure operator
	\begin{align*}
	\mathfrak{n}:\mathcal{P}(n)&\rightarrow \mathcal{P}(n)\\
	S & \mapsto \{i\in n \mid n=s+1 \text{ for some } s \in S\} 
	\end{align*} and the usual ordering $\leq$ as $\rho$. An arrow $p:(I, \mathfrak{n})\rightarrow (X, \mathfrak{c})$ is just a function such that
	\begin{equation*}
	\mathfrak{n}(p^{-1}(S))\leq p^{-1}(\mathfrak{c}(S))
	\end{equation*}
	that is, $p(i+1)\in \mathfrak{c}(\{p(i)\})$.
	So, for instance
	\begin{equation*}
	\reach{(\mathbb{N}, \mathfrak{n})}(S)=\{k \in \mathbb{N}\mid k=s+n \text{ for some } s\in S \}
	\end{equation*}
	where $\mathsf{n}:\mathcal{P}(\mathbb{N})\rightarrow \mathcal{P}(\mathbb{N})$ is defined as for $n$.
\end{example}

\subsection{Surroundedness}
In this section we will introduce a \emph{surrounded} operator (similar to the ``until'' operator of temporal logic) in order to generalize the analogous operator introduced in \cite{ciancia2016spatial}. 
\begin{definition}
	Let $(I,\rho)$ be an internal order in an elementary existential closure doctrine $(\mathscr{P}, \mathfrak{c})$, let $\phi$ and $\psi\in \mathscr{P}(C)$.
	We say that an arrow $p:I\rightarrow C$ is an \emph{escape route from $\phi$ avoiding $\psi$} if
	\begin{enumerate}
		\item at some point in $p$, $\phi$ holds: $\exists_{!_I}(\mathscr{P}_p(\phi))=\top$;
		\item from the points where $\phi$ holds we can reach a point where $\neg\phi$ holds: $\mathscr{P}_p(\phi)\leq {\downarrow}\mathscr{P}_p(\neg \phi)$;
		%\item $\uparrow\mathscr{P}_p(\phi)\wedge \mathscr{P}_p(\neg\phi) \neq \bot$;
		\item there is no point reachable from $\phi$ and which reaches $\neg\phi$ along the route, where $\psi$ holds:
		$\uparrow\mathscr{P}_p(\phi)\wedge \downarrow \mathscr{P}_p(\neg \phi)\wedge \mathscr{P}_p(\psi)=\bot$.
	\end{enumerate}
	We will denote with $\er{C}{\phi}{\psi}$ the set of such arrows. We also define
	\begin{equation*}
	\escape{C}{\phi}{\psi} :=\bigvee_{p\in \er{C}{\phi}{\psi}}  \exists_{p}(\top)
	\qquad\qquad
	\sur{C}{\phi}{\psi}:=\bigwedge_{p\in \er{C}{\phi}{\psi}}\phi \wedge \neg(\exists_p(\top))
	\end{equation*}
\end{definition}
Intuitively, $\escape{C}{\phi}{\psi}$ (read ``$\phi$ escapes $\psi$'') holds where $\phi$ holds and it is possible to escape avoiding $\psi$;
conversely, $\sur{C}{\phi}{\psi}$  (read ``$\phi$ is surrounded by $\psi$'') holds where $\phi$ holds and it is not possible to escape from it without avoiding $\psi$. 
Notice that these notions depend on the specific choice of the internal order $(I,\rho)$, hence we can deal with different reachability, with different shapes of escape routes, by choosing the adequate internal order.
% this can be linear, partial, total, branching, etc..


\begin{example}[cfr.~\cite{ciancia2016spatial}]
	Let us  consider the closure hyperdoctrine on pretopological spaces $(\mathscr{P}^p, c)$ as in \cref{def:topclosures}. In this case an internal order is just an ordered set $(I,\leq)$ equipped with a closure operator. Given $S$ and $T$ subsets of a chosen $(X, \mathfrak{c})$, then
	\begin{itemize}
		\item $p\in \er{(X, \mathfrak{c})}{S}{T}$ if and only if
		\begin{enumerate}
			\item $p^{-1}(S)\neq \emptyset$;
			\item for any $t$ such that $p(t)\in S$ there exists an $s\geq t$ with $p(s)\notin S$; 
			\item  $p(t)\notin T$ for any $t\in I$ for which there exist $s$ and $v\in I$ such that $p(s)\in A$, $p(v)\in T $ and $s \leq t \leq v$.
		\end{enumerate}
		\item   $x \in 	\escape{(X, \mathfrak{c})}{S}{T} $ if and only if there exists a continuous $p:I\rightarrow X, t, s \in I$ such that $t\leq s$, $p(t)=x$, $p(s)\notin S$ and for any pair $(u,v)\in \leq$ with $p(u)\in S$ and $p(v)\in T$ there are no $w$ between $u$ and $v$ such that $p(w)\in T$.
		\item $x\in \sur{(X, \mathfrak{c})}{S}{T}$ if and only if $x\in S$  and for any continuous $p:I\rightarrow X$ such that $p(t)=x$ for some $t\in I$, $p\notin \er{(X, \mathfrak{c})}{S}{T}$.
	\end{itemize}   
Therefore, this situation corresponds  to the surround operator defined in   \cite{ciancia2016spatial}.
\end{example}

\begin{theoremEnd}{thm1}
	Let $(\mathscr{P}, \mathfrak{c})$ be a boolean elementary closure hyperdoctrine, $(I, \rho)$ a preorder in it with $I$ $\mathscr{P}$-connected and such that, for all $\gamma\in \mathscr{P}(I)$,
	$\mathfrak{c}_I(\gamma)\wedge\neg \gamma \leq \uparrow \gamma$.
	Then, for any $\phi$ and $\psi \in \mathscr{P}(C)$:
	\begin{enumerate}
		\item if $\alpha\in \mathfrak{u}_{C}(\phi, \psi)$  and $p\in \er{C}{\phi}{\psi}$ then $\mathscr{P}_p(\alpha)=\bot$;
		\item $\phi \mathfrak{U}_C \psi \leq \sur{C}{\phi}{\psi}$.
	\end{enumerate}
\end{theoremEnd}

\begin{proofEnd}
	\begin{enumerate}
		\item By continuity we have
		\begin{align*}
		\mathfrak{c}_I(\mathscr{P}_p(\alpha))\wedge \mathscr{P}_p(\neg \alpha)&\leq \mathscr{P}_p(\mathfrak{c}_C(\alpha)) \wedge \mathscr{P}_p(\neg \alpha)\\&\leq \mathscr{P}_p(\mathfrak{c}_C(\alpha)\wedge \neg \alpha)\\&\leq		
		\mathscr{P}_p(\psi)
		\end{align*}
		By hypothesis,
		\begin{align*}
		\mathfrak{c}_I(\mathscr{P}_p(\alpha))\wedge \mathscr{P}_p(\neg \alpha)&\leq \uparrow\mathscr{P}_p(\alpha)\\&\leq \mathscr{P}_p(\phi)
		\end{align*} 
		and 
		\begin{align*}
		\mathfrak{c}_I(\mathscr{P}_p(\alpha))\wedge \mathscr{P}_p(\neg \alpha)&=\mathfrak{c}_I(\mathscr{P}_p(\alpha))\wedge \mathscr{P}_p(\neg \alpha)\wedge \top\\&=(\mathfrak{c}_I(\mathscr{P}_p(\alpha))\wedge \mathscr{P}_p(\neg \alpha)\wedge \mathscr{P}_{p}(\phi) )\vee(\mathfrak{c}_I(\mathscr{P}_p(\alpha))\wedge \mathscr{P}_p(\neg \alpha)\wedge \mathscr{P}_p(\neg \phi))\\&\leq\downarrow\mathscr{P}_p(\neg \phi)
		\end{align*}
		hence, since
		$p\in \er{C}{\phi}{\psi}$:
		\begin{align*}
		\mathfrak{c}_I(\mathscr{P}_p(\alpha))\wedge \mathscr{P}_p(\neg \alpha)&\leq \uparrow\mathscr{P}_p(\phi)\wedge \downarrow \mathscr{P}_p(\neg \phi)\wedge \mathscr{P}_p(\psi)\\&=\bot
		\end{align*}	
		and we conclude by connectedness.	
		\item By the previous point $\mathscr{P}_p(\alpha)=\bot$ for any $p\in \er{C}{\phi}{\psi}$ so $\mathscr{P}_p(\neg \alpha)=\top$ that implies $\alpha \leq \neg \exists_p(\top)$ from which the thesis follows.
		\qedhere
	\end{enumerate}
\end{proofEnd}

\iffalse 
\begin{definition}
	Let $(I, \rho)$ be an internal order in $(\mathscr{P}, \mathfrak{c})$, we say that it has the \emph{successor property} if
\end{definition}
\begin{theorem}
	If $(I,\rho)$ has the successor property then 
	\begin{equation*}
	\sur{C}{\phi}{\psi}=\phi \mathfrak{U}_C \psi
	\end{equation*}
\end{theorem}
\begin{proof}
\end{proof} 

\subsection{Internal orders}
\begin{definition}
	Let $\mathscr{P}:\catname{C}^{op}\rightarrow \catname{HA}$ be an hyperdoctrine, an \emph{internal order} in $\mathscr{P}$ is a pair $(I, \rho)$ where $I$ is an object of $\catname{C}$ and $\rho\in \mathscr{P}(I\times I)$ such that
	\begin{itemize}
		\item $\rho$ is \emph{reflexive}: \begin{equation*}
		\delta_I \leq \rho
		\end{equation*}
		\item $\rho $ is \emph{transitive}
		\begin{equation*}
		\mathscr{P}_{(\pi_1,\pi_2)}(\rho)\wedge\mathscr{P}_{(\pi_2,\pi_3)}(\rho)\leq \mathscr{P}_{(\pi_1,\pi_3)}(\rho)
		\end{equation*}
		\item  $\rho $ is \emph{antisymmetric}
		\begin{equation*}
		\rho\wedge\mathscr{P}_{(\pi_2,\pi_1)}(\rho) \leq \delta_I
		\end{equation*}
	\end{itemize}
	An internal order $(I, \rho)$ is \emph{total} if in addition
	\begin{equation*}
	\rho \vee \mathscr{P}_{(\pi_2,\pi_1)} (\rho) = \top
	\end{equation*}
	A \emph{internal monotone arrow} $f:(I,\rho)\rightarrow (J, \sigma)$ is an arrow of $\catname{C}$ such that
	\begin{equation*}
	\rho \leq \mathscr{P}_{f\times f}(\sigma)
	\end{equation*}
\end{definition}
\begin{proposition}\label{moror}
	contenuto...
\end{proposition}
\begin{proof}
	contenuto...
\end{proof}
\begin{proposition}
	Internal orders and monotone arrows form a category $\ipos{C}$, moreover the obvious forgetful functor $\ipos{C}\rightarrow \catname{C}$ is faithful and has a left adjoint.
\end{proposition}
\begin{proof}
	Identities are clearly internally monotone, if we have $f:(I, \rho)\rightarrow (J, \sigma)$ and $g:(J, \sigma)\rightarrow (K, \tau)$ then
	\begin{align*}
	\rho &\leq \mathscr{P}_{f\times f}(\sigma )\\&\leq\mathscr{P}_{f\times f}(\mathscr{P}_{g\times g}(\tau ) )\\&=\mathscr{P}_{(g\circ f) \times (g\circ f)}(\tau)
	\end{align*}
	Faithfulness of is clear, let's construct its left adjoint. Since
	\begin{equation*}
	\delta_I\leq \mathscr{P}_{f\times f}(\delta_J) 
	\end{equation*}
	for any arrow $f:I\rightarrow J$ in $\catname{C}$, we can define
	\begin{gather*}
	\catname{C}\rightarrow \ipos{C}\\
	\functor[l]{I}{f}{J}
	\functormapsto
	\functor[r]{(I, \delta_I)}{f}{(I, \delta_J)} 
	\end{gather*}
	Now, if $(J, \rho)$ is a preorder, and $f:I\rightarrow J$ an arrow in $\catname{C}$ we have
	\begin{align*}
	\delta_I &\leq \mathscr{P}_{f\times f}(\delta_J)\\&\leq \mathscr{P}_{f\times f}(\rho)
	\end{align*}
	So $f$ itself is the unique arrow $f:(I, \delta_I)\rightarrow (J, \rho)$ such that
	\begin{center}
		\begin{tikzpicture}
		\node(A)at(0,0){$I$};
		\node(B)at(2,0){$I$};			\node(C)at(2,-1.5){$\mathscr{U}(J,\rho)$};
		\draw[->](A)--(B) node[pos=0.5, above]{$1_I$};
		\draw[->](B)--(C) node[pos=0.5, right]{$\mathscr{U}(f)$};
		\draw[->](A)--(C) node[pos=0.5, below, xshift=-0.15cm, yshift=0.05cm]{$f$};
		\end{tikzpicture}
	\end{center}
	commutes, thus we're done.
\end{proof}
\begin{remark}
	The ordinary category $\catname{Pos}$ of posets is nothing more than $\ipos{Set}$.
\end{remark}
\subsubsection{The surrounded and propagation operators}
\begin{definition}Let $\catname{C}$ be a category, $(I, \rho)$ be an internal order in it, such that the right adjoint $(-)^I$ to $(-)\times I$ exists, then for any object $C$  we define the \emph{$(I,\rho)$-shaped surrounded operator $\Su{(I,\rho)}{C}$} and the \emph{$(I,\rho)$-shaped propagation operator} as the functions $\mathscr{P}(C)\times \mathscr{P}(C)\rightarrow \mathscr{P}(C)$ 
	such that	
	\begin{gather*}		
	\begin{split}
	\Su{(I,\rho)}{C}(\phi, \psi):=& \phi \wedge \forall_{\pi_C}((\mathscr{P}_{(\pi_C, ev\circ (\pi_{C^I}, \pi_1))}(\delta_C)\wedge \mathscr{P}_{(\pi_1,\pi_2)}(\rho))\Rightarrow (\mathscr{P}_{(\pi_C, ev\circ (\pi_{C^I}, \pi_2))}(\phi) \vee \\&\vee
	\exists_{(p_C, p_{C^I}, p_1, p_2)}(\mathscr{P}_{(p_1, p_3)}(\rho)\wedge\mathscr{P}_{(p_3, p_2)}(\rho) \wedge \mathscr{P}_{(p_C, ev\circ (p_{C^I}, p_3))}(\psi) )))
	\end{split}
	\end{gather*}
	where $\pi_{C^I}$, $\pi_C$, $\pi_1$, $\pi_2$ and $\pi_3$ are the projections from  $ C\times C^I\times I\times I $ while $p_{C^I}$,$p_C$, $p_1$, $p_2$ and $p_3$ those from $C\times  C^I\times I \times I\times I$.
\end{definition}


\begin{proposition}
	Let $\mathfrak{P}:\catname{C}^{op}\times \ipos{C}^{op}\rightarrow \catname{Pos}$ be the functor
	\begin{gather*}
	\functor[l]{(C, (I,\rho))}{(f,g)}{(D, (J, \sigma))}
	\functormapsto
	\rfunctorop{ \mathscr{P}( C\times I)\times \mathscr{P}( C \times I)^{op}}{\mathscr{P}_{f\times g}\times\mathscr{P}_{f\times g}^{op}}{ \mathscr{P}( D \times J)\times \mathscr{P}( D \times J)^{op}} 
	\end{gather*}
	and let $\mathscr{V}$ be the forgetful functor $\catname{HA}\rightarrow \catname{Pos}$, then there exists a oplax natural transformation 
	\begin{equation*}
	\su{}{}:\mathfrak{P}\rightarrow \mathscr{V}(\mathscr{P}(-\times \mathscr{D}(-)))
	\end{equation*}
	with $	\su{(I,\rho)}{C}$ as component in $(C, (I, \rho))$. 
\end{proposition}
\begin{proof}
	Monotonicity of each $	\su{(I,\rho)}{C}$ is clear, we have to check oplax naturality, i.e. that
	\begin{equation*}
	\su{(I,\rho)}{C}\circ \mathscr{P}_{f\times g}\times\mathscr{P}_{f\times g}^{op} \leq \mathscr{P}_{f\times g}\circ \su{(J,\sigma)}{D}
	\end{equation*}
	commutativity of
	\begin{center}
		\begin{tikzpicture}
		\node(A)at(0,0){$\mathscr{P}( D \times J)\times \mathscr{P}( D \times J)^{op} $};
		\node(B)at(5,0){$\mathscr{P}(D\times J)$};			\node(C)at(0,-2){$\mathscr{P}( C\times I)\times \mathscr{P}( C \times I)^{op}$};
		\node(D)at(5,-2){$\mathscr{P}(C\times I)$};
		\draw[->](A)--(B) node[pos=0.5, above]{$\su{(J,\sigma)}{D}$};
		\draw[->](C)--(D) node[pos=0.5, below]{$\su{(I,\rho)}{C}$};
		\draw[->](B)--(D) node[pos=0.5, right]{$\mathscr{P}_{f\times g}$};
		\draw[->](A)--(C) node[pos=0.5, left]{$\mathscr{P}_{f\times g}\times\mathscr{P}_{f\times g}^{op}$};
		\end{tikzpicture}
	\end{center}
	
	Let's compute
	\begin{align*}
	&\su{(I,\rho)}{C}(\mathscr{P}_{f\times g}\times\mathscr{P}_{f\times g}^{op}(\phi, \psi))=\su{(I,\rho)}{C}(\mathscr{P}_{f\times g}(\phi), \mathscr{P}_{f\times g}(\psi))
	\\&=
	\mathscr{P}_{f\times g}(\phi)\wedge \forall_{p_2}	((\rho \wedge \neg \mathscr{P}_{p_2}(\mathscr{P}_{f\times g}(\phi)))\Rightarrow \exists_{(\pi_1,\pi_3)}(\mathscr{P}_{(\pi_1,\pi_3)}(\rho)\wedge \mathscr{P}_{(\pi_1,\pi_2)}(\rho) \wedge \mathscr{P}_{\pi_2}(\mathscr{P}_{f\times g}(\psi))) )
	\\&\leq\mathscr{P}_{f\times g}(\phi)\wedge \forall_{p_2}	((\mathscr{P}_{g\times g}(\sigma ) \wedge \neg \mathscr{P}_{p_2}(\mathscr{P}_{f\times g}(\phi)))\Rightarrow \exists_{(\pi_1,\pi_3)}(\mathscr{P}_{(\pi_1,\pi_3)}(\rho)\wedge \mathscr{P}_{(\pi_1,\pi_2)}(\rho) \wedge \mathscr{P}_{\pi_2}(\mathscr{P}_{f\times g}(\psi))) )
	\end{align*}
	FINIRE I CONTI
\end{proof}
\begin{definition}
	Let $\mathscr{P}:\catname{C}^{op}\rightarrow \catname{HA}$ be a preclosure hyperdoctrine, and $\phi\in \mathscr{P}(C)$, we call it
	\begin{itemize}
		\item \emph{inhabited} if $\exists_{!_C}(\phi)=\top$ in $\mathscr{P}(1)$;
		\item \emph{connected} if $\mathfrak{c}(\psi)\wedge \varphi$ and $\mathfrak{c}(\varphi)\wedge \psi$ are inhabited for any $\psi$ and $\varphi$ such that $\varphi \vee \psi = \phi$.
	\end{itemize}
	An object $C$ is \emph{$\mathscr{P}$-connected} if $\top\in \mathscr{P}(C)$ is connected.
\end{definition}

\begin{theorem}
	Let $(I, \rho)$ be a partial order in $\mathscr{P}$ such that 
\end{theorem}
\begin{proof}
\end{proof} 

\fi 


\section{Logics for closure hyperdoctrines with paths}\label{sec:slcswp}
In this section we extend the logics for closure hyperdoctrines we have introduced in \cref{sec:slcs}, with formulae constructor for reasoning about sourroundedness and reachability.

\subsection{Syntax and derivation rules}
\begin{definition}
	A \emph{signature with paths} is a triple $\Sigma=(\Sigma, \iota, R)$ where
	\begin{itemize}
		\item $\Sigma$ is a signature as per \cref{def:signature};
		\item $\iota \in \abs{\Sigma}$ is called the \emph{interval type};
		\item $R: \iota, \iota$ is called the \emph{preorder of $\iota$}
	\end{itemize}
	A morphism  $\phi:(\Sigma_1, \iota_1, R_1)\rightarrow (\Sigma_2, \iota_2, R_2)$ is a morphism of signature  $(\phi_1,\phi_2,\phi_3)$ such that $\phi_1(\iota_1)=\iota_2$ and ${\phi_3}_{\iota, \iota}(R_1)=R_2$.
\end{definition}
\begin{remark}
	Signatures with paths and their morphisms with componentwise composition form a category $\catname{SignPath}$.
\end{remark}

\begin{definition}
	We add the following rule of well formation to the logic for the closure operators (\cref{def:slcssynt}):
	\begin{equation*}
	\inferrule*[right=$\mathcal{S}$-F]{\Gamma \vdash \phi:\propo \\
		\Gamma\vdash \psi:\propo}{\Gamma  \vdash \phi \mathcal{S}\psi:\propo}
	\end{equation*}
	
\end{definition}

\begin{definition}
	Given a signature $(\Sigma, \iota, R)$, its \emph{classifying category} is the category $\class{\Sigma, \iota, R}$ is just $\class{\Sigma}$.  
\end{definition}

	
\begin{definition}
	We define the following rules for the well-formed formulae previously defined:
	\begin{itemize}
		\item $R$'s rules:
		\begin{gather*}	
		\inferrule*[right=$R$-Refl]{\Gamma, x:\iota, y:\iota \mid \Phi\vdash x=_\iota y}{\Gamma, x:\iota, y:\iota \mid \Phi\vdash R(x,y)} \\
		%	\inferrule*[right=$R$-Anti]{\Gamma, x:\iota, y:\iota \mid \Phi\vdash R(x,y)\\\Gamma, x:\iota, y:\iota \mid \Phi\vdash R(y,x)}{\Gamma, x:\iota, y:\iota \mid \Phi\vdash x=_\iota y } \\
		\inferrule*[right=$R$-Trans]{\Gamma, x:\iota, y:\iota \mid \Phi\vdash R(x,y)\\\Gamma, y:\iota, z:\iota \mid \Phi\vdash R(y,z)}{\Gamma, x:\iota, z:\iota \mid \Phi\vdash R(x,z)}
		\end{gather*}
		
		\item $\mathcal{S}$'s rules: 
		\begin{gather*}
		%	\inferrule*[right=$\mathcal{S}$-I]{\{\Gamma, x:\sigma  \mid \Phi, \varphi \vdash \phi(x) \wedge \neg \exists t:\iota.(x=_\sigma p(t))\}_{p\in \esca{\Gamma}{\phi}{\psi}{\Phi}} }{\Gamma, x:\sigma \mid \Phi, \varphi \vdash 	\phi\mathcal{S}\psi}  
		\inferrule*[right=$\mathcal{S}$-I]{\text{for all } p:\iota \rightarrow \sigma,\quad  \Gamma, x:\sigma  \mid \Phi, \varphi \vdash \es{\Gamma,x:\sigma}{\phi}{\psi}\wedge \phi \wedge \neg \exists t:\iota.(x=_\sigma p(t))
			  %\Gamma, x:\sigma \vdash \varphi:\propo\}%{p\in \esca{\Gamma}{\phi}{\psi}{\Phi}}
		}{\Gamma, x:\sigma \mid \Phi, \varphi \vdash 	\phi\mathcal{S}\psi}  	
		\\
		\inferrule*[right=$\mathcal{S}$-E]{\Gamma, x:\sigma \mid \Phi \vdash \es{\Gamma,x:\sigma}{\phi}{\psi}}{\Gamma, x:\sigma  \mid  \Phi,\phi\mathcal{S}\psi  \vdash \phi \wedge \neg \exists t:\iota.(x=_\sigma p(t))}  
		\end{gather*} 
		where
		\begin{align*}
		\es{\Gamma,x:\sigma}{\phi}{\psi} :=\ & (\exists t:\iota.\phi[p(t)/x])\ \wedge \\
		& (\phi[p(t)/x] \Rightarrow \exists s:\iota.(R(t, s)\wedge \neg \phi[p(s)/x]))\ \wedge \\
		 & \neg (\exists s:\iota.(R(s,t)\wedge \phi[p(s)/x])\wedge \exists v:\iota.(R(t,v)\wedge \neg \phi[p(v)/x]) \wedge \psi[p(t)/x])
		\end{align*}
		
		%\begin{equation*}
		%\esca{\Gamma}{\phi}{\psi}{\Phi}:=\{p\in BOH \mid \Gamma, x:\sigma \mid \Phi \vdash \es{\Gamma}{\phi}{\psi} \text{ is derivable} \}
		%\end{equation*}
	\end{itemize}

	The \emph{Propositional Logic for Closure Operators with Paths on $\Sigma$} (PLCOwP) is given by PLCO (\cref{def:slcsrules}) extended with the rules above.
	Similarly for the \emph{Regular Logic for Closure Operators with Paths on $\Sigma$} (RLCOwP) and
	the \emph{First Order Logic for Closure Operators with Paths on $\Sigma$} (FOLCOwP).
	
	Derivability of sequents is defined in the usual way (\cite{pitts1995categorical}).
\end{definition}
%\begin{remark}
%PLCOwP corresponds to the Spatial Logic for Closure Spaces considered in \cite{ciancia2016spatial}.
%\end{remark}

\subsection{Categorical semantics of closure logics with paths}
\begin{definition} 
	Given an elementary closure hyperdoctrine $(\mathscr{P}, \mathfrak{c}):\catname{C}^{op}\rightarrow \catname{HA}$ and an internal preorder $(I, \rho)$, we will call the pair $((\mathscr{P}, \mathfrak{c}), (I,\rho))$ an \emph{elementary path hyperdoctrine}.
	An arrow of path hyperdoctrines $((\mathscr{P}, \mathfrak{c}), (I,\rho))\rightarrow ((\mathscr{S}, \mathfrak{d}), (J,\sigma))$ is a morphism $(\mathscr{F}, \eta )\in \catname{cEHD}((\mathscr{P}, \mathfrak{c}), (\mathscr{S}, \mathfrak{d}) )$ such that there exists an isomorphism $h:\mathscr{F}(I)\rightarrow J$ for which
	\begin{equation*}
	\eta_{I\times I}(\rho)=\mathscr{S}_{(h\circ \mathscr{F}(\pi_1), h\circ \mathscr{F}(\pi_2) )}(\sigma)
	\end{equation*}  
	We say that $(\mathscr{F}, \eta)$ is \emph{open} if it is as arrow $(\mathscr{P}, \mathfrak{c})\rightarrow (\mathscr{S}, \mathfrak{d})$
	
	Clearly this defines a $2$-subcategory $\catname{pEHD}$ of $\catname{cEHD}$.
\end{definition}
\begin{proposition}For any signature $(\Sigma, \iota, R)$, $(\lind{\Sigma}, (\iota, R))$ is a path hyperdoctrine.
\end{proposition}
\begin{proof} 
	We have only to show that $(\iota, R)$ is an internal preorder but this follows at once from the two $R$'s rules.
\end{proof}


\begin{definition}
	Let $((\mathscr{P},\mathfrak{c})(I,\rho))$ be a path hyperdoctrine.
	Then, a \emph{model of closure logic with paths} in it is just an open  morphism
	\begin{equation*}
	(\mathscr{M}, \mu):((\lind{\Sigma}, \mathfrak{c}_\Sigma), (\iota, R)\rightarrow ((\mathscr{P},\mathfrak{c}),(I,\rho))
	\end{equation*}
	Satisfability of sequents is defined as in the case of closure logics (\cref{def:slcsmodel}).
\end{definition}
\begin{remark}
	As for $\mathcal{U}$ we have not put any requirement on the interpretation of $\mathcal{S}$, but, in $(\lind{\Sigma}, \mathfrak{c}_\Sigma)$, for $\Gamma \vdash \phi:\propo$ and $\Gamma \vdash \psi:\propo$ we have
	\begin{equation*}
	[\phi\mathcal{S}\psi]=[\phi]\mathfrak{S}_\Gamma[\psi]
	\end{equation*}
	so we can again ask for \emph{continuous models}, i.e. models that preserves this equality.
\end{remark}
\iffalse 
\begin{proposition}
	For any pair of formulae $\phi$ and $\psi$ such that $\Gamma \vdash \phi:\propo$ and $\Gamma \vdash \psi: \propo$ and any model $(\mathscr{M}, \mu)$ the equalities
	\begin{equation*}
	\mu_{\Gamma}([\phi\mathcal{S}\psi])=\Su{(I,\rho)}{\mathscr{M}(\Gamma)}(\mu_{\Gamma}([\phi]),\mu_{\Gamma}([\psi]))\qquad 	\mu_{\Gamma}([\phi\mathcal{P}\psi])=\Pro{(I,\rho)}{\mathscr{M}(\Gamma)}(\mu_{\Gamma}([\phi]),\mu_{\Gamma}([\psi]))
	\end{equation*}
	hold.
\end{proposition}
\begin{proof}
	It is enough to notice that
	\begin{equation*}
	\begin{split}
	\mu_{\Gamma}([\phi\mathcal{S}\psi])&=	\mu_{\Gamma}([\surr(\phi, \psi)])\\&=\Su{(I,\rho)}{\mathscr{M}(\Gamma)}(\mu_{\Gamma}([\phi]),\mu_{\Gamma}([\psi]))
	\end{split}
	\begin{split}
	\mu_{\Gamma}([\phi\mathcal{P}\psi])&=	\mu_{\Gamma}([\propag(\phi, \psi)])\\&=\Pro{(I,\rho)}{\mathscr{M}(\Gamma)}(\mu_{\Gamma}([\phi]),\mu_{\Gamma}([\psi]))
	\end{split}
	\end{equation*}
\end{proof}
\fi
\begin{theorem}
	A sequent $\Gamma \mid \Phi\vdash \psi$ is satisfied by the \emph{generic model} $(1_{\class{\Sigma}}, 1_{\lind{\Sigma}})$ if and only if it is derivable.
\end{theorem}
\begin{proof}
	The proof is the same as for \cref{th:completeness}.
\end{proof}
	
\begin{corollary}
		The above defined categorical semantics for PLCOwP/RLCOwP/FOLCOwP (with or without equality) is sound and complete.
\end{corollary}

\fi

\section{Conclusions and future work}
\label{sec:concl}
In this paper we have introduced \emph{closure (hyper)doctrines} as a theoretical framework for studying the logical aspects of closure spaces.
First we have proved the generality of this notion by means of a wide range of examples arising naturally from topological spaces, fuzzy sets, algebraic structures, coalgebras, and covering at once also known cases such as Kripke frames and probabilistic frames.
Then, we have applied this framework to provide the first axiomatisation and sound and complete categorical semantics for various fragments of a logics for closure doctrines. In particular, the propositional fragment corresponds to the Spatial Logic for Closure Spaces \cite{ciancia2014specifying}, a modal logic for the specification and verification on spatial properties over preclosure spaces. But the flexibility of our approach allows us to readily obtain closure logics for a wide range of cases (including all the examples presented above).

\ifreport
Finally, we have extended closure hyperdoctrines with a notion of \emph{paths}. This allows us to provide sound and complete logical derivation rules also for the ``surroundedness'' operator, thus covering all the logical constructs of SLCS.
\fi

%The categorical approach we have followed in this paper is very general, as it can be applied for modeling logics with different kinds of modalities, such as temporal, epistemic, etc. \cite{awodey2014topos, schultz2019temporal}.
Albeit already quite general, the theory presented in this paper paves the way for several extensions. 
\ifreport
\else
Due to lack of space, we have not been able to present the constructions for modeling logical operators concerning \emph{surroundedness}. To this end, we need to endow doctrines with an object representing the ``type of paths''; for more details we refer to the extended version of this work \cite{cm:closurehyperdoctrines-extended}.

\fi
We can enrich the logic with other spatial modalities, e.g., the spatial counterparts of the various temporal modalities of CTL* \cite{emerson1986sometimes}.
It could be interesting to investigate a spatial logic with fixed points \emph{a la} $\mu$-calculus; to interpret such a logic, we could consider closure hyperdoctrines over Löb algebras. 
%  Penso che si possa introdurre anche la questione dei punti fissi o della guarded recursion notando che la questione diventa trovare operatori di prechiusura interessanti su una Lob algebra (per i punti fissi) o sul frame dei downward closed su set di N (per la ricorsione). 
Moreover, it would be interesting to develop some ``generic'' model checking algorithm for spatial logic. The abstraction provided by the categorical approach can guide the generalization of existing model checking algorithms, such as \cite{ciancia2014specifying}, and suggest new proof methodologies and minimisation techniques.

On a different direction, we are interested in the type theory induced by closure hyperdoctrines.
A Curry-Howard isomorphism would yield a functional programming language with constructors for spatial aspects, which would be very useful in \emph{collective spatial programming}, e.g. for collective adaptive systems.

	
%% Bibliography
%\bibliography{bibliog}
	
\clearpage
%% Appendix
\appendix


\section{Omitted proofs}\label{sec:proofs}
\printProofs
\end{document}
